%! TEX root = ./main.tex
\documentclass[12pt]{article}

%--------Packages-------------
\usepackage{kyrem1sty}
%----------------------------


%--------Bibliography---------
%\usepackage[backend=biber,style=alphabetic,doi=false,isbn=false,url=false,eprint=false]{biblatex}
%\addbibresource{INSERT .BIB PATH}
%----------------------------


%--------Hyper Setup-------
\hypersetup{%
  colorlinks=true,%
  linkcolor=blue,%
  citecolor=blue,%
  filecolor=blue,%
  menucolor=blue,%
  urlcolor=blue,%
  pdfnewwindow=true,%
  pdfstartview=FitBH
}   
%----------------------------


%--------Subfiles Setup-------
%\usepackage{subfiles}
%----------------------------


%--------Page Setup-----------
%\usepackage{geometry}\geometry{margin=1in}
\pagestyle{empty}%

\setlength{\hoffset}{-1.54cm}
\setlength{\voffset}{-1.54cm}

\setlength{\topmargin}{0pt}
\setlength{\headsep}{0pt}
\setlength{\headheight}{0pt}

\setlength{\oddsidemargin}{0pt}

\setlength{\textwidth}{195mm}
\setlength{\textheight}{250mm}
%----------------------------


%--------Metadata------------
\title{200A Homework 2}
\author{James Harbour}
%----------------------------


%--------Content-------------
\begin{document}
\maketitle

\begin{homeworkProblem}
  Suppose $G$ is a simple group and it has a subgroup $H$ of index $n$ where $n$ is an integer more than $1$. Prove that $G$ can be embedded into the symmetric group $S_n$.

  \begin{proof}
    Let $ G $ act by left multiplication on the set $ X $ of left cosets of $ H $ in $ G $. This furnishes a homomorphism $ \theta: G\to \Sym(X)\cong S_{n} $. Then $ \ker(\theta) $ is a normal subgroup of $ G $. As $ [G:H] >1 $, $ G $ acts nontrivially on $ X $ whence $ \ker(\theta)\neq G $. Thus by simplicity of $ G $, $ \ker(\theta)=\{e\} $ so $ \theta $ is an embedding.
  \end{proof}
\end{homeworkProblem}


\begin{homeworkProblem}
 For a group $G$, let $\operatorname{Aut}(G)$ be the group of automorphisms of $G$. Let
\[
  c : G \to \operatorname{Aut}(G),\qquad c(g) := c_g,\ \ \text{where } c_g(x) := gxg^{-1}\ \text{ for every } x \in G.
\]
\underline{\textbf{(a)}}: Prove that $c_g$ is an automorphism of $G$ and $c$ is a group homomorphism.

\begin{proof}
  Fix $ g\in G $. Let $ x,y\in G $. Then $$ c_{g}(xy) = gxyg^{-1} = gxg^{-1}gyg^{-1} = c_{g}(x)c_{g}(y), $$ so $ c_{g} $ is a group homomorphism.\\

  For $ h\in G $, $ c_{g}(g^{-1}hg) = h $, so $ c_{g} $ is surjective. If $ ghg^{-1} = c_{g}(h) = e $, then $ h=e $, so $ \ker(c_{g}) = \{e\} $ whence $ c_{g} $ is injective. Thus $ c_{g}\in Aut(G) $.\\

  Fix $ g,h\in G $ and $ x\in G $. Then 
  \[
    c_{g}c_{h}(x) = c_{g}(hxh^{-1}) = ghxh^{-1}g^{-1} = ghx(gh)^{-1} = c_{gh}(x),
  \]
  So $ c_{gh}= c_{g}c_{h} $, whence $ c $ is a group homomorphism.
\end{proof}


\underline{\textbf{(b)}}: Prove that $\ker c$ is the center $Z(G)$ of $G$.

\begin{proof}
  Suppose $ g\in\ker(c) $ so $ c_{g} = id_{G} $. Then for all $ x\in G $, $ x = c_{g}(x) = gxg^{-1} $, so $ g\in Z(G) $. Now suppose $ g\in Z(G) $. Then for $ x\in G $, $ c_{g}(x) = gxg^{-1}=xgg^{-1} = x $, so $ c_{g}=id_{G} $ whence $ g\in \ker(c) $.
\end{proof}


\underline{\textbf{(c)}}: The image of $c$ is called the group of inner automorphisms of $G$, and it is denoted by $\operatorname{Inn}(G)$. Prove that $\operatorname{Inn}(G)$ is a normal subgroup of $\operatorname{Aut}(G)$.

\begin{proof}
 Fix $ \phi\in \Aut(G) $ and let $ g\in G $. Set $ h:=\phi(g) $. For $ x\in G $
 \[
   \phi c_{g} \phi^{-1}(x) = \phi c_{g}(\phi^{-1}(x)) = \phi(\phi^{-1}(h) \phi^{-1}(x) \phi^{-1}(h^{-1})) = \phi(\phi^{-1}(hxh^{-1})) = hxh^{-1} = c_{\phi(g)}(x),
 \]
 so $ \phi c_{g} \phi^{-1} = c_{\phi(g)} \in \Inn(G)$. Thus $ \Inn(G)\triangleleft \Aut(G) $.
\end{proof}



\underline{\textbf{(d)}}: Prove that $\lvert Z(\operatorname{Aut}(G))\rvert \le \lvert \operatorname{Hom}(G, Z(G))\rvert$; in particular, if either $Z(G)=1$ or $G$ is perfect (that is, $G=[G,G]$), then $Z(\operatorname{Aut}(G))=\{1\}$.

\begin{proof}
As the function $ c $  is a homomorphism with image $ \Inn(G) $, it follows that $ G/Z(G) \cong \Inn(G) $. For $ \phi\in Z(\Aut(G)) $ and $ g\in G $, we have that $ c_{g} = c_{\phi(g)} $, so under the isomorphism we infer $ gZ(g) = \phi(g)Z(g) $. Hence there is some $ \eta(g)\in Z(G) $ such that $ \phi(g) = g \eta(g) $. 

For $ g,h\in G $,
\[
  gh \eta(gh) = \phi(gh) = \phi(g) \phi(h) = g\eta(g) h \eta(h) = gh\eta(g)\eta(h)
\]
whence $ \eta(gh) = \eta(g)\eta(h) $, so $ \eta $ is a group homomorphism.

Now for $ \phi\in Z(\Aut(G)) $, let $ \eta_{\phi} $ be the corresponding homomorphism obtained above. Suppose $ \phi,\psi\in Z(\Aut(G)) $ are such that $ \eta_{\phi} = \eta_{\psi} $. Then for $ g\in G $,
\[
  \phi(g) = g \eta_{\phi}(g) = g \eta_{\psi}(g) = \psi(g),
\]
so $ \phi=\psi $. Hence the assignment $ \phi\mapsto \eta_{\phi} $ is injective, whence $ |Z(\Aut(G))|=|\{\eta_{\phi}: \phi\in Z(\Aut(G))\}|\leq |\Hom(G,Z(G))| $.
\end{proof}

\end{homeworkProblem}


\begin{homeworkProblem}
Let $SL_2(\mathbb{R})$ be the set of real $2\times 2$ matrices with determinant $1$. For
$\begin{pmatrix} a & b \\ c & d \end{pmatrix} \in SL_2(\mathbb{R})$ and $z \in \mathbb{H} := \{ z \in \mathbb{C} \mid \operatorname{Im}(z) > 0\}$, let
\[
  \begin{pmatrix} a & b \\ c & d \end{pmatrix} \cdot z := \frac{az + b}{cz + d}.
\]
\underline{\textbf{(a)}}: Prove that
  \[
    \operatorname{Im}\!\left( \begin{pmatrix} a & b \\ c & d \end{pmatrix} \cdot z \right)
    \;=\; \frac{\operatorname{Im}(z)}{\lvert cz + d\rvert^{2}}.
  \]


\underline{\textbf{(b)}}: Prove that $\cdot$ is an action of $SL_2(\mathbb{R})$ on $\mathbb{H}$.


\end{homeworkProblem}


\begin{homeworkProblem}
Suppose $G$ is a finite group, $C \subseteq \mathbb{R}^n$ is a convex subset (that is, for all two points $P,Q$ in $C$, the segment $PQ$ is a subset of $C$). Suppose $G$ acts on $C$ by affine transformations; that means
\[
  \forall P,Q \in C,\ \forall t \in [0,1],\ \forall g \in G,\qquad g\cdot (tP+(1-t)Q)=t\,g\cdot P + (1-t)\,g\cdot Q.
\]
Prove that $G$ has a fixed point; that is, there exists $x \in C$ such that for all $g \in G$, $g\cdot x = x$.

\begin{proof}
  By definition, for $ c_{1},c_{2}\in C $, we have that $ \frac{c_{1}+c_{2}}{2}\in C $. Suppose that we have shown that for any $ c_{1},\ldots, c_{n}\in C $, we have $ \frac{c_{1}+\cdots+c_{n}}{n}\in C $. Observe that then
  \[
    \frac{c_{1}+ \cdots + c_{n+1}}{n+1} = \frac{n}{n+1}\lr{\frac{c_{1}+\cdots + c_{n}}{n}} + \lr{1-\frac{n}{n+1}} c_{n+1}\in C.
  \]
  Fix $ y\in C $. Then $ A_{G}(y) = \frac{1}{|G|}\sum_{h\in G}h\cdot y\in C $. For $ g\in G $, we have 
  \[
    g\cdot A_{G}(y) = \frac{1}{|G|}\sum_{h\in G} gh\cdot y = \frac{1}{|G|}\sum_{h\in G} h\cdot y = A_{G}(y),
  \]
  so $ A_{G}(y) $ is a fixed point of the affine action $ G\acts C $.

\end{proof}


\end{homeworkProblem}


\begin{homeworkProblem}
Suppose $G$ is a finite subgroup of the group $GL_n(\mathbb{R})$ of $n\times n$ invertible real matrices. Prove that there is a $G$-invariant inner product on $\mathbb{R}^n$.

\begin{proof}
  Define 
  \[
    \inp{v}{w}_{G}:=\frac{1}{|G|}\sum_{g\in G}\inp{g\cdot v}{g\cdot w}
  \]
\end{proof}

\end{homeworkProblem}


\begin{homeworkProblem}
  Suppose $H$ is a subgroup of $G$. Let
  \[
    C_G(H) := \{\, x \in G \mid \forall h \in H,\ xh = hx \,\}
  \]
  be the centralizer of $H$ in $G$, and
  \[
    N_G(H) := \{\, x \in G \mid xHx^{-1} = H \,\}
  \]
  be the normalizer of $H$ in $G$. Both of these are subgroups of $G$ and clearly $C_G(H) \subseteq N_G(H)$. Prove that $N_G(H)/C_G(H)$ can be embedded into $\operatorname{Aut}(H)$.

\end{homeworkProblem}

\begin{homeworkProblem}
Suppose $N$ is a finite cyclic normal subgroup of $G$. Prove that every subgroup of $N$ is normal in $G$.

\begin{proof}
  Let $ K\leq N $ have order $ k $. Then for $ g\in G $, $ gKg^{-1}\leq N $ has order k, but cyclic groups have unique subgroups of each given order, so $ gKg^{-1}=K $.
\end{proof}
\end{homeworkProblem}

\end{document}
