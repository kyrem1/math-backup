\documentclass[12pt,letterpaper]{article}

%--------Packages--------
\usepackage{amsmath, amsthm, amssymb}
\usepackage{xspace}
\usepackage{graphicx}
\usepackage{amssymb}
\usepackage{array}
\usepackage{braket}
\usepackage{multicol}
\usepackage{mathtools}
\usepackage{enumerate}
\usepackage{delarray}
\usepackage{mathtools}
\usepackage{fullpage}
\usepackage{faktor} % For quotients
\usepackage{mathrsfs}
%\usepackage{quiver}
%\usepackage{tikz}

\usepackage[linguistics]{forest}

\usepackage{import}
\usepackage{pdfpages}
\usepackage{transparent}
\usepackage{xcolor}

\newcommand{\incfig}[2][1]{%
    \def\svgwidth{#1\columnwidth}
    \import{./figures/}{#2.pdf_tex}
}

\pdfsuppresswarningpagegroup=1



%--------Page Setup--------

\pagestyle{empty}%

\setlength{\hoffset}{-1.54cm}
\setlength{\voffset}{-1.54cm}

\setlength{\topmargin}{0pt}
\setlength{\headsep}{0pt}
\setlength{\headheight}{0pt}

\setlength{\oddsidemargin}{0pt}

\setlength{\textwidth}{195mm}
\setlength{\textheight}{250mm}


%--------Macros--------

\newcommand{\ilm}[1]{\begin{psmallmatrix} #1 \end{psmallmatrix}}
\newcommand{\ilmb}[1]{\boxed{\begin{smallmatrix} #1 \end{smallmatrix}}}

\newcommand{\sub}{\subseteq}
\newcommand{\lcm}{\text{lcm}}
\newcommand{\ms}[1]{\mathscr{#1}}
\newcommand{\mc}[1]{\mathcal{#1}}
\newcommand{\mf}[1]{\mathfrak{#1}}
\newcommand{\m}{\mf{m}}
\newcommand{\sO}{\mathcal{O}}
\newcommand{\cyclic}[1]{\langle#1\rangle}
\newcommand{\units}[1]{#1 ^{\times}}
\newcommand{\la}{\langle}
\newcommand{\ra}{\rangle}
\newcommand{\lr}[1]{\left(#1\right)}
\newcommand{\divides}{\bigm|}
\newcommand{\restr}{\big|}


%----Switch phi and varphi
\let\temp\phi
\let\phi\varphi
\let\varphi\temp

\newcommand{\C}{\mathbb{C}}
\newcommand{\F}{\mathbb{F}}
\newcommand{\N}{\mathbb{N}\xspace}
\newcommand{\I}{\mathbb{I}\xspace}
\newcommand{\R}{\mathbb{R}\xspace}
\newcommand{\Z}{\mathbb{Z}\xspace}
\newcommand{\Q}{\mathbb{Q}\xspace}
\newcommand{\G}{\mathbb{G}\xspace}
\let\O\relax
\newcommand{\O}{\mathcal{O}}
\let\p\relax
\newcommand{\p}{\mathfrak{p}}
\let\q\relax
\newcommand{\q}{\mathfrak{q}}
\let\m\relax
\newcommand{\m}{\mathfrak{m}}
\DeclareMathOperator{\Spec}{Spec}
\DeclareMathOperator{\Specm}{Specm}
\DeclareMathOperator{\res}{res}
\DeclareMathOperator{\Tr}{Tr}
\DeclareMathOperator{\ord}{ord}
\DeclareMathOperator{\Sym}{Sym}
\DeclareMathOperator{\dv}{div}
\DeclareMathOperator{\alb}{alb}
\let\Im\relax
\DeclareMathOperator{\Im}{Im}
\DeclareMathOperator{\et}{et}
\DeclareMathOperator{\ck}{coker}
\DeclareMathOperator{\Reg}{Reg}
\DeclareMathOperator{\Cor}{Cor}
\DeclareMathOperator{\Ac}{at}
\DeclareMathOperator{\supp}{supp}
\DeclareMathOperator{\Hom}{Hom}
\DeclareMathOperator{\Pic}{Pic}
\DeclareMathOperator{\Gal}{Gal}
\DeclareMathOperator{\fc}{frac}
\DeclareMathOperator{\Ann}{Ann}
\DeclareMathOperator{\Mod}{Mod}
\DeclareMathOperator{\Cone}{Cone}
\DeclareMathOperator{\FI}{FI}
\DeclareMathOperator{\End}{End}
\DeclareMathOperator{\Alb}{Alb}
\DeclareMathOperator{\Ext}{Ext}
\DeclareMathOperator{\ab}{ab}
\DeclareMathOperator{\Jac}{Jac}
\DeclareMathOperator{\coker}{coker}
\DeclareMathOperator{\fr}{frac}
\DeclareMathOperator{\spn}{span}
\DeclareMathOperator{\im}{im}
\DeclareMathOperator{\rk}{rk}
\DeclareMathOperator{\GL}{GL}
\DeclareMathOperator{\Aut}{Aut}
\DeclareMathOperator{\ch}{char}
\DeclareMathOperator{\Fix}{Fix}
\DeclareMathOperator{\Inn}{Inn}


%----Analysis
\newcommand{\dd}[2][]{\frac{\partial^{#1}}{\partial {#2}^{#1}}}
\newcommand{\summ}{\sum\limits}
\newcommand{\norm}[1]{\left \vert \left \vert #1 \right \vert \right \vert}
\newcommand{\thicc}{\bigg}
\newcommand{\eps}{\varepsilon}
\newcommand*\cls[1]{\overline{#1}}


%--------Theorem environments--------
\newtheorem{definition}{Definition}[]
\newtheorem{Lemma}{Lemma}[]
\newtheorem{corollary}{Corollary}[]
\newtheorem{Theorem}{Theorem}[]
\theoremstyle{remark}
\newtheorem*{claim}{Claim}


\newenvironment{solution}
{\begin{proof}[Solution]}
{\end{proof}}


\makeatletter
\newcolumntype{"}{@{\hskip\tabcolsep\vrule width 1pt\hskip\tabcolsep}}
\makeatother

% --------Problem environment--------
\setlength\parindent{0pt}
\setcounter{secnumdepth}{0}
\newcounter{partCounter}
\newcounter{homeworkProblemCounter}
\setcounter{homeworkProblemCounter}{1}


\newenvironment{homeworkProblem}[1][-1]{
    \ifnum#1>0
        \setcounter{homeworkProblemCounter}{#1}
    \fi
    \section{Problem \arabic{homeworkProblemCounter}}
    \setcounter{partCounter}{1}
    \stepcounter{homeworkProblemCounter}
}


%--------Metadata--------
\title{MATH 8851 Homework 3}
\author{James Harbour}
\begin{document}

\maketitle


\begin{homeworkProblem}
     Let $G=A\ltimes_{\phi}B$ be the (external) semidirect product of groups
$A$ and $B$ corresponding to some homomorphism $\phi:A\to \Aut(B)$. Suppose
we are given presentations by generators and relations for both $A$ and $B$:
$$A=\la X_1 | R_1\ra \qquad B=\la X_2 | R_2\ra. $$
Find (with proof) a presentation for $G$ in terms of $X_1,X_2,R_1,R_2$ and $\phi$.

{\bf Note:} If you succeeded in proving Hall's Theorem asserting that an extension of finitely presented groups is finitely presented (HW\# 2.2), this problem should be straightforward. If you did not succeed in solving HW\# 2.2, you may want to start with this problem and then come back to HW\# 2.2.

\end{homeworkProblem}


\begin{homeworkProblem}
     Let $p$ be a prime. Prove that the lamplighter group 
$G_p=\Z\, wr\, \Z/p\Z$ is not finitely presented.

\begin{proof}
    By HW\# 2.4(c), $ G_{p} $ admits the following presentation
    \[
        \la x,y \mid y^p=1, [y,y^x]=1, [y, y^{x^2}]=1, [y, y^{x^{N-1}}]=1,\ldots\ra.
    \]
    Suppose, for the sake of contradiction, that $ G_{p} $ is finitely presented. Then there exists some $ N\in \N $ such that in fact we have the following finite presentation
    \[
        \la x,y \mid y^p=1, [y,y^x]=1, [y, y^{x^2}]=1, [y, y^{x^{N-1}}]=1\ra.
    \]
    and $ [y,y^{x^{i}}] = 1 $ for all $ i\in \N\cup\{0\} $. \\


    % TODO Insert proof of lemma about $ S_M $.

    By von Dyck's theorem, there exists a homomorphism $ \phi: G_{p}\to S_{M} $ such that $ \phi(x) = s,\ \phi(y) = t $. Now, observe that
    \[
        1 = \phi(1) = \phi([y,y^{x^{N}}])  = [\phi(y), \phi(y)^{\phi(x)^{N}}] = [t,t^{s^{N}}] \neq 1,
    \]
    which is a contradiction.
\end{proof}
{\bf Hint:} Now prove that this is impossible as follows. Show that for sufficiently large $M$
(depending on $N$) the symmetric group $S_M$ contains 2 elements $t$ and $s$
such that $t^p=1$ and $[t,t^{s^i}]=1$ for all $1\leq i\leq N-1$, but
$[t,t^{s^N}]\neq 1$. Then apply von Dyck's theorem to get a contradiction.

\end{homeworkProblem}



\begin{homeworkProblem}
 Recall from class that $\Aut^+(F_n)$ is the preimage of $SL_n(\Z)$
under the natural ``abelianization'' map $\pi: \Aut(F_n)\to GL_n(\Z)$
(which is surjective, as proved in Lecture~8) and thus 
$[\Aut(F_n):\Aut^+(F_n)]=2$. The goal of this problem is to prove that $\Aut^+(F_n)$ is generated by the elements $R_{ij}$ and $L_{ij}$
(recall that $R_{ij}$ sends $x_i$ to $x_i x_j$ and fixes all other $x_k$
and $L_{ij}$ sends $x_i$ to $x_j x_i$ and fixes all other $x_k$).

Define $H=\la R_{ij},L_{ij}\ra$. Then $H\subseteq \Aut^+(F_n)$, and to prove the equality it suffices to show that $[\Aut(F_n):H]=2$.\\

\textbf{(a)}: Recall that $\Aut(F_n)$ is generated by the elements $R_{ij}$, $L_{ij}$,
$I_i$ and $P_{\sigma}$, with $\sigma\in S_n$, where $I_i$ inverts $x_i$ and fixes all other $x_k$ and $P_{\sigma}$ sends $x_k$ to $x_{\sigma(k)}$ for all $k$.
Use this fact to prove that $H$ is normal in $\Aut(F_n)$.

\begin{proof}
    See paper.
\end{proof}

\textbf{(b)}: For any $1\leq i\neq j\leq n$ let $Q_{ij}$ be the element
of $\Aut(F_n)$ given by $x_i\mapsto x_j$, $x_j\mapsto x_i^{-1}$ and $x_k\to x_k$
for $k\neq i,j$. Prove by direct computation that $Q_{ij}\in \Aut^+(F_n)$.

\begin{proof}
    
\end{proof}

\textbf{(c)}: Given $g\in \Aut(F_n)$, let $\overline g$ denote the image of $g$
in $\Aut(F_n)/H$. Use (b) to show that $\overline{P_{(ij)}}=\overline{I_i}$
for any $i\neq j$ (here $(ij)$ is the transposition swapping $i$ and $j$). Deduce from this that $|\Aut(F_n)/H|=2$ and thus $[\Aut(F_n):H]=2$.
\end{itemize}

   
\end{homeworkProblem}


\begin{homeworkProblem}
 Recall from class that $IA_n$ (also called the Torelli subgroup of $\Aut(F_n)$) is the kernel of $\pi: \Aut(F_n)\to GL_n(\Z)$.

\begin{itemize}
\item[(a)] Prove that $IA_n$ contains $\Inn(F_n)$, the subgroup of inner automorphsisms of $F_n$.
\item[(b)] Magnus (1935) proved that $IA_n$ is generated by the elements
$K_{ij}$ with $1\leq i\neq j\leq n$ and $K_{ijm}$ with $i,j,m$ distinct
where $K_{ij}$ sends $x_i$ to $x_j^{-1}x_i x_j$ and fixes all other $x_k$
and $K_{ijm}$ sends $x_i$ to $x_i [x_j,x_m]$ and fixes all other $x_k$.
Verify that the elements $K_{ij}$ and $K_{ijm}$ indeed lie in $IA_n$.
\item[(c)] Use (b) to show that $IA_2=\Inn(F_2)$. We will discuss a different
proof of this result later in the course.
\end{itemize}

   
\end{homeworkProblem}

\begin{homeworkProblem}
  The proof of the Nielsen reduction theorem (Theorem~7.5) yields a general algorithm
which, given an $n$-tuple of elements of $F_n$, decides whether these elements
generate $F_n$ or not. In the case $n=2$ one can answer this question almost
immediately using the following commutator test.
\begin{Theorem}[Commutator test] Let $\{x,y\}$ be a free generating set of $F_2$,
and take any $u,v\in F_2$. Then $u$ and $v$ generate $F_2$ if and only if
the commutator $[u,v]=u^{-1}v^{-1}uv$ is conjugate (in $F_2$) to
$[x,y]$ or $[y,x]=[x,y]^{-1}$.
\end{Theorem}
\begin{itemize}
\item[(a)] Prove the `only if' ($\Rightarrow$) part of the commutator test.
{\bf Hint:} Use Nielsen moves.
\item[(b)] Now think of how you would prove the `if' part. I do not know
of a nice short algebraic argument. One possible proof is outlined in the following
paper of Shpilrain (see Proposition~2.4):

\end{itemize}

  
\end{homeworkProblem}
 \end{document}
