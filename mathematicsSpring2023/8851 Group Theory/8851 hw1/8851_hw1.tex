\documentclass[12pt,letterpaper]{article}

%--------Packages--------
\usepackage{amsmath, amsthm, amssymb}
\usepackage{xspace}
\usepackage{graphicx}
\usepackage{amssymb}
\usepackage{array}
\usepackage{braket}
\usepackage{multicol}
\usepackage{mathtools}
\usepackage{enumerate}
\usepackage{delarray}
\usepackage{mathtools}
\usepackage{fullpage}
\usepackage{faktor} % For quotients
\usepackage{mathrsfs}
%\usepackage{quiver}
%\usepackage{tikz}

\usepackage[linguistics]{forest}

\usepackage{import}
\usepackage{pdfpages}
\usepackage{transparent}
\usepackage{xcolor}

\newcommand{\incfig}[2][1]{%
    \def\svgwidth{#1\columnwidth}
    \import{./figures/}{#2.pdf_tex}
}

\pdfsuppresswarningpagegroup=1



%--------Page Setup--------

\pagestyle{empty}%

\setlength{\hoffset}{-1.54cm}
\setlength{\voffset}{-1.54cm}

\setlength{\topmargin}{0pt}
\setlength{\headsep}{0pt}
\setlength{\headheight}{0pt}

\setlength{\oddsidemargin}{0pt}

\setlength{\textwidth}{195mm}
\setlength{\textheight}{250mm}


%--------Macros--------

\newcommand{\ilm}[1]{\begin{psmallmatrix} #1 \end{psmallmatrix}}
\newcommand{\ilmb}[1]{\boxed{\begin{smallmatrix} #1 \end{smallmatrix}}}

\newcommand{\sub}{\subseteq}
\newcommand{\lcm}{\text{lcm}}
\newcommand{\ms}[1]{\mathscr{#1}}
\newcommand{\mc}[1]{\mathcal{#1}}
\newcommand{\mf}[1]{\mathfrak{#1}}
\newcommand{\m}{\mf{m}}
\newcommand{\sO}{\mathcal{O}}
\newcommand{\cyclic}[1]{\langle#1\rangle}
\newcommand{\units}[1]{#1 ^{\times}}
\newcommand{\la}{\langle}
\newcommand{\ra}{\rangle}
\newcommand{\lr}[1]{\left(#1\right)}
\newcommand{\divides}{\bigm|}
\newcommand{\restr}{\big|}


%----Switch phi and varphi
\let\temp\phi
\let\phi\varphi
\let\varphi\temp

\newcommand{\C}{\mathbb{C}}
\newcommand{\F}{\mathbb{F}}
\newcommand{\N}{\mathbb{N}\xspace}
\newcommand{\I}{\mathbb{I}\xspace}
\newcommand{\R}{\mathbb{R}\xspace}
\newcommand{\Z}{\mathbb{Z}\xspace}
\newcommand{\Q}{\mathbb{Q}\xspace}
\newcommand{\G}{\mathbb{G}\xspace}
\let\O\relax
\newcommand{\O}{\mathcal{O}}
\let\p\relax
\newcommand{\p}{\mathfrak{p}}
\let\q\relax
\newcommand{\q}{\mathfrak{q}}
\let\m\relax
\newcommand{\m}{\mathfrak{m}}
\DeclareMathOperator{\Spec}{Spec}
\DeclareMathOperator{\Specm}{Specm}
\DeclareMathOperator{\res}{res}
\DeclareMathOperator{\Tr}{Tr}
\DeclareMathOperator{\ord}{ord}
\DeclareMathOperator{\Sym}{Sym}
\DeclareMathOperator{\dv}{div}
\DeclareMathOperator{\alb}{alb}
\let\Im\relax
\DeclareMathOperator{\Im}{Im}
\DeclareMathOperator{\et}{et}
\DeclareMathOperator{\ck}{coker}
\DeclareMathOperator{\Reg}{Reg}
\DeclareMathOperator{\Cor}{Cor}
\DeclareMathOperator{\Ac}{at}
\DeclareMathOperator{\supp}{supp}
\DeclareMathOperator{\Hom}{Hom}
\DeclareMathOperator{\Pic}{Pic}
\DeclareMathOperator{\Gal}{Gal}
\DeclareMathOperator{\fc}{frac}
\DeclareMathOperator{\Ann}{Ann}
\DeclareMathOperator{\Mod}{Mod}
\DeclareMathOperator{\Cone}{Cone}
\DeclareMathOperator{\FI}{FI}
\DeclareMathOperator{\End}{End}
\DeclareMathOperator{\Alb}{Alb}
\DeclareMathOperator{\Ext}{Ext}
\DeclareMathOperator{\ab}{ab}
\DeclareMathOperator{\Jac}{Jac}
\DeclareMathOperator{\coker}{coker}
\DeclareMathOperator{\fr}{frac}
\DeclareMathOperator{\spn}{span}
\DeclareMathOperator{\im}{im}
\DeclareMathOperator{\rk}{rk}
\DeclareMathOperator{\GL}{GL}
\DeclareMathOperator{\Aut}{Aut}
\DeclareMathOperator{\ch}{char}
\DeclareMathOperator{\Fix}{Fix}


%----Analysis
\newcommand{\dd}[2][]{\frac{\partial^{#1}}{\partial {#2}^{#1}}}
\newcommand{\summ}{\sum\limits}
\newcommand{\norm}[1]{\left \vert \left \vert #1 \right \vert \right \vert}
\newcommand{\thicc}{\bigg}
\newcommand{\eps}{\varepsilon}
\newcommand*\cls[1]{\overline{#1}}


%--------Theorem environments--------
\newtheorem{definition}{Definition}[]
\newtheorem{Lemma}{Lemma}[]
\newtheorem{corollary}{Corollary}[]
\newtheorem{Theorem}{Theorem}[]
\theoremstyle{remark}
\newtheorem*{claim}{Claim}


\newenvironment{solution}
{\begin{proof}[Solution]}
{\end{proof}}


\makeatletter
\newcolumntype{"}{@{\hskip\tabcolsep\vrule width 1pt\hskip\tabcolsep}}
\makeatother

% --------Problem environment--------
\setlength\parindent{0pt}
\setcounter{secnumdepth}{0}
\newcounter{partCounter}
\newcounter{homeworkProblemCounter}
\setcounter{homeworkProblemCounter}{1}


\newenvironment{homeworkProblem}[1][-1]{
    \ifnum#1>0
        \setcounter{homeworkProblemCounter}{#1}
    \fi
    \section{Problem \arabic{homeworkProblemCounter}}
    \setcounter{partCounter}{1}
    \stepcounter{homeworkProblemCounter}
}


%--------Metadata--------
\title{MATH 8851 Homework 1}
\author{James Harbour}
\begin{document}

\maketitle

\begin{homeworkProblem}
    Prove the Schreier Subgroup Lemma (the statement is recalled below) without the extra assumption $1\in T$. {\bf Note:} You just need to slightly adjust the proof from class (where we assumed that $1\in T$).

    \begin{proof}
        Let $ H\leq G $, $ S\sub G $ a generating set for $ G $, and $ T\sub G$ a right transversal for $ H $ inside $ G $. Set 
        \[
            U = U(S,T) = \{ts\cdot \cls{ts}^{-1} : s\in S,\,t\in T\}.
        \]
        For $ s\in S, t\in T $, we have that 
        \[
            Hts = H \cls{ts} \implies H ts\cdot \cls{ts}^{-1} = H \implies ts\cdot \cls{ts}^{-1} \in H,
        \]
        so $ U\sub H $ whence $ \cyclic{U}\leq H $. Thus, it suffices to show the reverse inclusion. First, let $ t = \cls{1}\in T\cap H $. Then Choose $ s_{1} $


        Choose $ s_{1},\ldots, s_{n}\in S\cup S^{-1} $ such that $ t = \prod_{i=1}^{n}s_{i} $, and set $ t_{k}:=\cls{\prod_{i=1}^{k}s_{i}} $, $ t_{0} = t $. Then we have 
        \begin{align*}
            t = t_{0}^{-1}\lr{\prod_{i=0}^{n-1}t_{i}s_{i+1}t_{i+1}^{-1}}t_{n} =  t^{-1}\lr{\prod_{i=0}^{n-1}t_{i}s_{i+1}t_{i+1}^{-1}}t_{n} \implies t = \prod_{i=0}^{n-1}t_{i}s_{i+1}t_{i+1}^{-1} \in<
        \end{align*}



        
    \end{proof}


\end{homeworkProblem}

\begin{homeworkProblem}
    Let $F=F(X)$ for some set $X$ and $H$ a subgroup of $F$. Prove that $H$ always has a Schreier transversal in $F$
(with respect to $X$) in two different ways as follows:
\begin{itemize}
\item[(a)] Using Zorn's lemma
\item[(b)] Using suitable total order on $F$. 
\end{itemize}
{\bf Hint for (b):} Choose an arbitrary total order on $X\sqcup X^{-1}$
and consider the corresponding lexicographical order on $F$: given two elements $f\neq f'\in F$,
put $f<f'$ if one of the following holds:
\begin{itemize}
\item[(i)] $l(f)< l(f')$, where $l(\cdot)$ is the word length
\item[(ii)] $l(f)=l(f')$, and if $f$ and $f'$ first differ in $k^{\rm th}$ position,
then the $k^{\rm th}$ symbol in $f$ is smaller than the $k^{\rm th}$ symbol in $f'$.
\end{itemize}
Then form a transversal by choosing the smallest element in each right coset of $H$.

\begin{proof}[Proof (a)]
    Consider the poset 
    \[ 
        \mathscr{S} = \{T\sub F : T\text{ is a Schreier subset},\ Ht \neq Ht '\ \forall t\neq t ' \text{ in }T\} 
    \]
    ordered by inclusion. This poset is nonempty as it contains $ \{1\} $. Consider any linear chain $ (T_{\alpha})_{\alpha\in I} $ in $ \mathscr{S} $ and set $ T = \bigcup_{\alpha\in I}T_{\alpha} $. Suppose $ t_{1},t_{2}\in T $ have $ t_{1}\neq t_{2}$. By linearity of the chain, we may choose an $ \alpha\in I $ such that $ t_{1},t_{2}\in T_{\alpha} $ whence by assumption, $ Ht_{1} \neq Ht_{2} $. Now, suppose that $ w $ is a reduced word in $ T $. Again by linearity, there exists some $ \alpha\in I $ such that all of the symbols from $ T $ appearing in the reduced word decomposition for $ w $ actually lie inside $ T_{\alpha} $. Since $ T_{\alpha} $ is Schreier, every initial segment of $ w $ also lies in $ T_{\alpha} $ and thus $ T $. So, $ T\in \mathscr{S} $ is an upper bound for the chain inside $ \mathscr{S} $.\\

    By Zorn's lemma, there exists a maximal element $ T $ of $ \mathscr{S} $. We claim that $ T $ is a right transversal for $ H $. To this end, suppose for the sake of contradiction that $ T $ is not a right transversal for $ H $. Then there is some $ x\in F $ such that $ Hx\neq Ht $ for all $ t\in T $. \\
    
    Consider the set $ S $ consisting of all initial segments of reduced words in $ T\sqcup\{x\} $. By construction, $ S\supseteq T $ is a Schreier subset of $ F $. We claim that $ S $ is in fact an element of $ T $.
    % TODO... this might be bs since im not reducing and such
    
\end{proof}


\begin{proof}[Proof (b)]

\end{proof}


\end{homeworkProblem}


\begin{homeworkProblem}
    Let $F=F(x,y)$ be the free group on two generators. Consider the following two subgroups of $F$:
\begin{itemize}
\item[(a)] $H=[F,F]$, the commutator subgroup of $F$
\item[(b)] $H=\ker\,\pi$ where $\pi$ is the epimorphism from $F$ onto $S_3$ (symmetric group on $3$ letters)
which sends $x$ to $(12)$ and $y$ to $(23)$.
\end{itemize}
For each of these subgroups do the following:
\begin{itemize}
\item[(i)] Find a Schreier transversal $T$ for $H$ (with respect to $X=\{x,y\}$).
\item[(ii)] Draw the Schreier graph $Sch(H\setminus F,X)$ and the maximal tree $\mathcal T$ in $Sch(H\setminus F,X)$ corresponding to $T$
(we will define the natural bijection between the Schreier transversals and maximal trees in class on Monday, Jan 29)
\item[(iii)] Use the strong Nielsen-Schreier Theorem (the statement is recalled below) to find a free generating set for $H$.
\end{itemize}


\end{homeworkProblem}


\begin{homeworkProblem}
    Prove the {\it Schreier index formula}: If $F$ is a free group of finite rank and $H$ a subgroup of $F$ of finite index, then $$\rk(H)-1=(\rk(F)-1)\cdot [F:H].$$ 
    {\bf Hint:} Count the number of vertices and edges in the Schreier graph $Sch(H\setminus F,X)$ and use the fact that $H\cong \pi_1(Sch(H\setminus F,X))$. 
 
    \begin{proof}
        Let $ X = \{x_{1}, \ldots, x_{n}\} $ be a free generating set for $ F $ and assume without loss of generality that $ F = F(X) $. 
    \end{proof}
\end{homeworkProblem}


\begin{homeworkProblem}
    Use the Schreier Subgroup lemma to find a generating set with $2$ elements for the alternating group $A_n$. 

    \begin{proof}[Solution]
        Consider the generating set $ S = \{(12), (12\cdots n)\} $ for $ S_{n} $. The set $ T = \{e, (12)\} $ is a right transversal for $ A_{n} $ inside $ S_{n} $. Note that then, for a given permuation $ \sigma\in S_{n} $,
        \[
            \cls{\sigma} = \begin{cases}
                e & \sigma \text{ is even}\\
                (12) & \sigma \text{ is odd}.
            \end{cases}
        \]
        If $ n $ is odd, then $ U(S,T) $ consists of 
        \begin{align*}
            &e(12) \cls{e(12)}^{-1} = e \\
            &(12)(12) \cls{(12)(12)}^{-1} = e\\
            &e(12\cdots n) \cls{e(12\cdots n)}^{-1} = (12\cdots n)\\
            &(12)(12\cdots n) \cls{(12)(12\cdots n)}^{-1} = (12)(12\cdots n)(12).
        \end{align*}

        If $ n $ is even, then $ U(S,T) $ consists of 
        \begin{align*}
            &e(12) \cls{e(12)}^{-1} = e \\
            &(12)(12) \cls{(12)(12)}^{-1} = e\\
            &e(12\cdots n) \cls{e(12\cdots n)}^{-1} = (12\cdots n)(12)\\
            &(12)(12\cdots n) \cls{(12)(12\cdots n)}^{-1} = (12)(12\cdots n).
        \end{align*}

        Thus, Schreier Subgroup lemma produces the following $ 2 $-element generating sets for $ A_{n} $
        \[
            U(S,T) = \begin{cases}
                \{(12\cdots n), (12)(12\cdots n)(12)\} & n \text{ is odd}\\
                \{(12\cdots n)(12), (12)(12\cdots n)\} & n \text{ is even}
            \end{cases}
        \]
    \end{proof}

\end{homeworkProblem}


\begin{Lemma}[Schreier Subgroup Lemma] Let $G$ be a group, $H$ a subgroup of $G$, $S$ a generating set for $G$ and $T$ a right transversal for $H$ in $G$.
Then $H$ is generated by the set
$$U=U(S,T)=\{ts\cdot{\overline{ts}}^{\,-1}: s\in S, t\in T\}$$
where $\overline g$ is the unique element of $T$ such that $H\overline g=Hg$.
\end{Lemma}


\begin{Theorem}[Strong Nielsen-Schreier Theorem] Let $H$ a subgroup of $F(X)$, and let $T$ be a (right) Schreier transversal for $H$ (with respect to $X$). For every $x\in X$ and $t\in T$ let $h_{x,t}=xt\cdot {\overline{xt}}^{\,-1}$.
Let $$I=\{(x,t)\in X\times T: h_{x,t}\neq 1\}.$$ Then the elements $\{h_{x,t}: (x,t)\in I\}$ are all distinct and form a free 
generating set for $H$.
\end{Theorem}

\end{document}
