\documentclass[12pt]{amsart}

\usepackage{amsmath}
\usepackage{amssymb}
\usepackage{amsthm}
%\usepackage{psfig}

\newtheorem* {Theorem}    {Theorem}
\newtheorem* {Lemma}    {Lemma}


\begin{document}
 \pagenumbering{gobble}
\baselineskip=16pt
\textheight=8.5in
\textwidth=6in
%\parindent=0pt 
\def\sk {\hskip .5cm}
\def\skv {\vskip .08cm}
\def\cos {\mbox{cos}}
\def\sin {\mbox{sin}}
\def\tan {\mbox{tan}}
\def\intl{\int\limits}
\def\lm{\lim\limits}
\newcommand{\frc}{\displaystyle\frac}
\def\xbf{{\mathbf x}}
\def\fbf{{\mathbf f}}
\def\gbf{{\mathbf g}}

\def\dbA{{\mathbb A}}
\def\dbB{{\mathbb B}}
\def\dbC{{\mathbb C}}
\def\dbD{{\mathbb D}}
\def\dbE{{\mathbb E}}
\def\dbF{{\mathbb F}}
\def\dbG{{\mathbb G}}
\def\dbH{{\mathbb H}}
\def\dbI{{\mathbb I}}
\def\dbJ{{\mathbb J}}
\def\dbK{{\mathbb K}}
\def\dbL{{\mathbb L}}
\def\dbM{{\mathbb M}}
\def\dbN{{\mathbb N}}
\def\dbO{{\mathbb O}}
\def\dbP{{\mathbb P}}
\def\dbQ{{\mathbb Q}}
\def\dbR{{\mathbb R}}
\def\dbS{{\mathbb S}}
\def\dbT{{\mathbb T}}
\def\dbU{{\mathbb U}}
\def\dbV{{\mathbb V}}
\def\dbW{{\mathbb W}}
\def\dbX{{\mathbb X}}
\def\dbY{{\mathbb Y}}
\def\dbZ{{\mathbb Z}}

\def\la{{\langle}}
\def\ra{{\rangle}}
\def\Ker{{\rm Ker}}
\def\rk{{\rm rk}}
\def\summ{{\sum\limits}}

\bf\centerline{Math 8851. Homework \#1. To be completed by Thu, Feb 2}\rm
\vskip .1cm
1. Prove the Schreier Subgroup Lemma (the statement is recalled below) without the extra assumption $1\in T$.
{\bf Note:} You just need to slightly adjust the proof from class (where we assumed that $1\in T$).


2. Let $F=F(X)$ for some set $X$ and $H$ a subgroup of $F$. Prove that $H$ always has a Schreier transversal in $F$
(with respect to $X$) in two different ways as follows:
\begin{itemize}
\item[(a)] Using Zorn's lemma
\item[(b)] Using suitable total order on $F$. 
\end{itemize}
{\bf Hint for (b):} Choose an arbitrary total order on $X\sqcup X^{-1}$
and consider the corresponding lexicographical order on $F$: given two elements $f\neq f'\in F$,
put $f<f'$ if one of the following holds:
\begin{itemize}
\item[(i)] $l(f)< l(f')$, where $l(\cdot)$ is the word length
\item[(ii)] $l(f)=l(f')$, and if $f$ and $f'$ first differ in $k^{\rm th}$ position,
then the $k^{\rm th}$ symbol in $f$ is smaller than the $k^{\rm th}$ symbol in $f'$.
\end{itemize}
Then form a transversal by choosing the smallest element in each right coset of $H$.
\skv

3. Let $F=F(x,y)$ be the free group on two generators. Consider the following two subgroups of $F$:
\begin{itemize}
\item[(a)] $H=[F,F]$, the commutator subgroup of $F$
\item[(b)] $H=\Ker\,\pi$ where $\pi$ is the epimorphism from $F$ onto $S_3$ (symmetric group on $3$ letters)
which sends $x$ to $(12)$ and $y$ to $(23)$.
\end{itemize}
For each of these subgroups do the following:
\begin{itemize}
\item[(i)] Find a Schreier transversal $T$ for $H$ (with respect to $X=\{x,y\}$).
\item[(ii)] Draw the Schreier graph $Sch(H\setminus F,X)$ and the maximal tree $\mathcal T$ in $Sch(H\setminus F,X)$ corresponding to $T$
(we will define the natural bijection between the Schreier transversals and maximal trees in class on Monday, Jan 29)
\item[(iii)] Use the strong Nielsen-Schreier Theorem (the statement is recalled below) to find a free generating set for $H$.
\end{itemize}
\skv

4. Prove the {\it Schreier index formula}: If $F$ is a free group of finite rank and $H$ a subgroup of $F$ of finite index,
then $$\rk(H)-1=(\rk(F)-1)\cdot [F:H].$$ 
{\bf Hint:} Count the number of vertices and edges in the Schreier graph $Sch(H\setminus F,X)$
and use the fact that $H\cong \pi_1(Sch(H\setminus F,X))$. 
 
\skv
5. Use the Schreier Subgroup lemma to find a generating set with $2$ elements for the alternating group $A_n$.
\skv
\begin{Lemma}[Schreier Subgroup Lemma] Let $G$ be a group, $H$ a subgroup of $G$, $S$ a generating set for $G$ and $T$ a right transversal for $H$ in $G$.
Then $H$ is generated by the set
$$U=U(S,T)=\{st\cdot{\overline{st}}^{\,-1}: s\in S, t\in T\}$$
where $\overline g$ is the unique element of $T$ such that $H\overline g=Hg$.
\end{Lemma}

\begin{Theorem}[Strong Nielsen-Schreier Theorem] Let $H$ a subgroup of $F(X)$, and let $T$ be a (right) Schreier transversal for $H$ (with respect to $X$). For every $x\in X$ and $t\in T$ let $h_{x,t}=xt\cdot {\overline{xt}}^{\,-1}$.
Let $$I=\{(x,t)\in X\times T: h_{x,t}\neq 1\}.$$ Then the elements $\{h_{x,t}: (x,t)\in I\}$ are all distinct and form a free 
generating set for $H$.
\end{Theorem}

\end{document}



