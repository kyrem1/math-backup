\documentclass[a4paper,11pt]{article}
\usepackage{times}
\usepackage{amsthm}
\usepackage{amsmath}
\usepackage{amssymb}
\usepackage{mathrsfs}
\usepackage{pb-diagram}
\usepackage{xcolor}
\usepackage{braket}
\usepackage{mathtools}
\usepackage[italicdiff]{physics}
\usepackage{tikz}

\usepackage[backend=biber,maxnames=10]{biblatex}
\addbibresource{defrig.bib}

% List spacing 
\usepackage{enumitem}



%\usepackage[nosort,nocompress,noadjust]{cite}
%\renewcommand{\citeleft}{\textcolor{blue!50!black}{[}}
%\renewcommand{\citeright}{\textcolor{blue!50!black}{]}}
%\renewcommand{\citepunct}{\textcolor{blue!50!black}{,$\,$}}
%\renewcommand{\citemid}{\textcolor{blue!50!blacfk}{, }\textcolor{blue!50!black}}

\usepackage[linktocpage=true,bookmarks=false,hyperfootnotes=false,colorlinks,
    linkcolor={red!60!black},
    citecolor={blue!50!black},
    urlcolor={blue!80!black}]{hyperref}

\renewcommand{\eqref}[1]{\hyperref[#1]{(\ref{#1})}}


\pagestyle{plain}


\setlength{\evensidemargin}{0pt}
\setlength{\oddsidemargin}{0pt}
\setlength{\topmargin}{-20pt}
\setlength{\footskip}{55pt}
\setlength{\textheight}{670pt}
\setlength{\textwidth}{450pt}
\setlength{\headsep}{10pt}
\setlength{\parindent}{0pt}
\setlength{\parskip}{1ex plus 0.5ex minus 0.2ex}



\numberwithin{equation}{section}

{\theoremstyle{definition}\newtheorem{definition}{Definition}[section]
\newtheorem{thm}{Theorem}[section]
\newtheorem{cor}[thm]{Corollary}
\newtheorem{comment}[thm]{comment}
\newtheorem{lem}[thm]{Lemma}
\newtheorem{claim}[thm]{Claim}
\newtheorem{prop}[thm]{Proposition}
\theoremstyle{definition}
\newtheorem{defn}[thm]{Definition}
\theoremstyle{remark}
\newtheorem{rem}[thm]{Remark}
\newtheorem{ex}[thm]{Example}
\numberwithin{equation}{section}
\newtheorem{Setting}[thm]{Setting}




\newcommand{\bim}[3]{\mathord{\raisebox{-0.4ex}[0ex][0ex]{\scriptsize $#1$}{#2}\hspace{-0.25ex}\raisebox{-0.4ex}[0ex][0ex]{\scriptsize $#3$}}}





\newcommand{\A}{\mathcal{A}}
\newcommand{\rG}{\mathcal{G}}
\def\a{\mathbf{A}}
\newcommand{\B}{\mathcal{B}} 
\newcommand{\C}{\mathbb{C}}
\def\b{\mathbf{B}} 
\def\c{\mathbf{C}} 
\newcommand{\h}{\mathcal{H}}
\newcommand{\EL}{\mathcal{L}}
\newcommand{\D}{\mathcal{D}}
\newcommand{\I}{\mathcal{I}}
\newcommand{\J}{\mathcal{J}}

\newcommand{\Q}{\mathbb{Q}}

\def\o{\omega}
\def\O{\Omega}
\def\te{\theta}
\def\Te{\Theta}
\def\N{\mathbb{N}}
\def\T{\mathbb{T}}
\def\Z{\mathbb{Z}}
% \def\E{{\mathscr E}}
\newcommand{\E}{\mathbb{E}}
\def\F{\mathscr F}
\def\H{\mathcal H}
\def\K{\mathcal K}
\def\R{\mathbb{R}}
\def\Z{\mathbb Z}

\def\S{\mathcal S}
\def\e{{\sf e}}
\def\m{{\sf m}}
\def\la{\langle}
\def\ra{\rangle}
\def\x{\mathsf{x}}
\def\y{\mathsf{y}}
\def\z{\mathsf{z}}
\newcommand{\ol}[1]{\overline{#1}}
\newcommand{\Aut}{{\rm Aut}}
\newcommand{\spn}{{\rm span}}
\newcommand{\wot}{{\rm wot}}
\newcommand{\sot}{{\rm sot}}
\newcommand{\U}{\mathcal{U}}
\newcommand{\vN}[1]{\{#1\}''}


%%%%%%%% James's Macros %%%%%%%%
\def\sub{\subseteq}
\DeclareMathOperator{\Span}{Span}
\DeclarePairedDelimiterX{\inp}[2]{\langle}{\rangle}{#1, #2}
\newcommand{\mc}[1]{\mathcal{#1}}
\newcommand{\eps}{\epsilon}
\newcommand{\subwk}{\subset_{\rm{weak}}}
\makeatletter
\newcommand*\bigcdot{\mathpalette\bigcdot@{.5}}
\newcommand*\bigcdot@[2]{\mathbin{\vcenter{\hbox{\scalebox{#2}{$\m@th#1\bullet$}}}}}
\makeatother

%%%%%%%%%%%%%%%%%%%%%%%%%%%%%%%%

\def\r{{\rm r}}
\def\d{{\rm d}}

%\def\({\left(}
%\def\[{\left[}
%\def\){\right)}
%\def\]{\right]}

\def\si{\sigma}
\def\Si{\Sigma}
\def\G{{\sf G}}
\def\trace{{\sf tr}}
\def\CC{\mathbb C}
\def\NC{\mathcal N}
\def\wG{\widehat\G}
\def\p{\parallel}
\def\<{\langle}
\def\>{\rangle}
\providecommand{\norm}[1]{\lVert#1\rVert}
\newcommand{\lr}[1]{\left(#1\right)}
\newcommand{\lrvert}[1]{\left\lvert#1\right\rvert}

\newcommand*\cls[1]{\overline{#1}}


\numberwithin{equation}{section}






\begin{document}

\begin{center}
{\boldmath\LARGE\bf Discrete measured groupoid von Neumann algebras via malleable deformations and $1$-cohomology}


\vspace{1ex}

{\sc Felipe Flores, James Harbour}


\end{center}

\vspace{2ex}

\begin{abstract}\setlength{\parindent}{0pt}\setlength{\parskip}{1ex}\noindent

Given a probability measure preserving groupoid $\mathcal{G}$, we study properties of the corresponding von Neumann algebra $L(\mathcal{G})$ using the techniques of deformation-rigidity theory. Building on work of Sinclair and Hoff, we extend the Gaussian construction for equivalence relations to general measured groupoids. Using Popa's spectral gap argument, we then obtain structural properties about $L(\mathcal{G})$ including primeness and lack of property $(\Gamma)$. We also generalize results of de Santiago, Hayes, Hoff, and Sinclair to characterize maximal rigid subalgebras of $L(\mathcal{G})$ in terms of the corresponding groupoid $L^2$-cohomology.

\end{abstract}

\section{Introduction}



\section{Preliminaries}
\label{preliminaries section}

\subsection{Discrete measured groupoids}\label{meas-group}

We will work with groupoids $\rG$ over a unit space $X\equiv\rG^{(0)}$, identified as small categories in which all the morphisms (arrows) are invertible. 
The source and range maps are denoted by ${\rm d,r}:\rG\to \rG^{(0)}$ and the family of composable pairs by $\,\rG^{(2)}\!\subset\rG\times\rG$\,.  For $x\in\rG\,,\,A, B\subset\rG$ we use the notations
\begin{equation}\label{botations}
AB:=\{ab\,\mid\,a\in A,b\in B,\d(a)=\r(b)\},
\end{equation}
\begin{equation}\label{ottations}
Ax:= A\{x\}\quad \textup{and}\quad xA:=\{x\}A
\end{equation}
These sets could be void in non-trivial situations. A subset of the unit space $Y \subset \rG^{(0)}$ is called {\it invariant} if $Y=\r(\rG Y)$. 


Suppose that $\rG$ is a groupoid equipped with the structure of a standard Borel space 
such that the composition and the inverse map are Borel and 
$\d^{-1}(\{x\})$ is countable for all $x\in\rG^{(0)}$. Then the source and target maps are measurable, $\rG^{(0)}\subset\rG$ is 
a Borel subset, and $\d^{-1}(\{x\})$ is countable.


Now let $\mu$ be a probability measure on 
the set of units $\rG^{(0)}$. Then, 
for any measurable subset $A\subset\rG$, the function 
$\rG^{(0)}\ni x\mapsto \#\bigl (\d^{-1}(x)\cap A\bigr )$ 
is measurable, and the measure $\mu_\d$ on $\rG$ defined by 
$$
\mu_\d(A)=\int_{\rG^{(0)}} \#\bigl (\d^{-1}(x)\cap A\bigr )d\mu (x)=\int_{\rG^{(0)}} \#\bigl (Ax)d\mu (x)
$$
is $\sigma$-finite. The measure $\mu_\r$ is defined in an analogous manner, replacing $\d$ by $\r$. 

\begin{prop}\label{invr} Let $i: x\to x^{-1}$ be the inversion map in $\rG$. The following conditions on $\mu$ are equivalent. 
\begin{enumerate}
\item $\mu_\d=\mu_\r$, 
\item $i_*\mu_\d=\mu_\d$, 
\item for every Borel subset $E\subset\rG$ such that 
$\d|_{ E}$ and $\r|_{E}$ are injective we have 
$\mu(\d(E))=\mu(\r(E))$. 
\end{enumerate}
\end{prop}
Such a probability measure is called {\it invariant} and we denote $\mu_\rG=\mu_\r=\mu_\d$.

\begin{defn}\label{measuredgroupoid}
A discrete, measurable groupoid $\rG$ together with an invariant
probability measure 
on $\rG^{(0)}$ is 
called a {\it discrete measured groupoid}. 
\end{defn}

For $A\subset\rG^{(0)}$ one uses the standard notations 
\begin{equation*}\label{faneaka}
\rG_A\!:={\rm d}^{-1}(A)\,,\quad\rG^A\!:={\rm r}^{-1}(A)\,,\quad\rG_A^A:=\rG_A\cap\rG^A.
\end{equation*} If $A$ is Borel, then we equip $\rG_A^A$ with the normalized measure $\frac{1}{\mu(A)}\mu|_{A}$, so it becomes 
a discrete measured groupoid, called the {\it restriction} of 
$\rG$ to $A$ and is denoted by $\rG|_A$. 




As usual, the set of complex-valued, measurable, essentially bounded 
functions (modulo almost null functions) 
on $\rG$ with respect to $\mu_\rG$ is denoted by 
$L^\infty (\rG,\mu_\rG)$. For a function $\phi:\rG\rightarrow\CC$ and 
$x\in\rG^{(0)}$ we put 
\begin{equation*}\begin{aligned}
S(\phi)(x)&=\#\left\{g\in\rG\mid ~\phi(g)\ne
  0,s(g)=x\right\},\\ 
T(\phi)(x)&=\#\left\{g\in\rG\mid~\phi(g)\ne
  0,t(g)=x\right\}.
\end{aligned}
\end{equation*}
The {\it groupoid ring} $\CC\rG$ of $\rG$ is defined as 
$$
\CC\rG=\left\{\phi\in L^\infty(\rG,\mu_\rG)\mid \text{$S(\phi)$ and $T(\phi)$
    are essentially bounded on $\rG^{(0)}$}\right\}.
$$
$\CC\rG$ is a $^*$-algebra containing 
$L^\infty(\rG^{(0)})\equiv L^\infty(\rG^{(0)},\mu)$ 
as a subring. Multiplication 
is given by the convolution product 
\begin{equation}
\phi*\eta(x)=\sum_{yz=x}\phi(y)\eta(z)
\end{equation}
and the involution is defined by 
\begin{equation}
\phi^*(x)=\overline{\phi(x^{-1})}.
\end{equation}
The groupoid ring $\CC\rG$ of discrete measured groupoid $\rG$ 
is a weakly dense $^*$-subalgebra in the {\it von Neumann algebra} $L(\rG)$ of $\rG$. In fact, $L(\rG)$ is defined as the WOT-closure of $\CC\rG$, when the latter is $^*$-represented as convolution operators $$L_\psi(\xi)(x)=\psi*\xi(x)= \sum_{yz=x}\psi(y)\xi(z),\quad\textup{   for }\psi\in \CC\rG, \xi\in L^2(\rG,\mu_{\rG}).$$

The von Neumann algebra $L(\rG)$ has a finite trace
$\trace_{L(\rG)}$ induced by the invariant measure $\mu$. 
For $\phi\in\CC\rG\subset L(\rG)$ we have 
\begin{equation}
\trace_{L(\rG)}(\phi)=\int_{\rG^{(0)}}\phi(x)d\mu(x).
\end{equation}


\begin{defn}
  A (Borel) \textit{bisection} of $ \rG $ is a Borel subset $ \sigma\sub \rG $ such that the sets $ \sigma x $ and $ x\sigma $ have at most $ 1 $ element for every $ x\in G^{0} $. Borel bisections form an inverse semigroup with respect to the operation on sets introduced in \ref{botations}.
  
  
  The \textit{full pseudogroup} $ [[\rG]] $ of $ \rG $ is the inverse semigroup consisting of Borel bisections modulo the relation of being equal almost everywhere. The \textit{full group} $ [\rG] $ is the subset of $ [[\rG]] $ consisting of the Borel bisections $ \sigma $ such that $ \sigma\sigma^{-1} = \sigma^{-1}\sigma = \rG^{(0)} $. When $\sigma\in [\rG]$, $ \sigma x $ and $ x\sigma $ have exactly one element, so we identify them with said element.
\end{defn}

We now turn our attention to some of the main examples of discrete measured groupoids and their von Neumann algebras.


\begin{ex}\label{transformation}
Suppose $\G$ is a countable group acting in the standard probability space $(X,\mu)$, denoted by $g\cdot x\equiv g\cdot_\theta x$, for $g\in\G, x\in X$. The {\it transformation groupoid} $\rG:=\G\ltimes_\theta\!X$ has $\G\times X$ as underlying set. The multiplication is $$(g,x)(h,g^{-1}\cdot x):=(gh,x)$$ and inversion reads $$(g,x)^{-1}:=\big(g^{-1}\!,g^{-1}\cdot x\big).$$ So $\rG^{(0)}=\{\e\}\times X$ gets identified with $X$, so $\r(g,x)=x$ and $\d(g,x)=g^{-1}\cdot x$. In this case, $L(\rG)\cong L^\infty(X,\mu)\rtimes_\theta\G$ with its usual trace. In particular, if $X=\{x_0\}$ is a singleton, this construction gives the group algebra $L(\G)$ again with its usual trace.
\end{ex}

\begin{ex}\label{equivrel}
Let $(X,\mu)$ be a standard probability space and $\mathcal R\subset X\times X$ be an equivalence relation which is a measurable subset. $\mathcal R$ becomes a a discrete measured groupoid with the operations 
$$
\d(x,y)=(y,y)\,,\quad \r(x,y)=(x,x)\,,\quad (x,y)(y,z)=(x,z)\,,\quad(x,y)^{-1}\!=(y,x)\,.
$$
The unit space is $\mathcal R^{(0)}={\sf Diag}(X)$ and we identify it with $X$, via the map $(x,x)\mapsto x$. We say that $\mathcal R$ is measure preserving if the resulting groupoid is a discrete measured groupoid. The algebra $L(\mathcal R)$ introduced here coincides with the usual equivalence relation Von Neumann algebra (typically introduced using the full pseudogroup). See \textcite[Section 2.2]{hoff:16}. 
\end{ex} 

\begin{rem}\label{nonfree}
    A particular way to obtain equivalence relations is via group actions. Namely, if $\G$ is a countable group acting on a measure preserving way on the probability space $(X,\mu)$, we can form the equivalence relation $\mathcal R_{\G,X}$ given by the orbits of the action, more explicitly
    $$
    \mathcal R_{\G,X}=\{(x,y)\in X\times X \mid y=g\cdot x, \textup{ for some }g\in\G\},
    $$ 
    with the inherited Borel structure of $X\times X$. It is a famous result of Feldman and Moore that every discrete measured equivalence relation can be realized as $\mathcal R_{\G,X}$ for some pmp group action \cite[Theorem 1]{FM1:77}.
    
    Is worth noting that, when the action is essentially free (meaning that for almost all $x\in X$, the stabilizer of $x$ is trivial), the groupoids $\mathcal R_{\G,X}$ and $\G\ltimes_\theta\!X$ generate the same von Neumann algebra \cite{FM2:77}. However, in the non-free case, only the transformation groupoid recovers the crossed product construction. This fact illustrates that, if one wishes to study crossed products or more general algebras, one invariably has to consider groupoids.
\end{rem}


\begin{ex}\label{prodgrpds}
  Let $ \rG_{1} $, $ \rG_{2} $ be groupoids. We define a groupoid structure on the product $ \rG_{1}\times \rG_{2} $ as follows. The unit space is $ (\rG_{1}\times\rG_{2})^{(0)} = \rG_{1}^{(0)}\times\rG_{2}^{(0)} $, the maps $\r,\d$ are defined by $ \d(g_{1},g_{2}) = (\d(g_{1}),\d(g_{2})) $, $ \r(g_{1},g_{2}) = (\r(g_{1}),\r(g_{2})) $ and the operations are defined pointwise. If $ \rG_{1} $ and $ \rG_{2} $ are discrete measured groupoids, then so is $ \rG_{1}\times \rG_{2} $ by taking the product measure.
\end{ex}

\begin{defn}
    A discrete measured groupoid $\rG$ is called {\it ergodic} if $\mu(Y)\in\{0,1\}$ for every Borel invariant subset $Y\subset \rG^{(0)}$.
\end{defn}

We will deal mostly with ergodic groupoids, so it seems convenient to examine when our examples satisfy this condition.

\begin{rem}
The transformation groupoid introduced in Example \ref{transformation} is ergodic precisely when the action $\theta$ is ergodic. Moreover, note that, for $(g,x)\in \rG$ and $y\in Y$ with $\d(g,x)=y$, we have 
$$
\r\big((g,x)(\e,y)\big)=\r\big((g,x)\big)=x=g\cdot_\theta y, 
$$
so $\r(\rG Y)=\G\cdot_\theta Y$ and $\G\ltimes_\theta\!X$ is ergodic if and only if $\mu(\G\cdot_\theta Y)\in\{0,1\}$ for every borel subset $Y\subset X$. In particular, every group is an ergodic groupoid.
\end{rem}

\begin{rem}
In the case of equivalence relation groupoids $\mathcal R$ (Example \ref{equivrel}), we have $$
\r\big((x,y)(y,y)\big)=\r(x,y)=x, 
$$
so $\r(\rG Y)=\{x\in X\mid x\sim_{\mathcal R} y,\textup{ for some }y\in Y\}=: [Y]_{\mathcal R}$ and $\mathcal R$ is ergodic if $\mu([Y]_{\mathcal R})\in\{0,1\} $, for every Borel subset $Y\subset X$.

\end{rem}

\begin{rem}
A direct product of discrete measured groupoids $\rG_1\times \rG_2$ is ergodic if and only if both $\rG_1$ and $\rG_2$ are ergodic. 
\end{rem}

\subsection{Groupoid extensions}

For a discrete measured groupoid $\rG$, we recall that ${\rm Iso}(\rG)$ and $\mathcal R_\rG$ denote the isotropy subgroupoid and the equivalence relation induced by $\rG$. They are given by $${\rm Iso}(\rG)=\{g\in\rG\mid \r(g)=\d(g)\}$$ and $$\mathcal R_\rG=\{(x,y)\in X\times X\mid \exists g\in\rG\text{ with }\r(g)=x \text{ and }\d(g)=y\}$$








\subsection{Unitary representations and $1$-cohomology}

Given a collection of Hilbert spaces $\{\H_x\}_{x \in X}$, the Hilbert bundle $X \ast \H$ is the set of pairs $X \ast \H = \{(x, \xi_x) : x \in X, \xi_x \in \H_x\}$. 
A section $\xi$ of $X \ast \H$ is a map $x \mapsto \xi_x \in \H_x$. 

A {\it measurable Hilbert bundle} is a Hilbert bundle $X \ast \H$ endowed with a $\sigma$-algebra generated by the maps $\{(x, \xi_x) \mapsto \<\xi_x, \xi^n_x\>\}_{n = 1}^\infty$ for a {\it fundamental sequence of sections} $\{\xi^n\}_{n = 1}^\infty$ satisfying 

$(i)$ $\H_x = \ol{\spn \{\xi^n_x\}_{n = 1}^\infty}$ for each $x \in X$, and 

$(ii)$ the maps $\{x \mapsto \|\xi^n_x\|\}_{n = 1}^\infty$ are measurable. 

It is a useful fact that the $\sigma$-algebra of any measurable Hilbert bundle can be generated by an {\it orthonormal fundamental sequence of sections}, i.e. sections which moreover satisfy 

$(iii)$ $\{\xi^n_x\}_{n = 1}^{\infty}$ is an orthonormal basis of $\H_x$ for $x \in X$ with $\dim \H_x = \infty$, and if $\dim \H_x < \infty$, the sequence $\{\xi^n_x\}_{n = 1}^{\dim \H_x}$ is an orthonormal basis and $\xi^n_x = 0$ for $n > \dim \H_x$. 
\\

A {\it measurable section} of $X \ast \H$ is a section $\xi$ such that $x \mapsto (x,\xi_x) \in X \ast \H$ is a measurable map, or equivalently, such that the maps $\{x \mapsto \<\xi_x, \xi^n_x\>\}_{n = 1}^\infty$ are measurable for the fundamental sequence of sections $\{\xi^n\}_{n = 1}^{\infty}$. We let $S(X \ast \H)$ denote the vector space of measurable sections, identifying $\mu$-a.e. equal sections. It is also useful to reserve some notation for the sections with constant norm: 
$$
S_1(X \ast \H)=\{\xi\in S(X \ast \H) \mid \norm{\xi_x}_{\H_x}=1 \textup{ a.e.}\}.
$$ The elements in $S_1(X \ast \H)$ are called {\it normalized } sections. As hinted, wee will often abuse the notation and confuse the map  $x \mapsto (x, \xi_x)$ with $\xi$.  We then consider the {\it direct integral} 
\begin{align*}
\int_X^{\oplus} \H_x d\mu(x) = \{\xi \in S(X \ast \H) : \int_X \|\xi(x)\|^2 d\mu(x) < \infty\}
\end{align*}
which is a Hilbert space with inner product $\<\xi, \eta\> = \int_X \<\xi_x, \eta_x\> d\mu(x)$. If $a \in L^\infty(X)$ and $\xi \in \int_X^{\oplus} \H_x d\mu(x)$ we denote by $a\xi$ or $\xi a$ the element of $\int_X^{\oplus} \H_x d\mu(x)$ given by $[a\xi](x) = [\xi a](x) = a(x)\xi_x$. If $\{\xi^n\}_{n = 1}^{\infty}$ is an orthonormal fundamental sequence of sections, any $\xi \in \int_X^{\oplus} \H_x d\mu(x)$ has an expansion $\xi = \sum_{n = 1}^\infty a_n\xi^n$ where $a_n\in L^\infty(X)$ is given by $a_n(x) = \<\xi_x, \xi^n_x\>_{\H_x} $. 

\begin{defn}\label{unitaryrep}
A {\it unitary (resp. orthogonal) representation} of $\rG$ on a complex (real) measurable Hilbert bundle $X \ast \H$, with $X=\rG^{(0)}$ and a map $\rG\ni g\mapsto \pi(g) \in \U(\H_{\d(g)}, \H_{\r(g)})$ (in the real case, $\U(\H, \K)$ denotes the set of orthogonal maps from $\H$ onto $\K$) such that
$$
\pi(gh)=\pi(g)\pi(h), \quad\textup{ for almost all } (g,h)\in\rG^{(2)}
$$
and such that $\rG\ni g \mapsto \<\pi(g)\xi_{\d(g)}, \eta_{\r(g)}\>$ is a measurable map, for all $\xi, \eta \in S(X \ast \H)$. 
\end{defn}

\begin{ex}
Given a measurable Hilbert bundle $X \ast \K$ with orthonormal fundamental sequence $\Xi=\{\xi^n\}_{n=1}^\infty$, one can always define the {\it identity representation} ${\rm id}_\Xi$ by the formula 
$$
{\rm id}_\Xi(g)\xi^n_{\d(g)}=\xi^n_{\r(g)},
$$ for each $g\in\rG$, $n\in\N$.
\end{ex}

\begin{ex} \label{leftreg}
The {\it (left) regular representation} $\lambda_\rG$ of $\rG$ is obtained by taking $\H_x = \ell^2(\rG^x)$ for each $x \in X=\rG^{(0)}$, and form the measurable Hilbert bundle $X \ast \H$ with any fundamental sequence such that $S(X \ast \H)=\{\xi\textup{ is a measurable function}\mid \xi_x\in \H_{\r(x)}\}$. An example of such a fundamental sequence is achieved by fixing a topology in $\rG$ that generates the Borel structure and taking any sup-norm dense sequence $\{f_n\}_{n=1}^\infty$ in $\mathcal C_{\rm c}(\rG)$ to define $\xi_x^n=f_n|_{\rG^x}$.

The action of $\rG$ is given by 
$$
\lambda_\rG(g)\xi(h)=\xi(g^{-1}h),\quad\textup{ for } (g,h)\in\rG^{(2)}. 
$$  In this case, one has in a natural way that $$
\int_X^\oplus \H_x\dd \mu(x)\cong L^2(\rG,\mu_\rG). 
$$ 
More suggestively, for $ g\in \rG $ let $ \delta_{g}\in \ell^{2}(\rG^{\r(g)}) $ be the indicator function of $ \{g\}\sub \rG^{\r(g)}$. Then $ \lambda_\rG(g) \delta_{h} = \delta_{gh} $ for all $ (g,h)\in \rG^{(2)} $. This induces a representation $ [[\lambda_\rG]] $ of $ [[\rG]] $ on $ L^{2}(\rG) $. Then $ L(\rG) $ may also be defined as the von Neumann algebra generated by the partial isometries $ v_{\sigma} := [[\lambda_\rG]](\sigma)\in \mathbb B(L^{2}(\rG,\mu_\rG)) $.
\end{ex}

\begin{ex}\label{tensors}
    Given two representations $\pi_i$ on $X \ast \H^i$ ($i=1,2$), one may form their tensor product $\pi_1\otimes\pi_2$ by constructing the Hilbert bundle $X \ast \H$, with fibers $\H_x=\H_x^1\otimes \H_x^2$ and a fundamental sequence given by $\xi^{i,j}=\xi^i_1\otimes \xi_2^j$, in terms of the fundamental sequences $\{\xi_i^n\}_{n=0}^\infty$ of $X\ast \H^i$. The formula for $\pi_1\otimes\pi_2$ on simple tensors is $$\pi_1\otimes\pi_2(g)(\xi_1\otimes\xi_2)=\pi_1(g)\xi_1\otimes\pi_2(g)\xi_2.$$
\end{ex}


\begin{defn} \label{equiv}
Given representations $\pi$ on $X \ast \H$ and $\rho$ on $X \ast \K$, we say that $\pi$ and $\rho$ are {\it unitarily equivalent} if there is a family of unitaries $\{U_x \in \U(\H_x, \K_x)\}_{x \in X}$ with 
$$
U_{\r(g)}\pi(g) = \rho(g)U_{\d(g)} \quad \text{for all }g \in \rG,
$$
and such that $x \mapsto U_x\xi_x$ is in $S(X \ast \K)$ for each $\xi \in S(X \ast \H)$. 
\end{defn}

\begin{defn}\label{weakcont}
We say that $\pi$ is {\it weakly contained} in $\rho$, denoted $\pi \prec \rho$, if for any $\epsilon > 0$, $\xi \in S(X \ast \H)$, and $E \subset \rG$ with $\mu_\rG(E) < \infty$, there exists $\{\eta^1, \dots, \eta^m\} \subset S(X \ast \K)$ with
$$
\mu_\rG(\{g\in E : |\< \pi(g)\xi_{\d(g)}, \xi_{\r(g)} \> - \sum_{i = 1}^m \< \rho(g)\eta^i_{\d(g)}, \eta^i_{\r(g)} \>| \ge \epsilon\}) < \epsilon  
$$ 
\end{defn}

\textcolor{magenta}{Where does this definition comes from?}


Now we introduce the definitions of weakly mixing and mixing representations. They were taken from \cite[Definition 3.10]{gardella:17} and \cite[Definition 4.4]{kida:17}, respectively.

\begin{defn}\label{mixingreps}
  Let $\pi$ be a representation of a discrete discrete measured groupoid $\rG$ on a Hilbert bundle $X\ast \H$. Then $\pi$ will be called \begin{enumerate}
      \item[(i)]  \textit{weak mixing} if, for every $\epsilon >0$ and every $n\in \N$, and sections $ \eta^1,\ldots, \eta^n\in S(X\ast\H) $, there exists $ \si\in [\rG] $ such that
\[
  \int_{X} |\langle\eta^i_x,\pi(x\si)\eta_{\d(x\sigma)}^j\rangle| \dd{\mu(x)} \leq \epsilon
\]
for every $i,j=1,\ldots,n$.
\item[(ii)] {\it mixing} or $c_0$ if 
for every $\epsilon, \delta > 0$ and every pair of normalized sections $\xi, \eta \in S(X \ast \H)$, there is $E \subset X$ with $\mu(X \setminus E) < \delta$ such that 
$$
\left|\{g \in \rG^{E}_x : |\<\pi(g)\xi_x, \eta_{\r(g)}\>| > \epsilon\}\right| < \infty  \quad \text{for $\mu$-a.e. $x \in E$.}
$$
  \end{enumerate} 
\end{defn}


\begin{defn}\label{cohom}
    A {\it $1$-cocycle} for a representation $\pi$ on $X \ast \H$ is a measurable map $\rG\ni g \mapsto b(g) \in \H_{\r(g)}\subset X\ast \H$ such that 
\begin{equation}\label{cocycle} 
b(gh) = b(g) + \pi(g)b(h) \quad \text{for all } (g,h)\in \rG^{(2)}.
\end{equation} The $1$-cocycle $b$ is a {\it $1$-coboundary} if there is a measurable section $\xi$ of $X \ast \H$ such that 
\begin{equation}\label{cobound} 
b(g) = \xi_{\r(g)} - \pi(g)\xi_{\d(g)} \quad \textup{for  $\mu_\rG$-a.e. $g\in\rG$}.
\end{equation}
A pair of $1$-cocycles $b$ and $b'$ are {\it cohomologous} if $b - b'$ is a $1$-coboundary. The set of $1$-cocycles of $\pi$ is denoted $Z^1(\rG,\pi)$ and the set of $1$-coboundaries by $B^1(\rG,\pi)$ the quotient 
$$
H^1(\rG,\pi)=Z^1(\rG,\pi)/B^1(\rG,\pi)
$$ is called the $1$-cohomology group of the representation $\pi$ and it is typically endowed with the quotient topology after giving $Z^1(\rG,\pi)$ the topology of convergence in the measure $\mu$.
\end{defn}



The following result is due to \textcite[Theorem 3.19, Lemma 3.20]{anatharaman:05}.

\begin{lem}\label{bound}
Let $b$ be a 1-coboundary associated to the representation $\pi$ of $\rG$ in $X\ast \H$, then there exists a Borel subset $E\subset X$ of positive measure, such that for every $x\in E$, $\sup\{\norm{b(g)}\mid g\in \rG_E^x\}<\infty$. If $\rG$ is ergodic, both conditions are equivalent and $E$ can be chosen to have measure $1$.
\end{lem}

\begin{rem}
    The $1$-cocycles satisfying the second condition of the lemma are often called {\it bounded $1$-cocycles}.
\end{rem}



The following lemma is due to \textcite[Lemmas 2.1, 2.2]{hoff:16}, at least in its form for equivalence relations.

\begin{lem}\label{unbound}      
    Let $b$ be a 1-coboundary associated to the representation $\pi$ of $\rG$ in $X\ast \H$. Then $ b \in Z^{1}(\rG,\pi)\setminus B^{1}(\rG,\pi) $, i.e. $ b $ is \textit{unbounded} if and only if there exists a $ \delta >0 $ such that for all $ R>0 $ there is a $ \sigma\in [\rG] $ such that 
    \[
        \mu(\{x\in X: \norm{b(x\sigma)} \geq R\} > \delta.
    \]
\end{lem}

\begin{proof}
    We seek to reduce to the corresponding lemma for equivalence relations in \cite[Lemma 2.2]{hoff:16}. As such, note that by the above claim, $\widetilde{b}$ is also unbounded.
\end{proof}










\subsection{Malleable deformations}

In the following we survey the basics of Sorin Popa's deformation rigidity theory as well as various relevant approaches/results from \textcite{dSHH:21} which motivate this paper's main results. \textcolor{olive}{TODO: Add references to Popa's seminal works in def/rig}

The intuitive idea behind deformation/rigidity theory is to study rigidity results for a von Neumann algebra $ M $ which can deformed inside another algebra $ \widetilde{M}\supseteq M $ by an action $ \alpha: \R \to \Aut(\widetilde{M}) $ whilst containing subalgebras which are \textit{rigid} with respect to the deformation.

Let $ (M,\tau) $ be a tracial von Neumann algebra and $ \Aut(M) $ the group of trace-preserving $ * $-automorphism of $ M $. Then we have the following fundamental definition due to Popa.

\begin{defn}[Popa]
  Let $ \widetilde{M}\supseteq M $ be a trace-preserving inclusion of tracial von Neumann algebras.
  \begin{enumerate}
    \item A \textit{malleable deformation $ \alpha $ of $ M $ inside $ \widetilde{M} $} is a strongly-continuous action $ \alpha:\R\to \Aut(\widetilde{M}) $ such that $ \alpha_{t}(x)\xrightarrow{\norm{\cdot}_{2}}x $ as $ t\to 0 $ for every $ x\in \widetilde{M} $.
    \item An \textit{s-malleable deformation $ (\alpha,\beta) $ of $ M $ inside $ \widetilde{M} $ } is a malleable deformation $ \alpha $ combined with a distinguished involution $ \beta\in\Aut(\widetilde{M}) $ such that $ \beta\vert_{M} = id $ and $ \beta \alpha_{t} = \alpha_{-t} \beta $ for all $ t\in \R $.
  \end{enumerate}
\end{defn}


On its own, deformations do not give much information about the algebra itself; however, they do provide one with a quantitative way to locate subalgebras with prescribed properties that force them to be \textit{rigid} with respect to the deformation. Explicitly, given a malleable deformation $ \alpha $ of $ M $ inside $ \widetilde{M} $, a subalgebra $ Q\sub M $ is $ \alpha $-\textit{rigid} if the deformation converges uniformly on the unit ball of $ Q $, i.e.
$$
  \lim_{t\to 0}\sup_{x\in (Q)_{1}} \norm{\alpha_{t}(x)-x}_{2} =0.
$$

We now introduce the more recent notion of maximal rigidity for subalgebras studied in \cite{dSHH:21}.\textcolor{olive}{TODO: Expand upon this and why it helps us}

\begin{defn}[\textcite{dSHH:21}]
  Let $ (\alpha,\beta) $ be an $ s $-malleable deformation $ M $ inside $ \widetilde{M} $ where $ M $ and $ \widetilde{M} $ are both assumed to be finite. Then an $ \alpha $-rigid subalgebra $ Q\sub M $ is \textit{maximal $ \alpha $-rigid} if whenever $ P\sub M $ is an $ \alpha $-rigid subalgebra containing $ Q $, it follows that $ P = Q $.
\end{defn}


\begin{defn}
  Let $ \alpha $ be a malleable deformation of $ M $ inside $ \widetilde{M} $ where $ M$, $\widetilde{M} $ are finite. Suppose that $ Q\sub M $ is an $ \alpha $-rigid subalgebra of $ M $. Then a subalgebra $ P\sub M $ is an \textit{$\alpha$-rigid envelope of $ Q $} if 
  \begin{itemize}
    \item $ P $ is $ \alpha $-rigid
    \item $ P\supseteq Q $
    \item if $ N \sub M $ is $ \alpha $-rigid and $ N\supseteq Q $, then $ N\sub P $.
  \end{itemize}
\end{defn}

One would be justified in being skeptical as to whether rigid envelopes even exist for given subalgebras. Indeed, they do not exist in general; however, some of the main results in \textcite{dSHH:21} show that in many natural cases they do. We shall use these results in crucial ways and thus sketch them here for reference.

\begin{thm}[\textcite{dSHH:21} 1.2] \label{rigenvelope}
  Let $ (\alpha,\beta) $ be an $ s $-malleable deformation of tracial von Neumann algebras $ M\sub \widetilde{M} $. Then any $ \alpha $-rigid subalgebra $ Q\sub M $ with $ Q^{\prime}\cap \widetilde{M}\sub M $ is contained in a unique maximal $ \alpha $-rigid subalgebra $ P\sub M $.
\end{thm}

In the following we will utilize Popa's spectral gap argument. We include the relevant definitions for Hilbert bimodules as well. For a presentation of this version of the argument, see \cite[Theorem 3.2]{hoff:16}

\begin{defn}
    let $ N\subseteq M $ be a von Neumann subalgebra. An $M$-$M$ bimodule $ _M \H_M $ is said to be \textit{mixing relative to $ N $} if for any sequence $ (x_{n})_{n=1}^{\infty} $ in $ (M)_{1} $ such that $ \norm{\E_{N}(yx_{n}z)}_{2} \to 0 $ for every $ y,z\in M $, we have that 
    \[
        \lim_{n\to \infty} \sup_{y\in(M)_{1}}|\inp{x_{n} \xi y}{\eta}| = \lim_{n\to \infty} \sup_{y\in(M)_{1}}|\inp{y \xi x_{n} }{\eta}| = 0 \quad \text{ for all } \xi,\eta\in \H
    \]

\end{defn}

\begin{defn}
    An $ M $-$ N $ bimodule $ _M\H_N $ is said to be \textit{weakly contained} in an $ M $-$ N $ bimodule $ _M\K_N $, written $ _M\H_N \prec _M\K_N$, if for any $ \epsilon>0 $, finite subsets $ F_{1} \sub M $, $ F_{2}\sub N $, and $ \xi\in \H $, there are $ \eta_{1},\ldots, \eta_{n}\in \K $ such that 
    \[
        |\inp{x \xi y}{\xi} - \sum_{j=1}^{n}\inp{x \eta_{j}y }{\eta_{j}} | < \epsilon \quad \text{for all } x\in F_{1}, y\in F_{2}
    \]

\end{defn}


\begin{thm}[Popa's Spectral Gap Argument]\label{popaspectralgap}
    Let $ (\alpha,\beta) $ be an $ s $-malleable deformation of tracial von Neumann algebras $ M\sub \widetilde{M} $ and assume that $M$ has no amenable direct summands. Suppose further that that the orthocomplement bimodule $_{M} L^2(\widetilde{M}) \ominus L^2(M)_M $ is weakly contained in the coarse $M$-$M$ bimodule and mixing relative to an abelian subalgebra $A\sub M$.

    Then there is a central projection $z\in Z(M)$ such that 
    \begin{enumerate}
        \item $ \sup_{x\in (Mz)_{1}} \norm{\alpha_{t}(x)-x}_{2} \xrightarrow{t\to0}0 < +\infty $
        \item $M(1-z)$ is prime.
    \end{enumerate}
    
    
\end{thm}









\section{Gaussian extension of $\rG$ and the $s$-malleable deformation of $L(\rG)$} 

In this section we construct the s-malleable deformation that will be used to prove the main results. Gaussian actions have been used to construct $s$-malleable deformation for group von Neumann algebras in \cite{dSHH:21, ps:12, sinclair:11} and for equivalence relation von Neumann algebras in \cite{hoff:16}. We will follow the same reasoning in our wider context. 

\subsection{The Gaussian extension of $\rG$} 
Let $ \rG $ be a discrete ¿measured groupoid, $\pi$ an orthogonal representation of $\rG$ on a real Hilbert bundle $X \ast \H$, and let $\{\xi^n\}_{n = 1}^\infty$ be an orthonormal fundamental sequence of sections for $X \ast \H$. For $x\in X$, we consider the measure space
\begin{equation}
    (\Omega_x, \nu_x) = \prod_{i = 1}^{\dim \H_x} (\R, \tfrac{1}{\sqrt{2\pi}}e^{-s^2/2}ds),
\end{equation}
and define $\omega_x: \spn_\R (\{\xi_x^n\}_{n = 1}^{\dim \H_x}) \to \U(L^\infty(\Omega_x))$ by
\begin{equation}
    \omega_x\left(\sum_{n = 1}^{\dim \H_x} a_n\xi_x^{n}\right) = \exp\left({i\sqrt{2}\sum_{n = 1}^{\dim \H_x} a_nS_x^{n}}\right)
\end{equation} where $S^n_x$ is the $n$th-coordinate function $S_x^n((s_i)_{i = 1}^{\dim \H_x}) = s_n$ for $1\le n \le \dim \H_x$.

Then $\omega_x$ extends to a $\|\cdot\|_{\H_x} - \|\cdot\|_2$ continuous map $\omega_x : \H_x \to \U(L^\infty(\Omega_x))$ satisfying 
\begin{equation} \label{axioms}
\tau(\omega_x(\xi)) = e^{-\|\xi\|^2_{\H_x}}, \quad \omega_x(\xi + \eta) = \omega_x(\xi)\omega_x(\eta), \quad \omega_x(-\xi) = \omega_x(\xi)^*, \quad \forall\xi, \eta \in\H_x.
\end{equation}

For $x \in X$, one also has $D_x = \spn_\CC (\{\omega_x(\xi)\}_{\xi \in \H_x})\subset L^\infty(\Omega_x)$ is WOT-dense in $L^\infty(\Omega_x)$. Now for every $g\in\rG$, define a $*$-homomorphism $\rho(g): D_{\d(g)} \to L^\infty(\Omega_{\r(g)})$ by 
$$
\rho(g)\omega_{\d(g)}(\xi) = \omega_{\r(g)}(\pi(g)\xi),
$$
which is well defined and $\|\cdot\|_2$-isometric since \eqref{axioms} implies 
$$
\tau(\omega_{\d(g)}(\eta)^*\omega_{\d(g)}(\xi)) = \tau(\omega_{\r(g)}(\pi(g)\eta)^*\omega_{\r(g)}(\pi(g)\xi)) \quad \forall\xi, \eta \in \H_{\d(g)}.
$$
So extends $\rho(g)$ extends to a trace-preserving $*$-isomorphism $\rho(g): L^\infty(\Omega_{\d(g)}) \to L^\infty(\Omega_{\r(g)})$.  
Let $\theta_{g}: \Omega_{\d(g)} \to \Omega_{\r(g)}$ be the induced measure space isomorphism such that $\rho(g)\phi = \phi\circ \theta_{g}^{-1}$ for all $\phi \in L^\infty(\Omega_{\r(g)})$. 

So far, this construction provides an $X$-measurable bundle of commutative Von Neumann algebras $\mathcal B=\{L^\infty(\Omega_{x})\}_{x\in X}$ and note that the maps $\{\rho(g)\}_{g\in \rG}$ give us an action of $\rG$ on $\mathcal B$, so the natural thing to do is to produce a $s$-malleable deformation of $L(\rG)$ inside the (groupoid) crossed product.

We consider $\widetilde X\equiv X \ast \Omega=\bigsqcup_{x\in X}\{x\}\times \Omega_x$ as a measurable bundle with $\sigma$-algebra generated by the maps $(x, r) \mapsto \omega_x(\sum_{i \in I}a_i\xi^i_x)(r)$ for $I \subset \N$ finite and $a_i \in \R$. The natural measure $\mu \ast \nu$ on $X \ast \Omega$ is given by $[\mu \ast \nu](E) = \int_X \nu_x(E_x)d\mu(x)$, where $E_x = \{s \in \Omega_x: (x, s) \in E\}$. We define the \emph{Gaussian extension of $\rG$} to be the transformation groupoid $\widetilde \rG=\rG\ltimes_\theta (X \ast \Omega)$, explicitly given by 
\begin{itemize}
    \item As a set, $\widetilde \rG=\{(g,x,r)\in \rG\times  (X \ast \Omega)  \mid \r(g)=x\}$. The unit space is identified with $X \ast \Omega$
    \item The groupoid operations are 
    \begin{align*}
        \r(g,x,r)=(x,r), \quad &\quad\d(g,x,r)=(\d(g),\theta_g^{-1}(r)) \\
        (g,x,r)(h,\d(g),\theta_g^{-1}(r))=(gh,x,r), \quad&\quad (g,x,r)^{-1}=\big(g^{-1},\d(g),\theta_g^{-1}(r)\big)
    \end{align*}
    \item $\widetilde \rG$ inherits a natural measurable structure as a subset of the product $\rG\times  (X \ast \Omega) $. Lastly, $\mu*\nu$ plays the role of the invariant probability measure on $X \ast \Omega$.
\end{itemize}

Then we note that $L(\widetilde \rG)\cong \rG \ltimes_\rho \mathcal B$ and that $L(\widetilde \rG)$ contains copies of $L^\infty(\widetilde X,\mu*\nu) $ and $L(\rG)$ such that 
\begin{equation}\label{generated}
    L(\widetilde \rG)=  \vN{L^\infty(X \ast \Omega), \{u_\sigma\}_{\sigma\in[\rG]}}=\vN{L^\infty(X \ast \Omega), L(\rG)} \subset \B(L^2(\widetilde \rG)).
\end{equation} and we have the relation  $u_\sigma \phi u_\sigma^*=\rho(\sigma)\phi$, for $\sigma\in [\rG], \phi\in L^\infty(X\ast \Omega) $, where $\rho$ extends from an action of $\rG$ on $\mathcal B$ to the action of $[\rG]$ on $L^\infty(X\ast \Omega)$ given by \begin{equation}\label{comm}
    \rho(\sigma)\phi(x,r)=
\end{equation}
\textcolor{violet}{Is it that easy?}



\subsection{$s$-Malleable deformation of $L(\rG)$}

Let $ b $ be a 1-cocycle for the representation $ \pi $ on $ X*\H $. Set $ M = L(\rG) $, $ \widetilde{M}:=L(\widetilde{\rG}) $. For $ t\in\R $, define $ c_{t}:\widetilde{\rG}\to \mathbb S^{1} $ by
$$
  c_{t}(g,x,r) = \omega_{x}(tb(g))(r),
$$ which is a multiplicative function: Given $(g,x,r),(h,\d(g),\theta_g^{-1}(r))\in \widetilde{\rG}$, we see that: 
\begin{align*}
    c_t(gh,x,r)&=\omega_{x}(tb(g))(r)\omega_{x}(t\pi(g)b(h))(r) \\
&=\omega_{x}(tb(g))(r)\big[\rho(g)\omega_{\d(g)}(tb(h))\big](r) \\
&=\omega_{x}(tb(g))(r)\omega_{\d(g)}(tb(h))\big(\theta_g^{-1}(r)\big) \\
&=c_t(g,x,r)c_t(h,\d(g),\theta_g^{-1}(r)).
\end{align*} 
For $ t\in\R $ and $ \sigma\in [\widetilde{\rG}] $, let $ f_{c_{t},g}\in \mathcal{U}(L^{\infty}(X\ast \Omega)) $ be given by
$$
  f_{c_{t},\sigma}(x,r) = \omega_{x}(tb(x\sigma))(r)=c_t(x\sigma, x,r), 
$$ and so we obtain an SOT-continuous $\mathbb R$-action $ \alpha_{b,t}\in \Aut(\widetilde{M}) $ by 
$$
  \alpha_{b,t}(au_{\sigma}) = f_{c_{t},\sigma}au_{\sigma}.
$$ \textcolor{violet}{Is it clear that this is an automorphism?}

Now we compute
\begin{align*}
  \tau(f_{c_{t},\sigma}) &= \int_{X\ast \Omega} f_{c_{t},\sigma} \dd{\mu\ast \nu} = \int_{X} \int_{\Omega_{x}} f_{c_{t},\sigma}(x,r) \dd{\nu_{x}}(r) \dd{\mu(x)} \\
  &= \int_{X}\int_{\Omega_{x}} \omega_{x}(tb(x\sigma))(r) \dd{\nu_{x}}(r) \dd{\mu(x)} \\
  %&= \int_{X} \tau(\omega_{x}(tb(x\sigma)))  \dd{\mu(x)} \\
  &=\int_{X} e^{-t^2\norm{b(x\sigma)}^{2}}  \dd{\mu(x)}.
\end{align*} So
\begin{align*}
  \norm{\alpha_{b,t}(au_{\sigma})-au_{\sigma}}_{2}^{2} &= \norm{f_{c_{t},\sigma}au_{\sigma}-au_{\sigma}}_{2}^2 \leq \norm{a}^{2} \norm{f_{c_{t},\sigma}-1}_{2}^{2}  = 2\norm{a}^{2}(1-\Re \tau(f_{c_{t},\sigma})) \\
  &= 2\norm{a}^{2}\lr{1-\int_{X} e^{-t^2\norm{b(x\sigma)}^{2}}\dd{\mu(x)}}\xrightarrow{t\to0}0.
\end{align*}
And the convergence is uniform if and only if $b$ is bounded. Next, note that defining $\beta_x(\omega_x(\xi))=\omega_x(-\xi)=\omega_x(\xi)^*$ for $x\in X$ gives an $^*$-automorphism of $L^\infty(\Omega_x)$, which leads to $\beta\in\Aut(L^\infty(X\ast \Omega)) $ defined by $\beta(a)(x,r)=\beta_x(a(x,\cdot))(r)$, for $a\in L^\infty(X\ast \Omega)$.

\subsection{Maximal rigid subalgebras of $ L(\widetilde\rG) $}

\textcolor{violet}{Prove that $\pi|_\H$ weakly mixing implies that the koopman representation $\kappa$ of $\rG$ of the action of the gaussian construction.}

\begin{prop} \label{maxrig1}
  Let $ \rG $ be a discrete measured groupoid and $ \pi $ an orthogonal representation of $ \rG $ on a real Hilbert bundle $ X\ast \H $. Let $ b\in Z^{1}(\rG,\pi) $ be a cocycle and set 
  $$
    \mathcal{S} := \{g\in \rG : b(g) = 0\}.
  $$
  Then $ \mathcal{S} $ is a wide discrete measured subgroupoid of $ \rG $. Moreover, if $ L(\mathcal{S}) $ is diffuse and $ \pi\vert_{\mathcal{S}} $ is weak mixing, then $ L(\mathcal{S}) $ is a maximal rigid subalgebra for $ \alpha_{b} $.
\end{prop}

\begin{proof}
    We will follow the proof of \cite[Proposition 4.3]{dSHH:21}. $\mathcal S$ is easily seen to be a subgroupoid, every unit $x\in X$ satisfies $x^2=x$ and hence the cocycle identity implies $x\in \S$; Showing closedness for multiplication and inverses is easier. Now we note that $\alpha_{b,t}|_{L(\S)}={\rm id}_{L(\S)}$ and let us now show that the conjugation action of $L^2(\widetilde M)\ominus L^2(M)$ has no nonzero invariant vectors. We observe that, for $\sigma\in [\S]\subset [[\rG]]$, $$
    {\rm Ad}(u_\sigma)\big[ au_\sigma\big]
    $$
\end{proof}

\begin{prop} \label{maxrig2}
  Let $ \rG $ be a discrete measured groupoid and $ \pi $ an orthogonal representation of $ \rG $ on a real Hilbert bundle $ X\ast \H $. Let $ b\in Z^{1}(\rG,\pi) $ be a cocycle and suppose $\mathcal{S}\leq \rG$ is a discrete measured subgroupoid with $L(\mathcal{S})$ diffuse and such that $\pi|_\mathcal{S}$ is weakly mixing and $b|_\mathcal{S}$ is bounded. Let $P$ be the rigid envelope of $L(\mathcal{S})$. Then $P=L(\mathcal{S}')$, where $\mathcal{S}\leq \mathcal{S}'\leq \rG$ and $\mathcal{S}'$ is a maximal subgroupoid satisfying that $b|_{\mathcal{S}'}$ is bounded.
\end{prop}

\begin{proof}
    Since $\mathcal S$ is ergodic, Lemma \ref{bound} gives a measurable section $\xi:X\to X\ast \H$, such that $b(g)=\xi_{\r(g)}-\pi(g)\xi_{\d(g)}$, for all $g\in \mathcal S$. Now define $\tilde b(g):=b(g)-\xi_{\r(g)}+\pi(g)\xi_{\d(g)}$ and its associated subgroupoid
    $$
    \mathcal S'={\rm Ker}\,\tilde b=\{g\in\rG\mid \tilde b(g)=0\}.
    $$
    It follows that $\mathcal S'$ cointains $\mathcal S$ and is maximal under the condition that $b|_{\mathcal{S}'}$ is bounded: if $\mathcal S'\leq \mathcal S''$ and $b(g)=\xi_{\r(g)}'-\pi(g)\xi_{\d(g)}'$ for $g\in \mathcal S''$, then $\xi'-\xi$ is invariant under $\pi|_\mathcal{S}$, contradicting the fact that $\pi|_\mathcal{S}$ is weakly mixing.
    
    Now note that $\tilde b-b$ is a coboundary, so there is $E\subset X$ of measure 1 such that
    $$
    \sup_{g\in \rG_E^x} \norm{\tilde b(g)-b(g)}<\infty.
    $$ Let $\alpha_{\tilde b},\alpha_{b}$ be the associated $s$-malleable deformations inside $\widetilde M$. We observe \begin{align*}
        \norm{(\alpha_{\tilde b,t}-\alpha_{\tilde b,t})\big{|}_M}_{L^2(M)\to L^2(\widetilde M)}^2&\leq \sup_{\sigma\in [\rG]} 2-\tau(\overline{f_{\tilde{c}_t,\sigma}}f_{{c}_t,\sigma}+\overline{f_{{c}_t,\sigma}}f_{\tilde{c}_t,\sigma}) \\ 
        &\leq 2\sup_{\sigma\in [\rG]} \lr{1-\int_{X} e^{-t^2\norm{\tilde b(x\sigma)-b(x\sigma)}^2}\dd{\mu(x)}} \\ 
        &\leq 2-2\int_{X} e^{-t^2\sup_{g\in \rG_E^x} \norm{\tilde b(g)-b(g)}^2}\dd{\mu(x)}
    \end{align*} thus $$
    \lim_{t\to 0} \norm{(\alpha_{\tilde b,t}-\alpha_{\tilde b,t})\big{|}_M}_{\infty,2}^2=0.
    $$ So a diffuse subalgebra $Q\leq M$ is $\alpha_{b}$-rigid if and only if is $\alpha_{\tilde b}$-rigid, so by Proposition \ref{maxrig1}, $L(\mathcal S')$ is maximal rigid for $\alpha_b$ and equals $P$ (Theorem \ref{rigenvelope}).
\end{proof}


\section{Primeness}

%The goal of this section is to prove the following generalization of Theorem A in \textcite{hoff:16}:
%
%\begin{thm}[\textcolor{olive}{Conjectural}]
%  Let $ \rG $ be a discrete, ¿measured groupoid \textcolor{teal}{with no amenable direct summand} \textcolor{olive}{(TODO: explain what the heck this is)} which admits an unbounded 1-cocycle into a mixing orthogonal representation weakly contained in the regular representation. Then $ L(\rG)\not\equiv N \cls{\otimes} Q $ for any type II von Neumann algebras $ N $ and $ Q $ and \textcolor{teal}{hence
%  $$\rG \not \equiv \rG_{1} \times \rG_{2} $$ for any pmp $ \rG_{i} $ which have $ x\rG_{i} $ and $ \rG_{i}x $ infinite for a.e. $ x\in \rG_{i}^{(0)} $}
%
%  \textcolor{olive}{stuff about ergodic implies prime }
%\end{thm}

% \begin{rem}
%   Suppose that $ \rG = G\times \mathcal{R} $ where $ G $ is a countable discrete icc group and $ \mathcal{R} $ is a countable ergodic ¿measured equivalence relation. Then both $ L(\mathcal{R}) $ and $ L(G) $ are type $ II_{1} $ factors and $ L(\rG) \cong L(G)\cls{\otimes} L(\mathcal{R}) $, so $ L(\rG) $ is not prime.

%   Hence, the naive approach to generalizing Hoff's primeness results via translation into the groupoid setting must fail at some point. Regardless, we follow his various constructions until this point.
% \end{rem}

In parallel with Hoff's argument, we first analyze the ``transition'' maps $ \rho(g):L^{\infty}(\Omega_{d(g)}) \to L^{\infty}(\Omega_{r(g)}) $ utilized in the construction of the Gaussian extension of $ \rG $.

Since each fiber $ \Omega_{x} $ is a finite measure space, we have that $ \cls{L^{\infty}(\Omega_{x})}^{\norm{\cdot}_{2}} = L^{2}(\Omega_{x}) $, thus we may extend $ \rho(g) $ to an isometry $ \rho(g):L^{2}(\Omega_{d(g)})\to L^{2}(\Omega_{r(g)}) $. Finally, after restricting, we obtain a unitary map
$$
  \rho(g): L^{2}(\Omega_{d(g)})\ominus \C \to L^{2}(\Omega_{r(g)})\ominus \C
$$

Now form the Hilbert bundle $ X\ast \K $ as follows
\begin{itemize}
  \item $ \K_{x}:=L^{2}(\Omega_{x})\ominus \C $ for $ x\in X $
  \item $ \sigma $-algebra determined by fundamental sections $ \omega_{0}(\Span_{\Q}\{\xi^{n}\}_{n=1}^{\infty}) $, where $ \{\xi^{n}\}_{n=1}^{\infty} $ as before and 
    $$
    [\omega_{0}(\eta)](x) = \omega_{x}(\eta(x)) - \tau(\omega_{x}(\eta(x))) = \omega_{x}(\eta(x)) - e^{-\norm{\eta(x)}^{2}} \text{ for }\eta\in S(X\ast\H).
    $$
\end{itemize}

As $ \rho(gh) = \rho(g) \rho(h) $ for all $ (g,h)\in \rG^{(2)} $, we can regard $ \rho $ as a representation of $ \rG $ on $ X\ast \K $


\begin{lem}\label{fockspace}
  For each $ x\in X $, let $ \widehat{\H}_{x} = \bigoplus_{n=1}^{\infty}(\H_{x}\otimes_{\R}\C)^{\odot n} $. Then the representation $ \rho $ of $ \rG $ on $ X\ast \K $ is unitarily equivalent to the representation $ \widetilde{\pi} = \oplus_{n=1}^{\infty} \pi_{\C}^{\odot n} $ of $ \rG $ on $ X\ast \widehat{\H} $.
\end{lem}

\begin{proof}
  For $ x\in X $, set $ U_{x}:D_{x}\to \C\oplus \widehat{\H}_{x} $ by $ \omega_{x}(\xi)\mapsto e^{-\norm{\xi}^{2}}\bigoplus_{n=0}^{\infty}\frac{(i \sqrt{2} \xi)^{\odot n}}{n!} $ for $ \xi\in \H_{x} $. Note that $ U_{x} $ is well-defined and isometric as 
  $$
    \inp{e^{-\norm{\xi}^{2}}\bigoplus_{n=0}^{\infty}\frac{(i \sqrt{2} \xi)^{\odot n}}{n!}}{e^{-\norm{\eta}^{2}}\bigoplus_{n=0}^{\infty}\frac{(i \sqrt{2} \xi)^{\odot n}}{n!}} = \tau(\omega_{x}(\eta)^{*} \omega_{x}(\xi)).
  $$

  Note that $ \C\sub U_{x}(D_{x}) $ as $ z \cdot\omega_{x}(0)\mapsto z $ for all $ z\in \C $. Moreover, one can inductively check that $ \xi_{1}\odot \cdots \odot \xi_{n} \in \cls{U_{x}(D_{x})} $ for all $ \xi_{i}\in \H_{x} $ and $ n\in \N $. Thus, we can extend $ U_{x} $ to a unitary $ U_{x}: L^{2}(\Omega_{x})\to \C\oplus \widehat{\H}_{x} $.\\

  Fix $g\in \rG$. Then $\xi\in \H_{d(g)}$, we have that 
  \[
    [id_{\C}\oplus \widehat{\pi}](g) U_{d(g)} [\omega_{d(g)}(\xi)] =  U_{d(g)} \omega_{r(g)}(\pi(g)(\xi)) = U_{d(g)} \rho(g)[\omega_{d(g)}(\xi)].
  \]
    Lastly, by density we have that for all $f\in L^2(\Omega_{d(g)})$,
    \[
        [id_{\C}\oplus \widehat{\pi}](g) U_{d(g)} f =U_{d(g)} \rho(g)f.
    \]
    As $U_d(g)$ fixes $\C$, upon restricting to the orthocomplement we obtain a unitary equivalence.
\end{proof}

\subsection{$ L(\rG) $-$ L(\rG) $ bimodule from representation of $ \rG $}

Let $ M = L(\rG) $ and $ A = L^{\infty}(X) \sub M $. Then a representation $ \pi $ of $ \rG $ on a Hilbert bundle $ X\ast \H $ induces a group representation $ \widetilde{\pi}: [\rG]\to \U\lr{\int_{X}^{\oplus}\H_{x} \dd{\mu(x)}} $ by 

\[
  (\widetilde{\pi}(\sigma) \xi)(x) = \pi(r\vert_{\sigma}^{-1}(x)) \ \xi(d\circ r\vert_{\sigma}^{-1}(x))
\]
for $ \sigma\in [\rG] $, $ g\in\rG $, $ x\in X $. Utilizing Connes fusion over $ A $, we may form the $ A $-$ L(\rG) $ bimodule
\[
  \B(\pi) := \left[ \int_{X}^{\oplus}\H_{x}\dd{\mu(x)}\right] \otimes_{A} L^{2}(\rG).
\]
We wish to incorporate the representation $ \pi $ to upgrade $ \B(\pi) $ to an $ M $-$ M $ bimodule.


\begin{prop}\label{reptobim}
    The Hilbert space $ \B(\pi) $ has an $ L(\rG) $-$ L(\rG) $ bimodule structure such that 
    \[
        au_{\sigma}\cdot (\xi \otimes \eta) \cdot x = \widetilde{\pi}(\sigma)(\xi)\otimes au_{\sigma}\eta x
    \]
    for $ a\in A $, $ \sigma\in [\rG] $, $ \xi\in \int_{X}^{\oplus}\H_{x} \dd{\mu(x)} $, $ x\in M $, $ \eta\in L^{2}(\rG) $.
    Moreover, the following assertions hold true:
    \begin{enumerate}[noitemsep]
        \item For all $ \rG $-representations $\pi$ and $\rho$ such that $ \pi \subwk \rho $, we have that
            \[
                _M\B(\pi)_M \subwk {}_M\B(\rho)_M.
            \]
        \item Whenever $ \pi_{1} $ and $ \pi_{2} $ are $ \rG $-representations, we have that
            \[
                _M\B(\pi_{1}\otimes \pi_{2})_M \cong {}_M(\B(\pi_{1})\otimes_M \B(\pi_{2}))_M
            \]
        \item If $ \pi $ is a mixing $ \rG $-representation, then the bimodule $ {}_M \B(\pi)_M $ is mixing relative to $ A $.
    \end{enumerate}

\end{prop}


\begin{lem}[Fell's Absorption Principle for Groupoids]
  Let $ \pi $ be a representation of $ \rG $ on $ X\ast \H $ and $ \lambda_\rG $ the left regular representation of $ \rG $. Then for any orthonormal fundamental sequence of sections $ \Xi=\{\xi^n\}_{n=1}^\infty $ for the bundle $ X\ast \H $, we have that $ \pi\otimes \lambda_\rG $ is unitarily equivalent to $ {\rm id}_{\Xi}\otimes \lambda_\rG $
\end{lem}

\begin{proof}
  Let $ \Xi = \{\xi^{n}\}_{n=1}^{\infty} $ be an orthonormal fundamental sequence of sections for $ X\ast \H $. For $ g,h\in \rG $ and $n, m\in\N $, we compute
  \begin{align*}
    \inp{\pi(g) \xi_{\d(g)}^{n}\otimes \delta_{g}}{\pi(h) \xi_{\d(h)}^{m}\otimes \delta_{h}} &= \inp{\pi(g) \xi_{\d(g)}^{n}}{\pi(h) \xi_{\d(h)}^{m}}\cdot \inp{\delta_{g}}{\delta_{h}}\\
    &= \inp{\pi(g) \xi_{\d(g)}^{n}}{\pi(h) \xi_{\d(h)}^{m}}\cdot \delta_{g=h} \\
    &= \inp{\pi(g) \xi_{\d(g)}^{n}}{\pi(g) \xi_{\d(g)}^{m}}\cdot \delta_{g=h} \\ &=\delta_{g=h}\cdot \delta_{n=m} \\
    &= \inp{\xi_{\r(g)}^{n}\otimes \delta_{g}}{\xi_{\r(h)}^{m}\otimes \delta_{h}}
  \end{align*}

  So, for every $ x\in X $, setting $ U_{x}(\xi_{x}^{n}\otimes \delta_{g}) = \pi(g) \xi_{\d(g)}^{n} \otimes \delta_{g} $ defines a unitary on $ \H_{x}\otimes \ell^{2}(\rG^{x}) $. Now let $(g,h)\in \rG^{(2)}$, $n\in \N$, we see that 
\begin{align*}
    U_{\r(g)}\big({\rm id}_\Xi\otimes \lambda_\rG\big)(g)[\xi_{\r(h)}^n\otimes\delta_h]&=U_{\r(g)}(\xi_{\r(g)}^n\otimes\delta_{gh}) \\
    &=\pi(gh)\xi_{\d(h)}^n\otimes\delta_{gh} \\
    &=\big(\pi\otimes \lambda_\rG\big)(g)[\pi(h)\xi_{\d(h)}^n\otimes\delta_h] \\
    &=\big(\pi\otimes \lambda_\rG\big)(g)U_{\d(g)}[\xi_{\r(h)}^n\otimes\delta_h].
\end{align*} Implying $U_{\r(g)}\big({\rm id}_\Xi\otimes \lambda_\rG\big)(g)=\big({\rm id}_\Xi\otimes \lambda_\rG\big)(g)U_{\d(g)}$, for all $g\in\rG$. The measurability of $x\mapsto U_x$ is obvious. \end{proof}

\textcolor{violet}{TODO: check that weak containment and unitary equivalence of representations of $\rG$ pass to the corresponding bimodules we have defined.\\
Also Check that $\pi$ Holds in group and equivalence relation setting. Al}



\begin{thm}\label{prime}
    Let $\rG$ be a discrete measured groupoid with no amenable direct summand which admits an unbounded 1-cocycle into a mixing orthogonal representation weakly contained in the regular representation. Then $L(\rG) \not \cong N\otimes Q$ for any type II von Neumann algebras $N$ and $Q$. 
\end{thm}

\begin{proof}
   Let $\pi$ be such a representation, whence $\widehat{\pi}$ is mixing and weakly contained in $\lambda_\rG$. 
    \[
        L^2(X\ast \Omega) \ominus L^2(X) \cong \int_X^\oplus [L^2(\Omega_x) \ominus \C] \dd{\mu(x)} = \int_X^\oplus \K_x \dd{\mu(x)} 
    \]
    Tensoring over $A$ with $L^2(\rG)$, we obtain
    \begin{equation}\label{orthocomp}
        _M L^2(\widetilde{M}) \ominus L^2(M)_M \cong [L^2(X\ast \Omega) \ominus L^2(X)]\otimes_A L^2(\rG) \cong \B(\rho)
    \end{equation}

    By Lemmas \ref{fockspace} and \ref{reptobim} \textcolor{violet}{(reference containment part)}, we know that $\B(\rho) \cong \B(\widehat{\pi})$ and $\B(\widehat{\pi}) \subwk \B(\lambda)$ as $M$-$M$ bimodules. Moreover, Lemma \ref{reptobim} \textcolor{violet}{(reference mixing part)} combined with \eqref{orthocomp} imply that $_M L^2(\widetilde{M}) \ominus L^2(M)_M$ is mixing with respect to $A$. By assumption, $\B(\widehat{\pi})\subwk \B(\lambda)$. Observe that, as $M$-$M$ bimodules, 

    \[
        \B(\lambda) = \int_X^{\oplus} l^2(\rG^x) \dd{\mu(x)} \otimes_A L^2(\rG) \cong L^2(M) \otimes_A L^2(M)
    \]
    Since $A$ is amenable, \textcolor{violet}{$L^2(M)\otimes_A L^2(M)$ is weakly contained in the coarse.}\\


    As $ b $ is unbounded, there exists a $ \delta>0 $ such that for all $ R>0 $, there is some full group element $ \sigma\in [\rG] $ such that $ \mu(\{\norm{b(x \sigma)}\geq R\}) > \delta $. Without loss of generality assume $ \delta < 8 $.

    Now for $ x\in M $ we compute that,
    \begin{equation}\label{popatransez}
        \norm{\alpha_{t}(x) - \E_{M}(\alpha_{t}(x))}_{2} \leq \norm{\alpha_{t}(x) - x}_{2} + \norm{\E_{M}(x - \alpha_{t}(x))}_{2} = 2 \norm{\alpha_{t}(x) - x}_{2}.
    \end{equation}

    Suppose, for the sake of contradiction, that $ \alpha_{t}\to {\rm id }$ uniformly on $ (M)_{1} $. Choose $ t_{0} > 0 $ such that
    \[
        \sup_{x\in (M)_{1}} \norm{\alpha_{t_{0}}(x) - x}_{2} < 2+ \frac{1}{2}\sqrt{16-2\delta} =: \gamma
    \]
    i.e. so that $ \gamma^{2}-4\gamma > -\frac{\delta}{2} $. Then, for $ \sigma\in [\rG] $, we apply \eqref{popatransez} and compute
    \begin{align*}
        \norm{\E_{M}(\alpha_{t_{0}}(u_{\sigma}))}_{2} &\geq \norm{\alpha_{t_0}(u_{\sigma})}_{2} -\norm{\alpha_{t_0}(u_{\sigma}) - \E_{M}(\alpha_{t_0}(u_{\sigma}))}_{2}\\
        & \overset{\eqref{popatransez}}{\geq} 1 - 2\norm{\alpha_{t_0}(x) - x}_{2} > 1-2\gamma
    \end{align*}
    whence $  \norm{\E_{M}(\alpha_{t_{0}}(u_{\sigma}))}_{2}^{2} > 1-\frac{\delta}{2} $. Choose $ R>>0 $ such that $ e^{-2t_{0}^{2}R^2} < \frac{\delta}{2} $ and $ \sigma\in [\rG] $ with $ \mu(\{\norm{b(x \sigma)}\geq \sqrt{R}\}) < \delta $,

    \begin{align*}
        1-\frac{\delta}{2} \leq \norm{\E_{M}(\alpha_{t_{0}(u_{\sigma})})}_{2}^{2} &= \norm{f_{c_{t_0},\sigma}u_{\sigma}}_{2}^{2} = \tau ( f_{c_{t_{0}},\sigma} \cls{ f_{c_{t_{0}},\sigma}}) = \int_{X} e^{-2t_{0}^2 \norm{b(x \sigma)}^{2}} \dd{\mu(x)} \\
        &\leq \int_{\{\norm{b(x \sigma)}^{2}< R\}} \dd{\mu(x)} + \int_{\{\norm{b(x \sigma)}^{2} \geq R\}}e^{2t_{0}^{2}R^{2}}\dd{\mu(x)}\\
        &\leq \mu(\{\norm{b(x \sigma)}^{2}< R\}) + e^{-2t_{0}^{2}R^{2}} \mu(\{\norm{b(x \sigma)}^{2}\geq R\})  < 1- \frac{\delta}{2}
    \end{align*}
    which is absurd. Hence, we may apply Popa's spectral gap argument and conclude that $ M $ cannot be decomposed as a tensor product of two type II von Neumann algebras.


\end{proof}


\section{Fullness}
\subsection{Intertwining-by-bimodules}
We recall Sorin Popa's incredibly powerful \textit{intertwining-by-bimodules} technique \cite[Theorem 2.1]{popa:03}.
\begin{thm}[Popa's Intertwining by Bimodules]
    Let $ (M,\tau) $ be a tracial von Neumann algebra and $ P\sub pMp $, $ Q\sub qMq $ von Neumann subalgebras. Let $ \U\sub \U(P) $ be a subgroup which generates $ P $ as a von Neumann algebra. Then the following are equivalent:
    \begin{enumerate}
        \item There are projections $ p_{0}\in P $, $ q_{0}\in Q $, a $ * $-homomorphism $ \theta:p_{0}Pp_{0}\to q_{0}Qq_{0} $, and a nonzero partial isometry $ v\in q_{0}Mp_{0} $ such that $ \theta(x)v = vx $ for all $ x\in p_{0}Pp_{0} $.
        \item There does not exist a sequence $ (u_{n})_{n=1}^{\infty} $ in $ \U $ such that $ \norm{\E_{Q}(xu_{n}y)}_{2}\to0 $ for all $ x,y\in M $.
    \end{enumerate}
    If either of above equivalent conditions are satisfied, we write $ P\prec_{M}Q $.
\end{thm}

The following proposition essentially provides sufficient conditions on $ \rG $ for $ L(\rG) $ to belong to Drimbe's Class $ \mc{M} $ (see \cite[Definition 3.2]{drimbe:21}). 

\begin{prop}
    Let $\rG$ be a discrete measured groupoid which is strongly ergodic with no amenable direct summand. Assume further that $ \rG $ admits an unbounded 1-cocycle into a mixing orthogonal representation weakly contained in the regular representation. Let $ M = L(\rG) $, $ A = L^{\infty}(\rG^{(0)}) $, and assume $ M $ is a type $ \rm{II}_{1} $-factor.

    Suppose that $ N $ is a tracial von Neumann algebra and $ P\sub p(M\otimes N)p $ is a von Neumann subalgebra such that
    \begin{itemize}[noitemsep, topsep=1pt]
        \item $ P^{\prime}\cap p(M\otimes N)p $ is strongly nonamenable relative to $ 1\otimes N $, and
        \item $ P\prec_{M\otimes N} A\otimes N $.
    \end{itemize}
    Then $ P\prec_{M\otimes N} 1\otimes N$.
\end{prop}


The following lemma is well-known and due to Ioana, Popa and Vaes \cite[Lemma 2.3]{ipv:10}. We include its proof here for the sake of completeness.

\begin{lem}[Ioana, Popa and Vaes]\label{lem:gapweakcont}
   Let $ P\sub pMp $ be a von Neumann subalgebra, $ _M \H_{P} $ an $ M $-$ P $ bimodule, and $ \kappa>0 $. Suppose that there is a sequence $ (\xi_{n})_{n=1}^{\infty} $ in $ \H $ such that
   \begin{enumerate}
       \item $ \norm{[a,\xi_{n}]} \xrightarrow{n\to\infty}0 $ for all $ a\in P $,
       \item $ \norm{x \xi_{n}}\leq \kappa \norm{x}_{2} $ for all $ x\in M $,
       \item $ \limsup_{n\to\infty}\norm{p \xi_{n}}>0 $.
   \end{enumerate}
   Then there is some nonzero projection $ z\in \mathcal{Z}(P^{\prime}\cap pMp) $ such that $_M L^{2}(M) _{Pz} $ is weakly contained in $ _M \H_{Pz} $.

\end{lem}

\begin{proof}
    Without loss of generality, replace $ \xi_{n} $ with $ p \xi_{n} $ so that we may assume $ \xi_{n}=p \xi_{n} $.

    Consider the functional $ \phi_{n}:M\to \C $ given by $ \phi_{n}(x)=\inp{x \xi_{n}}{\xi_{n}} $. Then note that $ 0\leq \phi_{n}\leq \kappa^{2} \tau $, so by Radon-Nikodym there is some $ T_{n}\in pM_{+}p $ such that 
    \[
        \tau(xT_{n}) = \phi_{n}(x) = \inp{x \xi_{n}}{\xi_{n}} \quad \text{for all }x\in M.
    \]
    After passing to subsequences, we may assume there is some $ T\in pM_{+}p $ such that $ T_{n}\to T $ weakly and $ \tau(T) > 0 $. 

    Suppose $ x,y,a\in M $ and set $ m=xy^{*} $. Then we compute 
    \begin{align*}
        |\inp{[a,T_{n}]\widehat{x}}{\widehat{y}}| &= |\tau(y^{*}aT_{n}x)-\tau(y^{*}T_{n}ax)| \\
        &= |\tau(maT_{n}) - \tau(amT_{n})|\\
        &= |\inp{[m,a] \xi_{n}}{\xi_{n}}| \\
        &\leq |\inp{m[a,\xi_{n}]}{\xi_{n}}| + |\inp{m \xi_{n}a - am \xi_{n}}{\xi_{n}}|\\
        &\leq \norm{m}\norm{[a,\xi_{n}]}\norm{\xi_{n}} +|\inp{m \xi_{n}}{[\xi_{n},a^{*}]}|\\
        &\leq  \norm{m}\norm{[a,\xi_{n}]}\norm{\xi_{n}} + \norm{m}\norm{[\xi_{n},a^{*}]}\norm{\xi_{n}}
    \end{align*} 
    Letting $ J $ denote the modular conjugation operator, we observe that 
    \[
        \norm{[\xi_{n},a^{*}]} = \norm{J[\xi_{n},a^{*}]} = \norm{J(JaJ \xi_{n})-\xi_{n}a} =\norm{[a,\xi_{n}]},
    \]
    whence, we finally obtain 
    \begin{equation}
        |\inp{[a,T_{n}] \widehat{x}}{\widehat{y}}| \leq 2\norm{m}\norm{\xi_{n}}\norm{[a,\xi_{n}]}.
    \end{equation}
    By condition (3) and uniqueness of limits, it follows that $ T\in P^{\prime}\cap pMp $.

    Now choose $ \delta>0 $ small enough such that $ 1_{(\delta, \kappa] }(T^{1/2}) $ is a nonzero projection, and define $ f(t):=t^{-1}1_{(\delta, \kappa]}(t) $ for $ t\in \sigma(T^{1/2}) $. Then set $ S:= f(T^{1/2}) $ and $ q:=TS^{2} $. 

    %For $ x,a \in M $ and $ y\in Pq $, observe that
    %\begin{align*}
    %    \inp{x \widehat{a} y}{\widehat{a}} &= \tau(a^{*}xay) = \tau(a^{*}xaySTS) = \tau(Sa^{*}xayST)\\
    %    &= \lim_{n\to\infty} \tau(Sa^{*}xayST_{n}) = \lim_{n\to\infty} \inp{Sa^{*}xayS \xi_{n}}{\xi_{n}}\\
    %    &=\lim_{n\to\infty} \inp{xaS \xi_{n}y}{aS \xi_{n}}.
    %\end{align*}
    %To show weak containment, suppose $ F_{1}\sub M $, $ F_{2}\sub Pq $ are finite subsets, $ \eps>0 $, and $ a\in M $. Set $ \eta_{n}:= aS \xi_{n}\in \H $ for all $ n\in \N $. Since these subsets are finite, we may choose $ m\in \N $ such that $  $
    %\[
    %\left| \inp{x \widehat{a} y}{\widehat{a}} - \inp{x \eta_{m}y}{\eta_{m}}\right| < \eps\quad \text{for all }x\in F_{1},\,y\in F_{2}.
    %\]
    %By density of $ M $ inside $ L^{2}(M) $, this shows that we can in fact approximate any vector state in $ L^{2}(M) $ by vector states from $ \H $, whence we conclude that $\prescript{}{M}{L^{2}(M)}_{Pq}\prec\prescript{}{M}{\H}_{Pq}$.


    By \cite[Part III, Chapter 8.2]{dix:81}, there is some nonzero subprojection $ \widetilde{q} $ which is fundamental, i.e there exist nonzero, pairwise disjoint projections $ q_{1}, \ldots, q_{n}\in P^{\prime}\cap pMp $ and a central projection $ z\in \mathcal{Z}(P^{\prime}\cap pMp) $ such that
    \begin{itemize}
        \item $ \widetilde{q}=q_{1}\sim q_{2}\sim \cdots\sim q_{n} $
        \item $ \sum_{j=1}^{m} q_{j} = z $.
    \end{itemize}
    Then, as $ \E_{\mathcal{Z}(P^{\prime}\cap pMp)}(\cdot) $ is a center-valued trace, we have that
    \begin{align*}
        z = \E_{\mathcal{Z}(P^{\prime}\cap pMp)}(z)= \E_{\mathcal{Z}(P^{\prime}\cap pMp)}\lr{\sum_{i=1}^{m} q_{i} } = m\cdot\E_{\mathcal{Z}(P^{\prime}\cap pMp)}(\widetilde{q}) 
    \end{align*}
    Choose partial isometries $ v_{1},\ldots, v_{m}\in P^{\prime}\cap pMp $ such that $ v_{i}^{*}v_{i}=\widetilde{q} $ and $ v_{i}v_{i}^{*} = q_{i} $ for all $ 1\leq i \leq m $. As we are now working under $ \widetilde{q} $, we adujust the operator $ S $ via $ \widetilde{S}:= S \widetilde{q} $, so $ \widetilde{S}^{2}T = \widetilde{q} $.
    For $ x,a \in M $ and $ y\in Pz $, observe that
    \begin{align*}
        \inp{x \widehat{a} y}{\widehat{a}} &= \tau(a^{*}xay) = \tau(a^{*}xayz) =\sum_{i=1}^{m} \tau(a^{*}xayv_{i}v_{i}^{*})\\
        &=\sum_{i=1}^{m} \tau(a^{*}xav_{i}yv_{i}^{*})=\sum_{i=1}^{m} \tau(a^{*}xayv_{i}v_{i}^{*}v_{i}v_{i}^{*})=\sum_{i=1}^{m} \tau(a^{*}xayv_{i}\widetilde{S}T \widetilde{S}v_{i}^{*})\\
        &= \lim_{n\to\infty} \sum_{i=1}^{m}\tau(\widetilde{S}v_{i}^{*}a^{*}xayv_{i} \widetilde{S}T_{n})\\
        &=\lim_{n\to\infty}\sum_{i=1}^{m} \inp{\widetilde{S}v_{i}^{*}a^{*}xay v_{i}\widetilde{S} \xi_{n}}{\xi_{n}} =\lim_{n\to\infty}\sum_{i=1}^{m} \inp{xa v_{i}\widetilde{S} \xi_{n} y}{av_{i}\widetilde{S}\xi_{n}}  \\
    \end{align*}
    To show weak containment, suppose $ F_{1}\sub M $, $ F_{2}\sub Pz $ are finite subsets, $ \eps>0 $, and $ a\in M $. Set $ \eta_{n}^{i}:= av_{i}\widetilde{S} \xi_{n}\in \H $ for all $ n\in \N $ and $ 1\leq i \leq m $. Since these subsets are finite, we may choose $ k\in \N $ such that $  $
    \[
        \left| \inp{x \widehat{a} y}{\widehat{a}} - \sum_{i=1}^{m}\inp{x \eta_{k}^{i}y}{\eta_{k}}\right| < \eps\quad \text{for all }x\in F_{1},\,y\in F_{2}.
    \]
    By density of $ M $ inside $ L^{2}(M) $, this shows that we can in fact approximate any vector state in $ L^{2}(M) $ by vector states from $ \H $, whence we conclude that $\prescript{}{M}{L^{2}(M)}_{Pq}\prec\prescript{}{M}{\H}_{Pq}$.
\end{proof}
The preceeding lemma presents a certain relative amenability condition for direct summands. In the opposite case of strong nonamenability, one can use the contrapositive of the previous lemma to find unitaries in $ P $ for which commutation with these unitaries detects closeness to $ M $. Running this idea through a deformation gives the following example of Popa's spectral gap principle from \cite{drimbe:21} which we reproduce for the reader's convenience.
%\textcolor{violet}{Note that strong ergodicity here I think ensures that L(G) has no amenable direct summand but idk}
\begin{lem}
    Let $\rG$ be a discrete measured groupoid which is strongly ergodic with no amenable direct summand and which admits an unbounded 1-cocycle into a mixing orthogonal representation weakly contained in the left-regular representation. Letting $ M = L(\rG) $, suppose that $ P\sub pMp $ is a von Neumann subalgebra that is strongly nonamenable. Then $ \alpha_{t}\to id $ uniformly on $ (P^{\prime}\cap pMp)_{1} $. Furthermore, if $ (P^{\prime}\cap pMp)\prec_{M} A $, then $ (P^{\prime}\cap pMp) \prec_{M} \C1 $. 
\end{lem}

\begin{proof}
    Following the proof of Theorem \ref{prime}, we see that the amenability of $ A $ implies that 
    \[
        \prescript{}{M}{L^{2}(\widetilde{M})\ominus L^{2}(M)}_{M} \prec \prescript{}{M}{L^{2}(M)\otimes L^{2}(M)}
    \]
    Since $ P $ has is strongly nonamenable, for any nonzero central projection $ z\in \mathcal{Z}(P^{\prime}\cap pMp) $ we have that 
   $ _M L^{2}(M)_{Pz} $ is not weakly contained $ _M L^{2}(M) \otimes L^{2}(M)_{Pz} $, and thus is not weakly contained in $ _M L^{2}(\widetilde{M}) \ominus L^{2}(M)_{Pz} $.\\
   % \[
   %     \prescript{}{M}{L^{2}(M)}_{Pz} \not\prec  \prescript{}{M}{L^{2}(M)\otimes L^{2}(M)}_{M}
   % \]
   % and thus
   % \[
   %     \prescript{}{M}{L^{2}(M)}_{Pz} \not\prec \prescript{}{M}{L^{2}(\widetilde{M})\ominus L^{2}(M)}_{M}
   % \]

   Let $ \eps>0 $. By Lemma \ref{lem:gapweakcont}, there exist elements $ a_{1}, \ldots, a_{n}\in P $ and $ \delta>0 $ such that, for $ x\in (p \widetilde{M} p)_{1} $,
   \[
   \left[\norm{[x,a_{i}]}_{2} <\delta \text{ for all }1\leq i \leq n\right] \implies \norm{x-\E_{M}(x)}_{2}<\eps.
   \]
   Since $ \alpha_{t} $ is $ s $-malleable, there is a $ t_{0}>0 $ such that
   \[
       \norm{\alpha_{t}(a_{i})-a_{i}}_{2} < \delta/2\quad\text{for all }|t|<t_{0}, \text{ and }1\leq i \leq n
   \]
   For $ t\in (-t_{0},t_{0}) $ and $ x\in (P^{\prime}\cap pMp)_{1} $, we use the semigroup structure of the deformation to compute
   \begin{align*}
       \norm{a_{i}\alpha_{t}(x) - \alpha_{t}(x) a_{i}}_{2} &= \norm{\alpha_{-t}( a_{i} \alpha_{t}(x) - \alpha_{t}(x)a_{i})}_{2}= \norm{\alpha_{-t}(a_{i}) x - x \alpha_{-t}(a_{i})}_{2} \\
       &\leq\norm{\alpha_{-t}(a_{i}) x - x \alpha_{-t}(a_{i})}_{2} \\
       &\leq\norm{\alpha_{-t}(a_{i}) x -a_{i}x}_{2}+ \norm{xa_{i} -x \alpha_{-t}(a_{i})}_{2} \\
       &\leq 2 \norm{a_{i}-\alpha_{-t}(a_{i})}_{2} < \delta. \\
   \end{align*}
   By Popa's Transversality Inequality (TODO CITE CITE), we obtain that
   \[
       \norm{\alpha_{t}(x)-x}_{2}\leq \sqrt{2} \eps.
   \]
   as desired.
    
\end{proof}

Now we have shown that the groupoid von Neumann algebras in question belong to Drimbe's class $ \mathcal{M} $ from \cite[3.2]{drimbe:21}, whence appealing to \cite[Lemma 3.7]{drimbe:21} from the same paper, we obtain the following theorem.

\begin{thm}
     Let $\rG$ be a discrete measured groupoid which is strongly ergodic with no amenable direct summand and which admits an unbounded 1-cocycle into a mixing orthogonal representation weakly contained in the left-regular representation. Then $ M:=L(\rG) $ does not have property Gamma.
\end{thm}




\printbibliography

\end{document}


