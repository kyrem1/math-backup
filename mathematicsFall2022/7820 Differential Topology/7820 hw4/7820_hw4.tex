\documentclass[12pt,letterpaper]{article}

%--------Packages--------
\usepackage{amsmath, amsthm, amssymb}
\usepackage{xspace}
\usepackage{graphicx}
\usepackage{hhline}
\usepackage{amssymb}
\usepackage{array}
\usepackage{braket}
\usepackage{multicol}
\usepackage{mathtools}
\usepackage{enumerate}
\usepackage{delarray}
\usepackage{mathtools}
\usepackage{fullpage}
\usepackage{faktor} % For quotients
\usepackage{mathrsfs}

\usepackage[italicdiff]{physics} % For differentials
\usepackage{bbm} % For indicator

% \usepackage{quiver}
\usepackage[linguistics]{forest}




%--------Page Setup--------

\pagestyle{empty}%

\setlength{\hoffset}{-1.54cm}
\setlength{\voffset}{-1.54cm}

\setlength{\topmargin}{0pt}
\setlength{\headsep}{0pt}
\setlength{\headheight}{0pt}

\setlength{\oddsidemargin}{0pt}

\setlength{\textwidth}{195mm}
\setlength{\textheight}{250mm}


%--------Macros--------

\newcommand{\sub}{\subseteq}
\newcommand{\lcm}{\text{lcm}}
\newcommand{\mc}[1]{\mathcal{#1}}
\newcommand{\mf}[1]{\mathfrak{#1}}
\newcommand{\ms}[1]{\mathscr{#1}}
\newcommand{\sO}{\mathcal{O}}
\newcommand{\cyclic}[1]{\langle#1\rangle}
\newcommand{\units}[1]{#1 ^{\times}}
\newcommand{\la}{\langle}
\newcommand{\ra}{\rangle}
\newcommand{\lr}[1]{\left(#1\right)}
\newcommand{\lrvert}[1]{\left\lvert#1\right\rvert}
\DeclarePairedDelimiterX{\inp}[2]{\langle}{\rangle}{#1, #2}

%----Switch phi and varphi
% \let\temp\phi
% \let\phi\varphi
% \let\varphi\temp

\newcommand{\C}{\mathbb{C}}
\newcommand{\F}{\mathbb{F}}
\newcommand{\E}{\mathbb{E}}
\newcommand{\N}{\mathbb{N}\xspace}
\newcommand{\I}{\mathbb{I}\xspace}
\newcommand{\R}{\mathbb{R}\xspace}
\newcommand{\Z}{\mathbb{Z}\xspace}
\newcommand{\Q}{\mathbb{Q}\xspace}
\newcommand{\G}{\mathbb{G}\xspace}

\renewcommand{\H}{\mathcal{H}}
\newcommand{\M}{\mathcal{M}}

\DeclareMathOperator{\Spec}{Spec}
\DeclareMathOperator{\res}{res}
% \DeclareMathOperator{\Tr}{Tr}
\DeclareMathOperator{\ord}{ord}
\DeclareMathOperator{\Sym}{Sym}
% \DeclareMathOperator{\dv}{div}
\DeclareMathOperator{\alb}{alb}
\DeclareMathOperator{\img}{Im}
\DeclareMathOperator{\et}{et}
\DeclareMathOperator{\ck}{coker}
\DeclareMathOperator{\Reg}{Reg}
\DeclareMathOperator{\Cor}{Cor}
\DeclareMathOperator{\Ac}{at}
\DeclareMathOperator{\supp}{supp}
\DeclareMathOperator{\Hom}{Hom}
\DeclareMathOperator{\Pic}{Pic}
\DeclareMathOperator{\Gal}{Gal}
\DeclareMathOperator{\fc}{frac}
\DeclareMathOperator{\Ann}{Ann}
\DeclareMathOperator{\Mod}{Mod}
\DeclareMathOperator{\Cone}{Cone}
\DeclareMathOperator{\FI}{FI}
\DeclareMathOperator{\End}{End}
\DeclareMathOperator{\Alb}{Alb}
\DeclareMathOperator{\Ext}{Ext}
\DeclareMathOperator{\ab}{ab}
\DeclareMathOperator{\Jac}{Jac}
\DeclareMathOperator{\coker}{coker}
\DeclareMathOperator{\fr}{frac}
\DeclareMathOperator{\Int}{Int}
\let\Span\relax
\DeclareMathOperator{\Span}{Span}
\DeclareMathOperator{\Ran}{Ran}



%----Analysis
\newcommand{\summ}{\sum\limits}
% \newcommand{\norm}[1]{\left\lVert#1\right\rVert}
\newcommand{\thicc}{\bigg}
\newcommand{\eps}{\varepsilon}
\newcommand*\cls[1]{\overline{#1}}
\newcommand{\ind}{\mathbbm{1}}
\DeclareMathOperator{\sgn}{sgn}


%--------Theorem environments--------
\newtheorem{definition}{Definition}[]
\newtheorem{lemma}{Lemma}[]
\newtheorem{corollary}{Corollary}[]
\newtheorem{theorem}{Theorem}[]
\theoremstyle{remark}
\newtheorem*{claim}{Claim}


\newenvironment{solution}
{\begin{proof}[Solution]}
{\end{proof}}


\makeatletter
\newcommand{\thickhline}{%
    \noalign {\ifnum 0=`}\fi \hrule height 1pt
    \futurelet \reserved@a \@xhline
}
\newcolumntype{"}{@{\hskip\tabcolsep\vrule width 1pt\hskip\tabcolsep}}
\makeatother

% --------Problem environment--------
\setlength\parindent{0pt}
\setcounter{secnumdepth}{0}
\newcounter{partCounter}
\newcounter{homeworkProblemCounter}
\setcounter{homeworkProblemCounter}{1}


\newenvironment{homeworkProblem}[1][-1]{
    \ifnum#1>0
        \setcounter{homeworkProblemCounter}{#1}
    \fi
    \section{Problem \arabic{homeworkProblemCounter}}
    \setcounter{partCounter}{1}
    \stepcounter{homeworkProblemCounter}
}


%--------Metadata--------
\title{MATH 7820 Homework 4}
\author{James Harbour}

\begin{document}
\maketitle

\begin{homeworkProblem}
    Show that the degree (mod $ 2 $) of a composition of two smooth maps $ f $ and $ g $ equals 
    \[
        \deg_{2}(f)\cdot \deg_{2}(g)\,(\text{mod}\,2).
    \]
    
    \begin{proof}
        Let $ f:x\to Y $ and $ g:Y\to Z $. We must assume $ X,Y $ are compact, $ Y,Z $ are connected, and $ \dim(X) = \dim(Y) = \dim(Z) = n$ for all values above to be defined. Let $ z\in Z $ be a regular value of $ g\circ f $ and $ \{x_{1},\ldots,x_{k}\}=(g\circ f)^{-1}(z) = f^{-1}(g^{-1}(z)) $. 
        Fix $ 1\leq i \leq k $. As $ \dim T_{x_{i}}X = \dim T_{f(x_{i})}Y = n$ and $ d_{x_{i}}(g\circ f) = (d_{f(x_{i})}g)(d_{x_{i}}f) $ is surjective, it follows that both $d_{f(x_{i})}g  $ and $ d_{x_{i}}f $ are surjective. Thus, each $ y_{i}= f(x_{i}) $ is a regular point of $ g $ as well as a regular value of $ f $ with corresponding regular point $ x_{i} $.\\

        By well definedness of degree mod 2, for each $ y_{i} $, $ |f^{-1}(y_{i})| (\text{mod } 2) =\deg_{2}(f)$. Hence, we compute
        \[
            \deg_{2}(g\circ f) \equiv |f^{-1}(g^{-1}(z))| = \sum_{i=1}^{k}|f^{-1}(y_{i})| \equiv \sum_{i=1}^{k}\deg_{2}(f)\,\,(\text{mod }2) = \deg_{2}(f)\cdot \deg_{2}(g).
        \]
    \end{proof}
\end{homeworkProblem}


\begin{homeworkProblem}
    Give an example of manifolds $ M,N $, and of a smooth function $ F:M\to N $ with a regular value which is a limit point of critical values.

    \begin{proof}[Example]
        Consider $ f:\R\to \R $ given by $ f(x)=e^{-x}\sin(x) $. Then $ f '(x) = e^{-x}(\cos(x)-\sin(x)) $. 
        \[
             0 = f '(x) = e^{-x}(\cos(x)-\sin(x)) \iff \cos(x)=\sin(x),
        \]
        so the critical points of $ f $ are $ x_{k}=\frac{\pi}{4}+\pi k $ for $ k\in \Z $. These critical points correspond to the critical values
        \[
            f(x_{k}) = e^{\frac{-\pi}{4}-\pi k}\lr{\cos(\frac{-\pi}{4}-\pi k) + \sin(\frac{-\pi}{4}-\pi k)} \begin{dcases*}
            \sqrt{2}e^{-\frac{\pi}{4}-\pi k} & when $k$ is even\\
            -\sqrt{2}e^{-\frac{\pi}{4}-\pi k}& when $k$ is odd
            \end{dcases*}.
        \]
        Note $ \pm\sqrt{2}e^{-\frac{\pi}{4}-\pi k}\xrightarrow{k\to\infty}0 $, thus $ 0 $ is a limit point of the critical values $ \{f(x_{k}\}_{k\in\Z} $, but $ 0 $ is not a critical value of $ f $ as $ x_{k}\neq 0 $ for all $ k\in\Z $, so $ 0 $ is a regular value of $ f $ which is a limit point of critical values of $ f $.
    \end{proof}
\end{homeworkProblem}


\begin{homeworkProblem}
    If two smooth maps $ f,g $ from a manifold $ M $ to the unit sphere $ S^{k} $ satisfy 
    \[
        \norm{f(x)-g(x)}<2  
    \]
    for all $ x\in M $, prove that $ f $ is smoothly homotopic $ g $.

    \begin{proof}
        Fix $ t\in [0,1] $ and $ x\in M $. If $ t=0,1 $, then $ \norm{(1-t)f(x)+tg(x)} = 1 \neq 0 $. If $ t\in (0,1) $, then we compute 
        \[
            \norm{(1-t)f(x)+tg(x)}= \norm{t(f(x)-g(x))-f(x)} \geq t \norm{f(x)-g(x)} -\norm{f(x)} \geq 2t-1 >0.
        \]
        Thus, we may define the function $ H:M\times [0,1] \to S^{k}$ by
        \[
            H(x,t) = \frac{(1-t)f(x) + tg(x)}{\norm{(1-t)f(x) + tg(x)}}.
        \]
        This map is clearly continuous as, for each $ t\in [0,1], $ the map $ x\mapsto H(x,t) $ is smooth by smoothness of $ f $ and $ g $ and the fact that addition and scalar multiplication are smooth. Moreover, as $ f,g $ both map into the unit sphere already, $ H(x,0) = f(x) $ and $ H(x,1) = g(x) $.
    \end{proof}
\end{homeworkProblem}


\begin{homeworkProblem}
    Consider a simple closed curve $ C $ in $ \R^{2}\setminus \{(0,0)\} $, and let $ f:C\to \R $ be the distance to the origin: $ f(x,y) = \sqrt{x^{2}+y^{2}} $. Prove that the critical points of $ f $ are precisely the points where the curve $ C $ is tangent to some circle centered at the origin. \\

    \textit{Hint}: It may be helpfuls to consider the square of the function, and you may assume the curve $ C $ is parameterized, $ x=x(t),y=y(t),a\leq t\leq b $.

    \begin{proof}
        Let $ \gamma:[a,b]\to C $ parameterize $ C $ bijectively and let $ \gamma(t) = (x(t),y(t)) $. For any open subset $ U\sub C $, $ (U,\gamma^{-1}) $ is a chart. Suppose $ p\in C $ is a critical point. Write $ p = \gamma(s) $. Then 
        \[
            0 = 2 (f\circ\gamma)(s)\cdot (f\circ\gamma) '(s) = \dv{t}\vert_{t=s}f\circ\gamma = 2x(s)x '(s) + 2y(s) y '(s),
        \]
        so $ \inp{\gamma(s)}{\gamma '(s)} = 0 $, whence $ \gamma $ is tangent to the circle of radius $ \norm{\gamma(s)} $ at the point $ \gamma(s) $. \\

        On the other hand, suppose that $ p\in C $ is tangent to some circle centered at the origin. Then it must be true that the circle has radius $ \norm{p} $. Write $ p = \gamma(s) $. Then, as before, 
        \[
            (f\circ\gamma) '(s) = \frac{\inp{\gamma(s)}{\gamma '(s)}}{(f\circ\gamma)(s)} = 0,
        \]
        so $ p $ is a critical point of $ f $.
    \end{proof}
\end{homeworkProblem}


\end{document}
