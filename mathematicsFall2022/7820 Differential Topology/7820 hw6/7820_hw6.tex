\documentclass[12pt,letterpaper]{article}

%--------Packages--------
\usepackage{amsmath, amsthm, amssymb}
\usepackage{xspace}
\usepackage{graphicx}
\usepackage{hhline}
\usepackage{amssymb}
\usepackage{array}
\usepackage{braket}
\usepackage{multicol}
\usepackage{mathtools}
\usepackage{enumerate}
\usepackage{delarray}
\usepackage{mathtools}
\usepackage{fullpage}
\usepackage{faktor} % For quotients
\usepackage{mathrsfs}

\usepackage[italicdiff]{physics} % For differentials
\usepackage{bbm} % For indicator

\usepackage{quiver}
\usepackage[linguistics]{forest}




%--------Page Setup--------

\pagestyle{empty}%

\setlength{\hoffset}{-1.54cm}
\setlength{\voffset}{-1.54cm}

\setlength{\topmargin}{0pt}
\setlength{\headsep}{0pt}
\setlength{\headheight}{0pt}

\setlength{\oddsidemargin}{0pt}

\setlength{\textwidth}{195mm}
\setlength{\textheight}{250mm}


%--------Macros--------

\newcommand{\sub}{\subseteq}
\newcommand{\lcm}{\text{lcm}}
\newcommand{\mc}[1]{\mathcal{#1}}
\newcommand{\mf}[1]{\mathfrak{#1}}
\newcommand{\ms}[1]{\mathscr{#1}}
\newcommand{\sO}{\mathcal{O}}
\newcommand{\cyclic}[1]{\langle#1\rangle}
\newcommand{\units}[1]{#1 ^{\times}}
\newcommand{\la}{\langle}
\newcommand{\ra}{\rangle}
\newcommand{\lr}[1]{\left(#1\right)}
\newcommand{\lrvert}[1]{\left\lvert#1\right\rvert}

\DeclarePairedDelimiterX{\inp}[2]{\langle}{\rangle}{#1, #2}

%----Switch phi and varphi
% \let\temp\phi
% \let\phi\varphi
% \let\varphi\temp

\newcommand{\C}{\mathbb{C}}
\newcommand{\F}{\mathbb{F}}
\newcommand{\E}{\mathbb{E}}
\newcommand{\N}{\mathbb{N}\xspace}
\newcommand{\I}{\mathbb{I}\xspace}
\newcommand{\R}{\mathbb{R}\xspace}
\newcommand{\Z}{\mathbb{Z}\xspace}
\newcommand{\Q}{\mathbb{Q}\xspace}
\newcommand{\G}{\mathbb{G}\xspace}

\renewcommand{\H}{\mathcal{H}}
\newcommand{\M}{\mathcal{M}}

\DeclareMathOperator{\Spec}{Spec}
\DeclareMathOperator{\res}{res}
% \DeclareMathOperator{\Tr}{Tr}
\DeclareMathOperator{\ord}{ord}
\DeclareMathOperator{\Sym}{Sym}
% \DeclareMathOperator{\dv}{div}
\DeclareMathOperator{\alb}{alb}
\DeclareMathOperator{\img}{Im}
\DeclareMathOperator{\et}{et}
\DeclareMathOperator{\ck}{coker}
\DeclareMathOperator{\Reg}{Reg}
\DeclareMathOperator{\Cor}{Cor}
\DeclareMathOperator{\Ac}{at}
\DeclareMathOperator{\supp}{supp}
\DeclareMathOperator{\Hom}{Hom}
\DeclareMathOperator{\Pic}{Pic}
\DeclareMathOperator{\Gal}{Gal}
\DeclareMathOperator{\fc}{frac}
\DeclareMathOperator{\Ann}{Ann}
\DeclareMathOperator{\Mod}{Mod}
\DeclareMathOperator{\Cone}{Cone}
\DeclareMathOperator{\FI}{FI}
\DeclareMathOperator{\End}{End}
\DeclareMathOperator{\Alb}{Alb}
\DeclareMathOperator{\Ext}{Ext}
\DeclareMathOperator{\ab}{ab}
\DeclareMathOperator{\Jac}{Jac}
\DeclareMathOperator{\coker}{coker}
\DeclareMathOperator{\fr}{frac}
\DeclareMathOperator{\Int}{Int}
\let\Span\relax
\DeclareMathOperator{\Span}{Span}
\DeclareMathOperator{\Ran}{Ran}
\DeclareMathOperator{\ran}{ran}
\DeclareMathOperator{\ext}{ext}
\DeclareMathOperator{\GL}{GL}

%----Analysis
\newcommand{\summ}{\sum\limits}
% \newcommand{\norm}[1]{\left\lVert#1\right\rVert}
\newcommand{\thicc}{\bigg}
\newcommand{\eps}{\varepsilon}
\newcommand*\cls[1]{\overline{#1}}
\newcommand{\ind}{\mathbbm{1}}
\DeclareMathOperator{\sgn}{sgn}


%--------Theorem environments--------
\newtheorem{definition}{Definition}[]
\newtheorem{lemma}{Lemma}[]
\newtheorem{corollary}{Corollary}[]
\newtheorem{theorem}{Theorem}[]
\theoremstyle{remark}
\newtheorem*{claim}{Claim}


\newenvironment{solution}
{\begin{proof}[Solution]}
{\end{proof}}


\makeatletter
\newcommand{\thickhline}{%
    \noalign {\ifnum 0=`}\fi \hrule height 1pt
    \futurelet \reserved@a \@xhline
}
\newcolumntype{"}{@{\hskip\tabcolsep\vrule width 1pt\hskip\tabcolsep}}
\makeatother

% --------Problem environment--------
\setlength\parindent{0pt}
\setcounter{secnumdepth}{0}
\newcounter{partCounter}
\newcounter{homeworkProblemCounter}
\setcounter{homeworkProblemCounter}{1}


\newenvironment{homeworkProblem}[1][-1]{
    \ifnum#1>0
        \setcounter{homeworkProblemCounter}{#1}
    \fi
    \section{Problem \arabic{homeworkProblemCounter}}
    \setcounter{partCounter}{1}
    \stepcounter{homeworkProblemCounter}
}


%--------Metadata--------
\title{MATH 7820 Homework 6} 
\author{James Harbour}

\begin{document}
\maketitle
    

\begin{homeworkProblem}
    Denote the standard coordinates on $ \R^{2} $ by $ x,y $, and let 
    \[
        X = -y \pdv{x} +x \pdv{y} \quad\text{ and }\quad Y = x \pdv{x} + y \pdv{y} 
    \]
    be vector fields on $ \R^{2} $. Find a $ 1 $-form $ \omega $ on $ \R^{2}\setminus\{(0,0)\} $ such that $ \omega(X) = 1 $ and $ \omega(Y) = 0 $.

    \begin{proof}[Solution]
        Let $ dx $ and $ dy $ be dual to $ \pdv{x}  $ and $ \pdv{y} $ respectively. Let $ f,g:\R^{2}\setminus \{(0,0)\}\to \R $ be given by 
        \[
            f(x,y) = \frac{-y}{x^{2}+y^{2}} \quad\text{ and }\quad g(x,y)=\frac{x}{x^{2}+y^{2}}.
        \]
        Now define a $ 1 $-form $ \omega $ by $ \omega = f(x,y)\dd{x}+g(x,y)\dd{y} $. Then we compute that, for $ p=(x,y)\in \R^{2}\setminus \{(0,0)\} $,
        \[
            \omega_{p}(X_{p}) = (f(x,y)\dd{x}+ g(x,y)\dd{y})\lr{-y \pdv{x} +x \pdv{y}} = \frac{x^{2}}{x^{2}+y^{2}}+\frac{y_{2}}{x^{2}+y^{2}} = 1,
        \]
        \[
            \omega_{p}(Y_{p}) = (f(x,y)\dd{x}+ g(x,y)\dd{y})\lr{x \pdv{x} +y \pdv{y}} = \frac{-xy}{x^{2}+y^{2}} + \frac{xy}{x^{2}+y^{2}} = 0.
        \] 
    \end{proof}
\end{homeworkProblem}



\begin{homeworkProblem}
    Suppose $ (U,x^{1},\ldots,x^{n}) $ and $ (V, y^{1},\ldots,y^{n}) $ are two charts on $ M $ with $ U\cap V\neq \emptyset $. Then a $ C^{\infty} $ 1-form $ \omega $ on $ U\cap V $ has two different local expressions:
    \[
        \omega = \sum_{j}a_{j}\dd{x}^{j} = \sum_{i}b_{i} \dd{y}^{i}.
    \]
    Find a formula for $ a_{j} $ in terms of $ b_{i} $.

    \begin{proof}
        Fix $ p\in U\cap V $. For $ 1\leq k\leq n $, there exist $ c_{1}^{k},\ldots, c_{n}^{k}\in\R $ such that 
        \[
            \pdv{x^{k}}\eval_{p} = \sum_{l=1}^{n} c_{l}^{k} \pdv{y^{l}}\eval_{p}.
        \]
        Now applying both sides of this expression to the coordinate function $ y^{m} $, we see for $ 1\leq m\leq n $ that
        \[
            c_{m}^{k} = \sum_{l=1}^{n} c_{l}^{k} \pdv{y^{m}}{y^{l}}\eval_{p} = \pdv{y^{m}}{x^{k}} \eval_{p}.
        \]
        Now, we compute
        \begin{align*}
            a_{k}(p) &= \omega_{p}\lr{\pdv{x^{k}} \vert_{p}} = \sum_{i}b_{i}(p) (\dd{y^{i}})_{p} \lr{\pdv{x^{k}}\eval_{p}} \\
            &= \sum_{i}b_{i}(p) (\dd{y^{i}})_{p} \lr{\sum_{l}\pdv{y^{l}}{x^{k}}\eval_{p} \pdv{y^{l}}\eval_{p} } = \sum_{i}b_{i}(p) \pdv{y^{i}}{x^{k}} \eval_{p}.
        \end{align*}
    \end{proof}
\end{homeworkProblem}



\begin{homeworkProblem}
    Prove that a vector bundle whose fiber is an $ n $-dimensional vector space is trivial (i.e. is isomorphic to a product bundle) if and only if it admits $ n $ sections $ s_{1},\ldots, s_{n} $ such that $ s_{1}(x),\ldots, s_{n}(x)$ are linearly independent for each point $ x $ in the base.
    
    \begin{proof}\ \\
        \underline{$ \implies $}: Suppose we are given homeomorphisms $ f,g $ as below such that $ f $ is a linear isomorphism on fibers and the following diagram commutes:
        \[\begin{tikzcd}
        	E && {B\times\mathbb{R}^n} \\
        	\\
        	B && B
        	\arrow["\pi", from=1-1, to=3-1]
        	\arrow["g", from=3-1, to=3-3]
        	\arrow["f", from=1-1, to=1-3]
        	\arrow["p"{description}, from=1-3, to=3-3]
        \end{tikzcd}\]
        Let $ e_{1},\ldots,e_{n}\in \R^{n} $ be the standard basis for $ \R^{n} $, and for $ 1\leq i\leq n $ define $ s_{i}:B\to E $ by $ s_{i}(x) = f^{-1}(g(x),e_{i}) $. Now suppose that $ x\in B $ and $ \lambda_{i}\in \R $ are such that in $ \pi^{-1}(\{x\}) $ we have $ 0 = \sum_{i} \lambda_{i}s_{i}(x)$. Then we use linearity of $ f^{-1} $ on fibers to compute that
        \[
            0 = \sum_{i} \lambda_{i} s_{i}(x) = \sum_{i} \lambda_{i} f^{-1}(g(x),e_{i}) = f^{-1}\lr{g(x), \sum_{i} \lambda_{i}e_{i}},
        \]
        whence by linearity $ \sum_{i} \lambda_{i}e_{i} = 0 $ in $ p^{-1}(g(x)) $, so by independence $ \lambda_{i}=0 $ for all $ i $. Hence $ s_{1}(x),\ldots, s_{n}(x) $ are linearly independent. \\

        \underline{$ \impliedby $}: For $ x\in B $, let $ \beta(x) $ denote the basis $ \beta(x) = \{s_{i}(x): 1\leq i\leq n\} $. Define a map $ \phi:E\to B\times \R^{n} $ as follows: for $ p\in E $ there exists a unique $ x\in B $ such that $ p\in \pi^{-1}(\{x\}) $, so set $ \phi(p) = (x, [p]_{\beta(x)}) $. Note that, on each fiber, the map $ \phi $ is given by $ p\mapsto [p]_{\beta(x)} $ and is thus a linear isomorphism by assumption. Moreover, the map $ \phi $ satisfies the following commutative diagram:
        \[\begin{tikzcd}
        	E && {B\times\mathbb{R}^n} \\
        	\\
        	B && B
        	\arrow["\pi", from=1-1, to=3-1]
        	\arrow["{id_B}", Rightarrow, no head, from=3-1, to=3-3]
        	\arrow["\phi", from=1-1, to=1-3]
        	\arrow["p", from=1-3, to=3-3]
        \end{tikzcd}\]
        Now define a map $ \psi:B\times\R^{n}\to E $ by $ \psi(x, v_{1}, \ldots, v_{n}) = \sum_{i}v_{i}s_{i}(x) $. Note that this map is again clearly a linear isomorphism on fibers by assumption. Moreover, it is clear that $ \psi $ is the inverse of $ \phi $, thus giving an isomorphism of vector bundles.
        
    \end{proof}
\end{homeworkProblem}



\begin{homeworkProblem}
    Suppose $ E_{1}\to B $, $ E_{2}\to B $ are two vector bundles over the same base. It may be assumed without loss of generality that they are both trivialized over the same collection of charts $ \{U^{i}\} $ covering $ B $. Denote their transition maps $ \phi_{ij}:U_{i}\cap U_{j} \to \GL_{m}(\R) $, $ \psi_{ij}:U_{i}\cap U_{j}\to \GL_{n}(\R) $. The tensor product of these two bundles is a vector bundle over the same base $ B $, defined by taking the tensor product of the transition maps at each point, with values in $ \GL_{mn}(\R) $.\\

    Carry out this construction in the case of two copies of the non-trivial line bundle over the circle discussed in class, and identify their tensor product.

    \begin{proof}
        Under the construction in class, we have two sets $ U,V $ covering $ S^{1} $ with intersection two intervals. Moreover, the transition map $ \phi:U\cap V \to \GL_{1}(\R) $ is given sending one interval to $ +1 $ and the other interval to $ -1 $. This line bundle then becomes the Mobius band. Taking the tensor product of this map with itself and then applying the standard isomorphism $ \R\otimes\R\to \R $ given by multiplication, it follows that the new transition map is $ \phi \otimes\phi = \phi \cdot \phi = 1 $, whence the new line bundle simply becomes the trivial bundle.
    \end{proof}
\end{homeworkProblem}




\end{document}
