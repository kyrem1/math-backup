\documentclass[12pt,letterpaper]{article}

%--------Packages--------
\usepackage{amsmath, amsthm, amssymb}
\usepackage{xspace}
\usepackage{graphicx}
\usepackage{hhline}
\usepackage{amssymb}
\usepackage{array}
\usepackage{braket}
\usepackage{multicol}
\usepackage{mathtools}
\usepackage{enumerate}
\usepackage{delarray}
\usepackage{mathtools}
\usepackage{fullpage}
\usepackage{faktor} % For quotients
\usepackage{mathrsfs}

\usepackage[italicdiff]{physics} % For differentials
\usepackage{bbm} % For indicator

\usepackage{quiver}
\usepackage[linguistics]{forest}




%--------Page Setup--------

\pagestyle{empty}%

\setlength{\hoffset}{-1.54cm}
\setlength{\voffset}{-1.54cm}

\setlength{\topmargin}{0pt}
\setlength{\headsep}{0pt}
\setlength{\headheight}{0pt}

\setlength{\oddsidemargin}{0pt}

\setlength{\textwidth}{195mm}
\setlength{\textheight}{250mm}


%--------Macros--------

\newcommand{\sub}{\subseteq}
\newcommand{\lcm}{\text{lcm}}
\newcommand{\mc}[1]{\mathcal{#1}}
\newcommand{\mf}[1]{\mathfrak{#1}}
\newcommand{\ms}[1]{\mathscr{#1}}
\newcommand{\sO}{\mathcal{O}}
\newcommand{\cyclic}[1]{\langle#1\rangle}
\newcommand{\units}[1]{#1 ^{\times}}
\newcommand{\la}{\langle}
\newcommand{\ra}{\rangle}
\newcommand{\lr}[1]{\left(#1\right)}
\newcommand{\lrvert}[1]{\left\lvert#1\right\rvert}
\DeclarePairedDelimiterX{\inp}[2]{\langle}{\rangle}{#1, #2}

%----Switch phi and varphi
% \let\temp\phi
% \let\phi\varphi
% \let\varphi\temp

\newcommand{\C}{\mathbb{C}}
\newcommand{\F}{\mathbb{F}}
\newcommand{\E}{\mathbb{E}}
\newcommand{\N}{\mathbb{N}\xspace}
\newcommand{\I}{\mathbb{I}\xspace}
\newcommand{\R}{\mathbb{R}\xspace}
\newcommand{\Z}{\mathbb{Z}\xspace}
\newcommand{\Q}{\mathbb{Q}\xspace}
\newcommand{\G}{\mathbb{G}\xspace}

\renewcommand{\H}{\mathcal{H}}
\newcommand{\M}{\mathcal{M}}

\DeclareMathOperator{\Spec}{Spec}
\DeclareMathOperator{\res}{res}
% \DeclareMathOperator{\Tr}{Tr}
\DeclareMathOperator{\ord}{ord}
\DeclareMathOperator{\Sym}{Sym}
% \DeclareMathOperator{\dv}{div}
\DeclareMathOperator{\alb}{alb}
\DeclareMathOperator{\img}{Im}
\DeclareMathOperator{\et}{et}
\DeclareMathOperator{\ck}{coker}
\DeclareMathOperator{\Reg}{Reg}
\DeclareMathOperator{\Cor}{Cor}
\DeclareMathOperator{\Ac}{at}
\DeclareMathOperator{\supp}{supp}
\DeclareMathOperator{\Hom}{Hom}
\DeclareMathOperator{\Pic}{Pic}
\DeclareMathOperator{\Gal}{Gal}
\DeclareMathOperator{\fc}{frac}
\DeclareMathOperator{\Ann}{Ann}
\DeclareMathOperator{\Mod}{Mod}
\DeclareMathOperator{\Cone}{Cone}
\DeclareMathOperator{\FI}{FI}
\DeclareMathOperator{\End}{End}
\DeclareMathOperator{\Alb}{Alb}
\DeclareMathOperator{\Ext}{Ext}
\DeclareMathOperator{\ab}{ab}
\DeclareMathOperator{\Jac}{Jac}
\DeclareMathOperator{\coker}{coker}
\DeclareMathOperator{\fr}{frac}
\DeclareMathOperator{\Int}{Int}
\let\Span\relax
\DeclareMathOperator{\Span}{Span}
\DeclareMathOperator{\Ran}{Ran}



%----Analysis
\newcommand{\summ}{\sum\limits}
% \newcommand{\norm}[1]{\left\lVert#1\right\rVert}
\newcommand{\thicc}{\bigg}
\newcommand{\eps}{\varepsilon}
\newcommand*\cls[1]{\overline{#1}}
\newcommand{\ind}{\mathbbm{1}}
\DeclareMathOperator{\sgn}{sgn}


%--------Theorem environments--------
\newtheorem{definition}{Definition}[]
\newtheorem{lemma}{Lemma}[]
\newtheorem{corollary}{Corollary}[]
\newtheorem{theorem}{Theorem}[]
\theoremstyle{remark}
\newtheorem*{claim}{Claim}


\newenvironment{solution}
{\begin{proof}[Solution]}
{\end{proof}}


\makeatletter
\newcommand{\thickhline}{%
    \noalign {\ifnum 0=`}\fi \hrule height 1pt
    \futurelet \reserved@a \@xhline
}
\newcolumntype{"}{@{\hskip\tabcolsep\vrule width 1pt\hskip\tabcolsep}}
\makeatother

% --------Problem environment--------
\setlength\parindent{0pt}
\setcounter{secnumdepth}{0}
\newcounter{partCounter}
\newcounter{homeworkProblemCounter}
\setcounter{homeworkProblemCounter}{1}


\newenvironment{homeworkProblem}[1][-1]{
    \ifnum#1>0
        \setcounter{homeworkProblemCounter}{#1}
    \fi
    \section{Problem \arabic{homeworkProblemCounter}}
    \setcounter{partCounter}{1}
    \stepcounter{homeworkProblemCounter}
}


%--------Metadata--------
\title{MATH 7820 Homework 5}
\author{James Harbour}

\begin{document}
\maketitle

\begin{itemize}
    \item A manifold $ M $ is \emph{oriented} if for all $ x\in M $ there is a choice of orientation on $ T_{x}M $ such that there is some chart $ (U,\phi) $ around $ x $ such that $ d_{\phi(x)} \phi^{-1} $ carries the standard orientation on $ \R^{n} $ to the chosen orientation on $ T_{x} M $.
\end{itemize}

\begin{homeworkProblem}
    In class we discussed the induced orientation of the tangent space $ T_{x}(\partial M) $ of the boundary $ \partial M $ of an oriented manifold $ M $, at each $ x\in \partial M $. Prove that this is in fact an orientation of $ \partial M $ i.e. that it depends smoothly on the point $ x\in \partial M $.

    \begin{proof}
        Choose an orientaton for $ M $. We can induce an orientation on $ \partial M $ by for each $ p\in \partial M $ choosing a positively oriented basis for $ T_{p}M $ $ (v_{1},\ldots, v_{n}) $ such that $ \{v_{2},\ldots, v_{n}\}\sub T_{p}(\partial M) $ and $ v_{1} $ points inwards. There exists an open $ U\sub M $ about $ p $ and a diffeomorphism $ \phi:U\to \phi(U)\sub H^{n} $ such that $ d_{\phi(x)} \phi^{-1}(e_{i}) = v_{i} $ where $ \{e_{1},\ldots, e_{n}\} $ is the standard oreintation on $ \R^{n} $.

        %TODO adjust to definition Krushkal gave

        Since $ \dim T_{p}(\partial M) = n-1 $ and $ v_{i}\in T_{p}(\partial M) $ for $ i\geq 2 $, it follows that $ (v_{2}, \ldots, v_{n}) $ is an ordered basis for $ T_{p}(\partial M) $. Now let $(U, \phi) $ be a chart around $ p $ satisfying the property above. Let $ \widetilde{\phi}:= \phi\vert_{U\cap \partial M} $. Then $ (U\cap \partial M, \widetilde{\phi}) $ is a chart on $ \partial M $ around $ p $. 
        Observe that, for $ i\geq 2 $
        \[
            (d_{\widetilde{\phi(p)}}\widetilde{\phi}^{-1}) (e_{i}) = d_{\widetilde{\phi(p)}} \phi^{-1}\vert_{\phi(U\cap\partial M)}(e_{i}) = d_{\phi(p)} \phi^{-1} \circ d_{\phi(p)}\iota_{\phi(U\cap\partial M)\hookrightarrow \phi(U)} (e_{i}) = d_{\phi(p)} \phi^{-1} \mqty(I_{n}\vert 0) e_{i} = d_{\phi(p)} \phi^{-1}(e_{i}) = v_{i},
        \]
        thus giving an orientation on $ \partial M $.

    \end{proof}
\end{homeworkProblem}


\begin{homeworkProblem}
    Show that the tangent bundle $ TM $ of any (orientable or not) manifold $ M $ is orientable. 

    \begin{proof}
        Let $ (U_{\alpha}, \phi_{\alpha})_{\alpha} $ be an atlas on $ R^{n} $ adapted to $ M $. Then $ (V_{\alpha},\Phi_{\alpha})_{\alpha} $ given by $ V_{\alpha} = U\times \R^{n} $ and $ \Phi(p,v) = (\phi(p), d_{p}(v)) $. Suppose that $ U_{\alpha}\cap U_{\beta} \neq \emptyset $. Let $ T = \Phi_{\beta}\circ \Phi_{\alpha}^{-1} $ and $ t = \phi_{\beta}\circ \phi_{\alpha}^{-1} $. Fix $ (x,y) = \Phi_{\alpha}(p,v)\in \Phi_{\alpha}(V_{\alpha}) $. Then we compute
        \[
            T(x,y) = \Phi_{\beta}(p,v) = (\phi_{\beta}(p), d_{p} \phi_{\beta}(v)) = (t(x), d_{p} \phi_{\beta} d_{x} \phi_{\alpha}^{-1}(y)) = (t(x), d_{x}t(y)).
        \]
        Note that with respect to the standard basis, $ d_{x}t(y) $ does not depend on $ y $ and is a linear map, so by our previous homework $ J(d_{x}t(y)) = d_{x}t(y) = J(t)(x) $. Thus $ J(T)(x,y) = \mqty(J(t)(x) & \cdot \\ \cdot & J(t)(x)) $, which has positive determinant as it is the square of the determinant of $ J(t)(x) $. Thus this atlas is a positive atlas for $ TM $, which implies that $ TM $ is orientable.
    \end{proof}

    % Use Cech cocyle identity.
\end{homeworkProblem}


\begin{homeworkProblem}
    Given disjoint manifolds $ M^{m}, N^{n} $ in $ \R^{k+1} $, the linking map $ \lambda: M\times N \to S^{k} $ is defined by
    \[
        \lambda(x,y) = \frac{x-y}{|x-y|}.
    \]
    If $ M,N $ are compact, oriented, and without boundary, and $ m+n = k $, then the integer valued degree of $ \lambda $ is called the \textit{linking number} $ l(M,N) $. Prove that 
    \[
        l(N,M) = (-1)^{(m+1)(n+1)}l(M,N).
    \]
    If $ M $ bounds an oriented compact manifold $ W $ disjoint from $ N $, prove that $ l(M,N)=0 $.


    \begin{proof}
        Note that the orientations on $ M $ and $ N $ induce orientations on $ M\times N $. Let $ \tau:N\times M \to M\times N $ be the transposition map, $ A:S^{k}\to S^{k} $ the antipodal map, and $ \lambda ' $ the linking map on $ N\times M $. Then the following diagram commutes:
\[\begin{tikzcd}
	{N\times M} && {S^k} \\
	\\
	{M\times N} && {S^k}
	\arrow["A"', from=3-3, to=1-3]
	\arrow["\lambda"', from=3-1, to=3-3]
	\arrow["\tau"', from=1-1, to=3-1]
	\arrow["{\lambda'}", from=1-1, to=1-3]
\end{tikzcd}\]
        Note that, by previous homework, $ \deg(\lambda ') = \deg(A)\deg(\tau)\deg(\lambda) $. We compute the degree of $ \tau $. As $ \tau $ is a bijection and every point is regular, for fixed $ p = (y,x)\in N\times M $ we have that $ \deg(\tau) = \sgn(d_{p}\tau) $. Given orientations for $T_{p} N $ and $T_{p} M $, we patch them together to an orientation for $T_{p}( N\times M) $. Consider the corresponding basis for $ T_{p}(M\times N) $. Under these bases, $ d_{p}\tau $ is represented by the matrix $ \mqty(0_{m\times n} & I_{m}\\I_{n} & 0_{n\times m}) $. This matrix has determinant $ (-1)^{m\cdot n} $, so $ \deg(\tau) = (-1)^{m\cdot n} $
        Thus, 
        \[
            l(N,M) = \deg(\lambda ') = \deg(A)\deg(\tau)l(M,N) = (-1)^{k+1}(-1)^{m\cdot n}l(M,N) = (-1)^{(m+1)(n+1)}l(M,N).
        \]
    \end{proof}
\end{homeworkProblem}


\begin{homeworkProblem}
    Given an integer $ n $, construct a smooth map $ f:S^{1}\times S^{1} \to  S^{1}\times S^{1}$ such that $ \deg(f) = n $. (integer-valued degree).

    \begin{proof}[Solution]
        Define $ f(\theta,\phi) = (n\cdot \theta \text{ mod } 2\pi, \phi) $. Note that $ (0,0)) $ is a regular value for $ f $, so 
        \[
            \deg(f) = \sum_{k=0}^{n} d_{(\frac{2\pi k}{n},0)}f = n
        \]
        as each restriction to $ S^{1} $ is orientation preserving.
        % NOTE degree is sum over points in inverse image of regular value of sign of determinant of differential at that point

    \end{proof}
\end{homeworkProblem}








\end{document}
