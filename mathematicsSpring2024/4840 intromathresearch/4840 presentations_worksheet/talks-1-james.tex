%! TEX root = ./main.tex
\documentclass[12pt]{article}

%--------Packages-------------
\usepackage{kyrem1sty}
%----------------------------


%--------Bibliography---------
%\usepackage[backend=biber,style=alphabetic,doi=false,isbn=false,url=false,eprint=false]{biblatex}
%\addbibresource{INSERT .BIB PATH}
%----------------------------


%--------Hyper Setup-------
\hypersetup{%
  colorlinks=true,%
  linkcolor=blue,%
  citecolor=blue,%
  filecolor=blue,%
  menucolor=blue,%
  urlcolor=blue,%
  pdfnewwindow=true,%
  pdfstartview=FitBH
}   
%----------------------------


%--------Subfiles Setup-------
%\usepackage{subfiles}
%----------------------------


%--------Page Setup-----------
%\usepackage{geometry}\geometry{margin=1in}
\pagestyle{empty}%

\setlength{\hoffset}{-1.54cm}
\setlength{\voffset}{-1.54cm}

\setlength{\topmargin}{0pt}
\setlength{\headsep}{0pt}
\setlength{\headheight}{0pt}

\setlength{\oddsidemargin}{0pt}

\setlength{\textwidth}{195mm}
\setlength{\textheight}{250mm}
%----------------------------


%--------Metadata------------
\title{Presentations Worksheet}
\author{James Harbour}
%----------------------------


%--------Content-------------
\begin{document}

\begin{center}
  \Large\textbf{Evan}: \textit{Matroids and Greedy Algorithms on Them}
\end{center}

\section*{Peer Feedback}

\begin{itemize}
  \item While I appreciated the style of sticking to one board and not meandering too much, I think it would be more instructive if you wrote bigger and used more boards to make up for it.
  \item The basic examples of graphs and matrices were great. I would love to have seen more examples which are more involved.
  \item I didn't really buy that the greedy algorithm you described ``converges'', whatever that means. Maybe a couple comments about that would have been nice. 
\end{itemize}

\section*{Three Things}

\begin{itemize}
  \item \textit{Motivation}: Matroids generalize the notion of ``independence'' to other settings.
  \item \textit{Definition}: A matroid is a pair $ (E, I) $ where $ E $ is a set called the ``ground set'' and $ I\sub \mathcal{P}(E) $ satisfies 
    \begin{itemize}
      \item $ \emptyset \in I $,
      \item if $ A\in I $ and $ B\sub A $, then $ B\in I $,
      \item if $ A,B\in I $ and $ |A| > |B| $, then there is some $ a\in A $ such that $ B\cup \{a\} \in I $.
    \end{itemize}
  \item \textit{Point of Talk}: Given a weighted matroid (matroid with weights), there is a greedy algorithm which gives an independent set of maximum weight and size. Generalizes Kruskal's algorithm for finding spanning forests.
\end{itemize}


\newpage


\begin{center}
  \Large\textbf{Ethan}: \textit{Quasisymmetric Functions}
\end{center}

\section*{Peer Feedback}
\begin{itemize}
  \item While the starting motivation was good, I think you introudced too much notation too fast in the beginning. In my opinion, a math talk should have a continuous, slow increase in difficulty throughout the talk, not a sharp jump.
  \item I find math talks the most engaging when the speaker barely uses their notes and gives their exposition in an impromptu manner. Try to wean off using notes so much.
  \item Try to stick to the time limit set by the organizers for math talks.
\end{itemize}

\section*{Three Things}
\begin{itemize}
  \item Quasisymmetric functions are an attempt to build a larger ring of generating functions than the symmetric fyunctions by using compositions instead of partitions as our fundamental combinatorial object.
  \item Analgous to how semistandard Young tableaux are used in the theory of symmetric functions, we have a notion of tableaux for compositions called \textit{semistandard reverse composition tableaux}.
  \item In this way, quasisymmetric Schur functions can be built, and these in fact can be used to write ordinary Schur functions as a sum of quasisymmetric Schur functions.
\end{itemize}


\newpage

\begin{center}
  \Large\textbf{Samir}: \textit{Symmetry of Grothendieck Polynomials}
\end{center}

\section*{Peer Feedback}
\begin{itemize}
  \item It is hard to give any critical feedback to someone who has already nearly mastered the art of giving math talks. Your boardwork and pacing were excellent
  \item My one critique is to maybe think a bit more deeply about the example you utilized so that you can respond quicker to questions regarding it.
  \item I would have loved a more in depth discussion of the history of Grothendieck polynomials and how they arose.
\end{itemize}

\section*{Three Things}

\begin{itemize}
  \item Analagous to ordinary tableaux, we have a collection $ SVT(\lambda) $ of what are called \textit{set-valued tableaux} corresponding to a partition $ \lambda $.
  \item \textit{Definition}: Given $ \lambda\vdash d $,
    \[
      G_{\lambda}= \sum_{T\in SVT(\lambda)} \beta^{|T|-| \lambda|} x^{\omega(T)}.
    \]
  \item $ G_{\lambda} = S_{\lambda} + (\text{higher degree terms in $ \beta $}) $. Symmetry is shown via a generalization of the Bender-Knuth involution.
\end{itemize}


\newpage



\end{document}
