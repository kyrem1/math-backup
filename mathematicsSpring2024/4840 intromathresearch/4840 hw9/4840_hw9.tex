%! TEX root = ./main.tex
\documentclass[12pt]{article}

%--------Packages-------------
\usepackage{kyrem1sty}
%----------------------------

%--------Bibliography---------
\usepackage[backend=biber,style=alphabetic,doi=false,isbn=false,url=false,eprint=false]{biblatex}
\addbibresource{/home/kyrem1/Mathematics/bibs/comborefs.bib}
%\addbibresource{comborefs.bib}
%----------------------------


%--------Hyper Setup-------
\hypersetup{%
  colorlinks=true,%
  linkcolor=blue,%
  citecolor=blue,%
  filecolor=blue,%
  menucolor=blue,%
  urlcolor=blue,%
  pdfnewwindow=true,%
  pdfstartview=FitBH
}   
%----------------------------


%--------Other Setup---------
\usepackage{ytableau} % For young diagrams
\usepackage{todonotes}
\newcommand{\Jenn}[1]{\todo[size=\tiny]{#1
      \\ \hfill --- Jenn}}
\usepackage{caption}
% Adjust caption spacing and font
\captionsetup{
  font=small, % Adjust the font size
  labelfont=bf,
  format=plain, % Use plain format to avoid any unwanted effects
  justification=raggedright, % Ensure the caption is justified, which can help with spacing
  singlelinecheck=false, % Applies justification setting even when the caption is a single line
}
%----------------------------


%--------Subfiles Setup-------
%\usepackage{subfiles}
%----------------------------


%--------Page Setup-----------
\usepackage{geometry}\geometry{margin=1in}
\pagestyle{empty}%

% \setlength{\hoffset}{-1.54cm}
% \setlength{\voffset}{-1.54cm}

% \setlength{\topmargin}{0pt}
% \setlength{\headsep}{0pt}
% \setlength{\headheight}{0pt}

% \setlength{\oddsidemargin}{0pt}

% \setlength{\textwidth}{195mm}
% \setlength{\textheight}{250mm}
%----------------------------
\setlength{\marginparwidth}{2cm}
\usepackage{todonotes}


%--------Metadata------------
\title{Assignment 9}
\author{James Harbour}
%----------------------------


%--------Content-------------
\begin{document}

\maketitle

\section{Three things responses}
\subsection{Comments about the responses}

Two points that often came up in the responses two my talk are
\begin{enumerate}
  \item My quote about viewing groups as their actions: ``Groups, as men, shall be known by their actions.'' This is an incredibly poignant quote and I am very glad people internalized this.
  \item The definition of irreducibility: A representation $ (\pi,V) $ is \textit{irreducible} if its only proper subrepresentation is $ 0 $.
\end{enumerate}

Something that people brought up in the three things exercise responses that I would not have prioritized as a main point of my talk is that I ended up using $ \C^{n} $ in my example instead of $ \R^{n} $ or $ k^{n} $. If I recall correctly, I mumbled a bit about this making things more clear but in hindsight it definitely obfuscates the point and my example is totally fine in characteristic zero (probably arbitrary characteristic but I don't like thinking about modular representations). The reason I did this was probablty because I am infinitely more comfortable with the algebraically closed characteristic zero case, but I should not have stressed this in the talk as it was confusing.

\subsection{Questions}

\begin{itemize}
  \item \textit{Relating to my talk}: To what extent can the representation theory of $ S_{n} $ be encoded in the ring of symmetric functions? In my talk, I proposed the existence of a dictionary between irreducible representations of $ S_{n} $ and Schur functions, but can the ring structure of the representation ring of $ S_{n} $ be encoded in some type of combinatorial ``product'' structure (maybe not the usual product) inside the ring of symmetric functions? 
  \item \textit{Relating to Anthony's talk}: Can the Littlewood-Richard coefficients be described purely in terms of representation theory?
\end{itemize}

Answer to both questions: yes the algebra structure, namely addition and multiplication, in symmetric functions can be recovered from purely the representation theory of symmetric groups.

The first step is to work over only with representations over $ \C $, since then isomorphism classes of representations are determined entirely by their characters. For a finite group $ G $, we may consider the representation ring of $ G $ over $ \C $, $ R_\C(G) $. If $  V_{1},\ldots,V_{r}  $ are the irreducible complex representations of $ G $, then the representation ring of $ G $
\[
  R_{\C}(G) = \bigoplus_{i=1}^{r}\Z V_{i} = \bigoplus_{i=1}^{r}\Z\chi_{V_{i}}
\]
i.e. it is a free $ \Z $ module of rank $ r $ generated by the (isomorphism classes of the) irreducible representations (or their characters since we are over $ \C $). 

By the Specht module construction, we know that the irreducible representations of $ S_{n} $ are the Specht modules, so $ R_{\C}(S_{n}) \cong \bigoplus_{\lambda\vdash n}\Z S_{\lambda} $. The product structure in $ \R_{\C}(S_{n}) $ is determined by its generators, so consider two partitions $ \lambda,\mu\vdash n $. Since this ring is free, there exist $ g_{\lambda,\mu}^{\nu}\in \Z $ such that
\[
  S_{\lambda}\otimes S_{\mu} = \sum_{\nu\vdash n} g_{\lambda,\mu}^{\nu} S_{\nu}.
\]
These coefficients $ g_{\lambda,\mu}^{\nu} $ are called Kronecker coefficients and they are incredibly difficult to understand, hence ordinary multiplication in $ \R_{\C}(S_{n}) $ is difficult to understand. At the level of symmetric functions, this does induce a new product called the \textit{Kronecker product}
\[
  s_{\lambda}\star s_{\mu} = \sum_{\nu} g_{\lambda,\mu}^{\nu} s_{\nu}.
\]

One difficulty with this approach is that we are only looking at one graded piece of the ring of symmetric functions and trying to stay within this piece---a philosophy counter to that of the notion of grading. Hence, we will incorporate all of $ S_{n} $-representations as $ n $ ranges into one ring. Consider the graded abelian group
\[
  R = \bigoplus_{n\geq0} R_{\C}(S_{n}) =\bigoplus_{n\geq0} \bigoplus_{\lambda\in \Par(n)} \Z S_{\lambda} = \bigoplus_{\lambda\in \Par} \Z S_{\lambda}
\]
with grading $ R_{n} = R_{\C}(S_{n}) $. We define a graded ring structure on $ R $ as follows. Let $ n,m\geq 0 $ and $\lambda\vdash n $, $ \mu\vdash m $. The product of $ S_{\lambda} $  and $ S_{\mu} $ needs to be a representation of $ S_{n+m} $. Since $ S_{\lambda} $ and $ S_{\mu} $ are a priori representations of different groups, we only have access to an external tensor product $ S_{\lambda}\boxtimes S_{\mu} $. The problem now is that this is a representation of $ S_{n}\times S_{m} $, not $ S_{n+m} $.

To fix this, note that we have a canonical inclusion $ S_{n}\times S_{m}\hookrightarrow S_{n+m} $. All we have to do now is consider the induced representation under this inclusion to obtain a representation of $ S_{n+m} $. Hence, we define the product in $ R $ by 
\[
  S_{\lambda}\bigcdot S_{\mu} := \Ind_{S_{n}\times S_{m}}^{S_{n+m}} (S_{\lambda}\otimes S_{\mu}) \in R_{\C}(S_{n+m})
\]
Depending on whether I have time and space, I will show that this in fact induces the ordinary product on the ring of symmetric functions. Hence, expressing in terms of generators gives the LR coefficients.

\newpage


\section{Classmate Feedback}

Some suggestions for improvement which came up multiple times:
\begin{itemize}
  \item I used some acronyms that I am used to using without explicitly defining them (e.g. f.d.v.s) and this led to a good bit of confusion.
  \item With characters, I should have included an example as a coupole people mentioned they got lost in this part.
  \item The most mentioned critique was my handwaving of the definition of an irreducible representation until the very end when asked. I should have written this down explicitly way earlier in the talk.
  \item I definitely should have cut down on the scope of the talk to go more slowly.
  \item My handwriting was commented on multiple times. I definitely need to work on this, maybe by using bigger chalk.
\end{itemize}
Some points of success which where noted multiple times:

\begin{itemize}
  \item People seemed to like my conversational tone
  \item Multiple people mentioned my organization of the talk and liked the structure.
  \item People seemed to like my mix of intuition and rigor. To be fair I really do wish I could have proven at least something substantial but this was pretty much impossible with the time given.
\end{itemize}

\end{document}
