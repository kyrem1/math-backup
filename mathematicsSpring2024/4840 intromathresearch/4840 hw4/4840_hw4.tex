%! TEX root = ./main.tex
\documentclass[12pt]{article}

%--------Packages-------------
\usepackage{kyrem1sty}
%----------------------------


%--------Bibliography---------
\usepackage[backend=biber,style=alphabetic,doi=false,isbn=false,url=false,eprint=false]{biblatex}
\addbibresource{/home/kyrem1/Mathematics/bibs/comborefs.bib}
%----------------------------


%--------Hyper Setup-------
\hypersetup{%
  colorlinks=true,%
  linkcolor=blue,%
  citecolor=blue,%
  filecolor=blue,%
  menucolor=blue,%
  urlcolor=blue,%
  pdfnewwindow=true,%
  pdfstartview=FitBH
}   
%----------------------------


%--------Other Setup---------
\usepackage{ytableau} % For young diagrams
\usepackage{caption}
% Adjust caption spacing and font
\captionsetup{
  font=small, % Adjust the font size
  labelfont=bf,
  format=plain, % Use plain format to avoid any unwanted effects
  justification=raggedright, % Ensure the caption is justified, which can help with spacing
  singlelinecheck=false, % Applies justification setting even when the caption is a single line
}
%----------------------------


%--------Subfiles Setup-------
%\usepackage{subfiles}
%----------------------------


%--------Page Setup-----------
%\usepackage{geometry}\geometry{margin=1in}
\pagestyle{empty}%

\setlength{\hoffset}{-1.54cm}
\setlength{\voffset}{-1.54cm}

\setlength{\topmargin}{0pt}
\setlength{\headsep}{0pt}
\setlength{\headheight}{0pt}

\setlength{\oddsidemargin}{0pt}

\setlength{\textwidth}{195mm}
\setlength{\textheight}{250mm}
%----------------------------


%--------Metadata------------
\title{Intro Math Research Hw4}
\author{James Harbour}
%----------------------------


%--------Content-------------
\begin{document}

\maketitle


\section{Abstracts}

\subsection*{Talk Abstract}
``Groups, as men, shall be know by their actions'' -Guillermo Moreno. In this talk, we introduce the field of representation theory and its connection to combinatorics. Through examples, we hint towards a deep connection between the combinatorics of partitions and the actions of symmetric groups on vector spaces. The only background required will be linear algebra and some knowledge of group theory.


\subsection*{Paper Abstract}

In this paper we exposit one of the fundamental results linking representation theory and algebraic combinatorics called Schur-Weyl duality. It provides a dictionary between the representation theory of finite symmetric groups and the representation theory of the general linear group of a finite dimensional complex vector space. Through this dictionary, we obtain representation theoretic constructions of many aspects of symmetric function theory, including Schur functions and Littlewood-Richardson coefficients.

\newpage
\section{Outlines}

\subsection{Talk Outline}
\begin{itemize}
  \item Run through the standard example $ \pi:  D_{2n} \to O(3) $
  \item Introduce definition of representations of finite groups.
  \item Get relevant defs to say $ V\otimes V \cong Sym^{2} \oplus \Lambda^{2}V$.
  \item State $ V\otimes V\otimes V \cong Sym^{3}V \oplus \Lambda^{3}V \oplus \text{ something else }$.
  \item Hint how that something else is related to the partition $ (2,1) $ of $ 3 $.
\end{itemize}


\subsection{Paper Outline}


\begin{itemize}
  \item Set up relevant preliminary representation theoretic definitions (group algebra stuff etc.)
  \item Talk about the generic representation theory of $ S_{n} $ 
  \item Define Young symmetrizers and Specht Modules 
  \item Talk about the commuting left and right actions of $ GL(V) $ and $ S_{n} $ respectively.
  \item State and cite the double centralizer theorem to prove Schur-Weyl duality. 
  \item Obtain relations to Schur functions, Littlewood-Richardson coefficients, plethyism, etc.
\end{itemize}


\newpage

\section{Symmetric Polynomials/Functions Exposition}

%\textbf{Assignment}: Revise your write-up on symmetric polynomials, focusing on the extra tip
%,in § 2.1 (``indispensable, interesting, illustrative.'') If you did not write about the basis of homogeneous symmetric
%functions last time, this should be included in your two pages. Add a third
%page dedicated to the combinatorial definition of Schur functions and their
%symmetry.

\subsection*{Preliminary Considerations} 
Throughout this article, fix a (unital) commutative ring $ R $ and a field $ k $. For simplicity, we work over vector spaces instead of general modules.

\begin{notation*}
  Let $ X $ be a set such that $ X = \{x_{i}\}_{i\in I} $ for some indexing set $ I $. By $ k[X] $ and $ k\llbracket X \rrbracket $, we denote the rings of (commutative) polynomials and power series (resp.) in indeterminates $ \{x_{i}\} $. We utilize multi-index notation throughout. Hence for $ \alpha_{\bigcdot}:I\to \N\cup\{0\} $ finitely supported, we write $ x_{\alpha} = \Pi_{i\in I}x_{i}^{\alpha_{i}} $ (where $ x_{i}^0 := 1 $ formally).
\end{notation*}

\subsection{Algebraic Background}

Often in algebra, elements of a given object may be decomposed into a sum of simpler elements which are, in a sense, ``homogenous.'' For example, any polynomial in $ n $-variable may be decomposed into a sum of simpler polynomials each of which are futher sums of monomials of the same total degree.  In this way, a polynomial is split into a sum of homogenous parts. This behavior is codified in the notion of \textit{grading}.

\begin{definition}\label{def:gralg}
A \textit{graded $ k $-algebra} is a $ k $-algebra $ A $ together with a direct sum decomposition
\[
  A = \bigoplus_{i=0}^{\infty} A_{i}
\]
with $ A_{0},A_{1},\ldots $ vector spaces such that $ A_{i}\cdot A_{j}\sub A_{i+j} $ for all $ i,j\in \N\cup\{0\} $. For fixed $ i $, elements of $ A_{i} $ are called \textit{homogenous}. The choice of such a direct sum decomposition is a \textit{grading} for $ A $.
\end{definition}

%TODO insert exposition

\begin{keyexample*}
  As before, for $ X = \{x_{i}\}_{i\in I} $, we may give the ring $ k[X] $ a canonical grading by declaring $ A_{0}:=k $ and
  \[
    A_{n} := \Span_{k}\{x_{\alpha}: \alpha \text{ multi-index such that } \sum_{i\in I} \alpha_{i} = n\}.
  \]
\end{keyexample*} 
The reader is cautioned that not every $ k $-algebra has a nontrivial grading. In fact, it can be shown that the ring of formal power series $ k\llbracket x \rrbracket $ does not have a nontrivial grading.


\subsection{Symmetric Polynomials}

\begin{definition}\label{def:sympoly}
  The permutation group $ S_{n} $ acts naturally on the polynomial ring $ k[x_{1},\ldots,x_{n}] $ by defining $ \sigma \cdot x_{i_{1}}^{\alpha_1}\cdots x_{i_l}^{\alpha_{l}} := x_{\sigma(i_{1})}^{\alpha_{1}}\cdots x_{\sigma(i_{l})}^{\alpha_{l}}$ and extending by linearity. The ring of \textit{symmetric polynomials} in $ n $ indeterminates is the fixed points of this action, namely $ k[x_{1},\ldots, x_{n}]^{S_{n}} $.
\end{definition}

\subsection{Partitions and Compositions}
\begin{definition}\label{def:comp}\ 
  \begin{itemize}
    \item A \textit{partition} of $ n\in \N $ is a finite sequence $ \alpha=(\alpha_{1},\ldots, \alpha_{l}) $ of weakly decreasing positive integers which sum to $ n $. We denote the set of partitions of $ n $ by $ \Par(n) $. We denote the statement $ [\lambda\in \Par(n)] $ by $ \lambda\vdash n $. Also, we write $ \Par:=\bigcup_{n\geq0}\Par(n) $.
    \item A \textit{weak composition} of $ n\in \N $ is a (finitely supported) sequence $\alpha=(\alpha_{i})_{i=1}^{\infty}\in (\N\cup\{0\})^{\N}$ such that $ \sum_{i} \alpha_{i} = n $. The length of a weak composition $ \alpha $ is given by
  \[
    l(\alpha) :=\max\{i\in \N: \alpha_{i}\neq 0\}.
  \]
  \end{itemize}

  
\end{definition}

\begin{example}
  For $ n=5 $, the sequences $ \alpha=(1, 0, 2, 2, 0, 0, \ldots) $ and $ \beta=(2,0,1,2,0,0,\ldots) $ are distinct  weak compositions but neither are valid partitions of $ 5 $ due to the presence of a $ 0 $ between positive entries. On the other hand, $ \lambda = (2,2,1) $ is a partition of $ 5 $.
\end{example}

\subsection{Symmetric Functions}

\begin{definition}[pg. 308 in \cite{stanley2}]\label{def:symfunc}
    The ring $ \Lambda_{k} $ of symmetric functions over a field $ k $ is the subring of all $ f\in k\llbracket x_{1}, x_{2}, \ldots\rrbracket $ such that
    \[
      f(x_{\sigma(1)}, x_{\sigma(2)},\ldots) = f(x_{1},x_{2},\ldots) \text{ for all } \sigma\in \Sym(\N).
    \]
\end{definition}
\begin{remark}
    For the algebraically-minded, there is a more natural construction of $ \Lambda_{k} $ by viewing the ring as the colimit of a certain directed system of injections of symmetric polynomial rings
    \[
      k[x_{1},\ldots,x_{n}]^{S_{n}} \xhookrightarrow{\phi_{n}} k[x_{1},\ldots,x_{n+1}]^{S_{n+1}}.
    \]
    The construction of these maps $ \phi_{n} $ is somewhat involved. This does justify the intuition that a symmetric function is simply taking a symmetric polynomial and adding more data, as any element of a direct limit of inclusions is faithfully represented by an element of one of the constituent objects.

\end{remark}
\begin{definition}
    A symmetric function $ f\in \Lambda_{k} $ is homogenous of degree $ n $ if 
    \[
      f(x) = \sum_{\alpha\text{ weak composition of }n}c_{\alpha} x^{\alpha},
    \]
    where the $ c_{\alpha} $ are elements of $ k $. The set of degree $ n $ homogenous symmetric functions is denoted $ \Lambda_{k}^{n} $. these subspaces give $ \Lambda_{k} $ the structure of a graded $ k $-algebra, namely:
\end{definition}

\begin{itemize}
  \item Each $ \Lambda_{k}^{n} $ is a $ k $-vector space,
  \item $ \Lambda_{k}^{i} \Lambda_{k}^{j}\sub \Lambda_{k}^{i+j} $,
  \item $ \Lambda_{k}=\bigoplus_{n=0}^{\infty} \Lambda_{k}^{n} $ as $ k $-vector spaces.
\end{itemize}
The first interesting basis of $ \Lambda_{k} $ is the \textit{monomial symmetric functions}. Given $ \lambda\vdash n $, define $ m_{\lambda}\in \Lambda_{k}^{n} $ by 
\[
  m_{\lambda}:= \sum_{\alpha} x^{\alpha}
\]
where the sum is over all distinct permutations of the entries of $ \lambda $. The set $ \{m_{\lambda}: \lambda\vdash n\} $ forms a basis for $ \Lambda_{k}^{n} $, whence $\bigcup_{n\geq0} \{m_{\lambda}: \lambda\vdash n\} = \{m_{\lambda}: \lambda\in\Par\} $ forms a basis for $ \Lambda_{k} $.

\subsection{Complete Homogenous Symmetric Functions}

From the monomial symmetric functions, we may form another interesting basis for $ \Lambda_{k} $ called the \textit{complete homogenous symmetric functions} $ h_{\lambda} $ by setting 
\[
  h_{\lambda}:=\prod_{i=1}^{\infty} \sum_{\nu\vdash \lambda_{i}} m_{\nu}.
\]
where $ \lambda = (\lambda_{1},\lambda_{2},\ldots) $. Again, the set $ \{h_{\lambda}: \lambda\vdash n\} $ is a basis for $ \Lambda_{k}^{n} $ and the set $ \{h_{\lambda}: \lambda\in\Par\} $ is a basis for $ \Lambda_{k} $.
\subsection{Schur Functions}

\begin{definition}\label{def:youngdiagram}
  Given $ \lambda\vdash n $, the \textit{Ferrers diagram of shape $ \lambda $} is the set $\{(i,j)\in \N^{2} : j\in \N, 1\leq i \leq \lambda_{j}\}$ depicted as points in $ \R^{2} $. The \textit{Young diagram of shape $ \lambda $} is depicted identically to the Ferrers diagram except the points are replaced with squares. The \textit{size} of the diagram is the number of entries, namely $ n $. We depict the case $ (5,2,1)\vdash 8 $ below.
\end{definition}

\begin{figure}[h]
\centering
% Align tableaux at the top of their respective minipages
\ytableausetup{boxframe=0pt, aligntableaux=top}
\begin{minipage}[t]{0.20\textwidth}
\centering
\ydiagram[\bullet]{1,2,5}
% Adjust space below to align captions uniformly
\vspace{5pt} % Adjust the space to ensure alignment if necessary
\caption*{\textit{Ferrers Diagram} for the partition (5,2,1)}
\end{minipage}
\quad % This adds some space between the diagrams
\begin{minipage}[t]{0.20\textwidth}
\centering
\ytableausetup{boxframe=normal, aligntableaux=top}
\begin{ytableau}
*(white)\\
*(white) & *(white) \\
*(white) & *(white) & *(white) & *(white) & *(white) \\
\end{ytableau}
\vspace{5pt} % Keep consistent vertical space
\caption*{\textit{Young Diagram} for the partition (5,2,1)}
\end{minipage}
\quad
\begin{minipage}[t]{0.20\textwidth}
\centering
\begin{ytableau}
9 \\
5 & 5 \\
1 & 1 & 1 & 4 & 6 \\
\end{ytableau}
\vspace{5pt} % Adjust as needed
\caption*{\textit{Semi-standard Young tableau} for the partition (5,2,1)}
\end{minipage} 
\quad
\begin{minipage}[t]{0.20\textwidth}
\centering
\begin{ytableau}
8 \\
5 & 7 \\
1 & 2 & 3 & 4 & 6 \\
\end{ytableau}
\vspace{5pt} % Uniform spacing for alignment
\caption*{\textit{Standard Young tableau} for the partition (5,2,1)}
\end{minipage} 
\end{figure}


\begin{definition}\label{def:youngtableaux}
  Given $ \lambda\vdash n $ and a Young diagram of shape $ \lambda $, a \textit{semi-standard Young tableau of shape $ \lambda $} is a filling of the boxes of the Young diagram with positive integers such that 
  \begin{itemize}
    \item the entries are weakly increasing along rows,
    \item the entries are strictly increasing up columns.
  \end{itemize}
  A semi-standard Young tableau of size $ n $ is said to be \textit{standard} if the elements of $ \{1,\ldots,n\} $ each appear exactly once in the tableau. We write $ SSYT(\lambda) $ and $ SYT(\lambda) $ for the sets of semi-standard and standard Young tableaux of shape $ \lambda $. Given a semi-standard Young tableaux $ \mathcal{T} $, the \textit{weight} of $ \mathcal{T} $ is a function $ \alpha=\alpha_{\mathcal{T}}:\N \to \N$ given by
  \[
    \alpha(i) := \text{number of times $ i $ appears in $ \mathcal{T} $}.
  \]
  Note that $ \alpha(i) = 0 $ for sufficiently large $ i $, so we may write $ x^{\alpha} = x_{1}^{\alpha(1)}x_{2}^{\alpha(2)} \cdots $ and obtain a valid monomial. We write $ SSYT(\lambda,\alpha) $ for the set of semi-standard Young tableaux of shape $ \lambda $ and weight $ \alpha $.
\end{definition}


\begin{definition}\label{def:schur}
  Given $ \lambda\vdash n $, the \textit{Schur function} indexed by $ \lambda $ is defined by
 \[
  s_{\lambda} := \sum_{\mathcal{T}\in SSYT(\lambda) } x^{\alpha_{\mathcal{T}}} = \sum_{\alpha} \sum_{\mathcal{T}\in SSYT(\lambda,\alpha)} x^{\alpha}
\] 
where $ \alpha_{\mathcal{T}} $ denotes the weight of the Young tableau $ \alpha $.
\end{definition}

\begin{proposition}
  The Schur function $ s_{\lambda} $ is a symmetric function.
\end{proposition}


\begin{proof}
  Let $ \sigma\in S_{\infty}$. Then by definition of the action of $ S_{\infty} $ we have 
  \[
    \sigma \cdot s_{\lambda} = \sum_{\alpha} |SSYT(\lambda,\alpha)|\cdot \sigma\cdot x^{\alpha}= \sum_{\alpha} |SSYT(\lambda,\alpha)| x^{\sigma\alpha} = \sum_{\alpha} |SSYT(\lambda,\sigma\alpha)| x^{\alpha}.
  \]
  By the Bender-Knuth involution, we have that for fixed weak composition $ \alpha $, $ |SSYT(\lambda,\alpha)| = |SSYT(\lambda,\sigma \alpha)|$, whence 
  \[
    \sigma s_{\lambda} = \sum_{\alpha} |SSYT(\lambda, \sigma \alpha)|x^{\alpha} = \sum_{\alpha}|SSYT(\lambda,\alpha)|x^{\alpha} = s_{\lambda}.
  \]
\end{proof}


\end{document}
