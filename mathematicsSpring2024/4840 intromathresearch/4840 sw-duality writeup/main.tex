%! TEX root = ./main.tex
\documentclass[12pt]{article}

%--------Packages-------------
\usepackage{kyrem1sty}
%----------------------------


%--------Bibliography---------
\usepackage[backend=biber,style=alphabetic,doi=false,isbn=false,url=false,eprint=false]{biblatex}
\addbibresource{/home/kyrem1/Mathematics/bibs/comborefs.bib}
%----------------------------


%--------Hyper Setup-------
\hypersetup{%
  colorlinks=true,%
  linkcolor=blue,%
  citecolor=blue,%
  filecolor=blue,%
  menucolor=blue,%
  urlcolor=blue,%
  pdfnewwindow=true,%
  pdfstartview=FitBH
}   
%----------------------------


%--------Other Setup-------
\usepackage{ytableau} % For young diagrams
%\usepackage{todonotes}
%\newcommand{\Jenn}[1]{\todo[size=\tiny]{#1
%      \\ \hfill --- Jenn}}
\usepackage{caption}
% Adjust caption spacing and font
%\captionsetup{
%  font=small, % Adjust the font size
%  labelfont=bf,
%  format=plain, % Use plain format to avoid any unwanted effects
%  justification=raggedright, % Ensure the caption is justified, which can help with spacing
%  singlelinecheck=false, % Applies justification setting even when the caption is a single line
%}
%----------------------------


%--------Subfiles Setup-------
\usepackage{subfiles}
%----------------------------


%--------Page Setup-----------
%\usepackage{geometry}\geometry{margin=1in}
\pagestyle{empty}%

\setlength{\hoffset}{-1.54cm}
\setlength{\voffset}{-1.54cm}

\setlength{\topmargin}{0pt}
\setlength{\headsep}{0pt}
\setlength{\headheight}{0pt}

\setlength{\oddsidemargin}{0pt}

\setlength{\textwidth}{195mm}
\setlength{\textheight}{250mm}
%----------------------------


%--------Metadata------------
\title{Schur Weyl Duality and The Frobenius Formula}
\author{James Harbour}
%----------------------------


%--------Content-------------
\begin{document}
\maketitle

\begin{abstract}
  In this paper we exposit one of the fundamental results linking representation theory and algebraic combinatorics called Schur-Weyl duality. It provides a dictionary between the representation theory of finite symmetric groups and the representation theory of the general linear group of a finite dimensional complex vector space. Through this dictionary, we obtain representation theoretic constructions of many aspects of symmetric function theory, including Schur functions, Kostka numbers, and internal/external products on the symmetric function ring.
\end{abstract}

\tableofcontents



\subfile{./tex/repbackground.tex}


\subfile{./tex/symmetricgroupreps.tex}

\subfile{./tex/schurfunctors.tex}



\printbibliography

\end{document}
