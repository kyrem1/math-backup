%! TEX root = ../main.tex
\documentclass[../main.tex]{subfiles}


\begin{document}

\section{Representation Theory Background}

\subsection{Group Representations}

\begin{definition}
    A \textit{representation} of a group $ G $ is a pair $ (\pi, V) $ where $ V $ is a $ \C $-vector space and $ \pi: G\to \GL(V) $ is a group homomorphism.
\end{definition}

\begin{definition}
    A morphism between two representations $ (\pi, V) $ and $ (\rho, W) $ of a group $ G $ is a linear map $ T:V\to W $ such that $ T \pi(g) = \rho(g) T $ for all $ g\in G $.
\end{definition}



\subsection{Character Theory and Orthogonality Relations}



\subsection{Fundamental Examples}



\end{document}
