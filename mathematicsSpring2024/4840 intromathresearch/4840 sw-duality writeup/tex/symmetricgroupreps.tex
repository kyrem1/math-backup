%! TEX root = ../main.tex
\documentclass[../main.tex]{subfiles}


\begin{document}

\section{Representations of $ S_{n} $}
\subsection{Partitions, Young Diagrams, and Tabloids}
% TODO cut out parts of young diagram definitions that arent needed
To begin discussing the representation theory of $ S_{n} $, we first lay out some combinatorial groundwork with which to build the theory upon.
\begin{definition}\label{def:youngdiagram}
  Given $ \lambda\vdash n $, the \textit{Ferrers diagram of shape $ \lambda $} is the set $\{(i,j)\in \N^{2} : j\in \N, 1\leq i \leq \lambda_{j}\}$ depicted as points in $ \R^{2} $. The \textit{Young diagram of shape $ \lambda $} is depicted identically to the Ferrers diagram except the points are replaced with squares. The \textit{size} of the diagram is the number of entries, namely $ n $. We depict the case $ (5,2,1)\vdash 8 $ below.
\end{definition}

\begin{figure}[h]
\captionsetup{
  font=small, % Adjust the font size
  labelfont=bf,
  format=plain, % Use plain format to avoid any unwanted effects
  justification=raggedright, % Ensure the caption is justified, which can help with spacing
  singlelinecheck=false, % Applies justification setting even when the caption is a single line
}
\centering
% Align tableaux at the top of their respective minipages
\ytableausetup{boxframe=0pt, aligntableaux=top}
\begin{minipage}[t]{0.20\textwidth}
\centering
\ydiagram[\bullet]{1,2,5}
% Adjust space below to align captions uniformly
\vspace{5pt} % Adjust the space to ensure alignment if necessary
\caption*{\textit{Ferrers Diagram} for the partition (5,2,1)}
\end{minipage}
\quad % This adds some space between the diagrams
\begin{minipage}[t]{0.20\textwidth}
\centering
\ytableausetup{boxframe=normal, aligntableaux=top}
\begin{ytableau}
*(white)\\
*(white) & *(white) \\
*(white) & *(white) & *(white) & *(white) & *(white) \\
\end{ytableau}
\vspace{5pt} % Keep consistent vertical space
\caption*{\textit{Young Diagram} for the partition (5,2,1)}
\end{minipage}
\quad
\begin{minipage}[t]{0.20\textwidth}
\centering
\begin{ytableau}
9 \\
5 & 5 \\
1 & 1 & 1 & 4 & 6 \\
\end{ytableau}
\vspace{5pt} % Adjust as needed
\caption*{\textit{Semi-standard Young tableau} for the partition (5,2,1)}
\end{minipage} 
\quad
\begin{minipage}[t]{0.20\textwidth}
\centering
\begin{ytableau}
8 \\
5 & 7 \\
1 & 2 & 3 & 4 & 6 \\
\end{ytableau}
\vspace{5pt} % Uniform spacing for alignment
\caption*{\textit{Standard Young tableau} for the partition (5,2,1)}
\end{minipage} 
\end{figure}
\begin{definition}\label{def:lambdatableau}
  Given $ \lambda\vdash n $, a $ \lambda $-tableau is simply a filling of the boxes of the Young diagram of shape $ \lambda $ with the elements of $ \{1,\ldots,n\} $ without repetition (and no other restrictions). Denote the set of $ \lambda $-tableaux by $ YT(\lambda) $. Note that $ S_{n}\acts YT(\lambda) $ by permuting labels.
\end{definition}


\begin{definition}\label{def:youngtabloid}
  Given $ \lambda\vdash n $, define an equivalence relation $ \sim $ on $ YT(\lambda) $ by $ \mathcal{T}\sim \mathcal{T}^{\prime} $ if and only if $ \mathcal{T}^{\prime} $ can be obtained from $ \mathcal{T} $ by permuting the entries of each row. % TODO POSSIBLY COLUMNS CUZ FRENCH NOTATION
 An equivalence class with respect to this relation is called a \textit{$ \lambda $-tabloid}. If $ \mathcal{T} $ is a $ \lambda $-tableau, we write $ \{\mathcal{T}\} $ for the corresponding $ \lambda $-tabloid. Finally, we write $ Tab(\lambda):= YT(\lambda)/\sim $ for the set of $ \lambda $-tabloids. Note that the action of $ S_{n} $ on $ \lambda $-tableaux descends to an action on $ \lambda $-tabloids.
\end{definition}

\begin{example}
Below, we show two equivalent and inequivalent $ (5,2,1) $-tableaux. The first two tableau are equal as tabloids, whereas the last is not equal to the first two.

\begin{figure}[h]
\captionsetup{
  font=small, % Adjust the font size
  labelfont=bf,
  format=plain, % Use plain format to avoid any unwanted effects
  justification=raggedright, % Ensure the caption is justified, which can help with spacing
  singlelinecheck=false, % Applies justification setting even when the caption is a single line
}
% Align tableaux at the top of their respective minipages
\centering
\ytableausetup{boxframe=normal, aligntableaux=top}
\[
\begin{ytableau}
8 \\
5 & 7 \\
1 & 2 & 3 & 4 & 6 \\
\end{ytableau}
\sim\quad
\begin{ytableau}
8 \\
7 & 5 \\
1 & 3 & 2 & 6 & 4 \\
\end{ytableau}
\not\sim\quad
\begin{ytableau}
8 \\
1 & 7 \\
5 & 2 & 3 & 4 & 6 \\
\end{ytableau}
\]
%\vspace{5pt} % Uniform spacing for alignment
\end{figure}


 
\end{example}



\begin{definition}\label{def:youngtableaux}
  Given $ \lambda\vdash n $ and a Young diagram of shape $ \lambda $, a \textit{semi-standard Young tableau of shape $ \lambda $} is a filling of the boxes of the Young diagram with positive integers such that 
  \begin{itemize}
    \item the entries are weakly increasing along rows,
    \item the entries are strictly increasing up columns.
  \end{itemize}
  A semi-standard Young tableau of size $ n $ is said to be \textit{standard} if the elements of $ \{1,\ldots,n\} $ each appear exactly once in the tableau. We write $ SSYT(\lambda) $ and $ SYT(\lambda) $ for the sets of semi-standard and standard Young tableaux of shape $ \lambda $. Given a semi-standard Young tableau $ \mathcal{T} $, the \textit{weight} of $ \mathcal{T} $ is a function $ \alpha=\alpha_{\mathcal{T}}:\N \to \N$ given by
  \[
    \alpha(i) := \text{number of times $ i $ appears in $ \mathcal{T} $}.
  \]
  Note that $ \alpha(i) = 0 $ for sufficiently large $ i $, so we may write $ x^{\alpha} = x_{1}^{\alpha(1)}x_{2}^{\alpha(2)} \cdots $ and obtain a valid monomial. We write $ SSYT(\lambda,\alpha) $ for the set of semi-standard Young tableaux of shape $ \lambda $ and weight $ \alpha $.
\end{definition}


%TODO examples of equivalent and inequivalent lambda-tableaux.


\subsection{Construction of Specht Modules}
Young diagrams will give projection operators $ P_{\lambda}:\C[S_{n}]\to\C[S_{n}] $ which commute with the action of $ S_{n} $, whence the image $ P_{\lambda}(\C[S_{n}]) $ gives a subrepresentation of the regular representation. These subrepresentations will end up being precisely the irreducible representations of $ S_{n} $. Throughout this section, $ \lambda\vdash n$ will be fixed.\\

\begin{definition}\label{def:rowandcolgroup}
  Given a $ \lambda $-tableau $ \mathcal{T} $, define the \textit{row group} $ R_{\mathcal{T}} $ to be the subgroup of $ S_{n} $ which permutes only the labels in the rows of $ \mathcal{T} $ and the \textit{column group} $ C_{\mathcal{T}} $ as the subgroup which permutes only the labels in the columns of $ \mathcal{T} $. 
\end{definition}
As an illustrative example for these groups, note that the equality below implies $ (57)(23)(64)\in R_{\mathcal{T}} $.
\begin{figure}[h]
\captionsetup{
  font=small, % Adjust the font size
  labelfont=bf,
  format=plain, % Use plain format to avoid any unwanted effects
  justification=raggedright, % Ensure the caption is justified, which can help with spacing
  singlelinecheck=false, % Applies justification setting even when the caption is a single line
}
% Align tableaux at the top of their respective minipages
\centering
\ytableausetup{boxframe=normal, aligntableaux=top}
\[
  (57)(23)(64)\cdot
\begin{ytableau}
8 \\
5 & 7 \\
1 & 2 & 3 & 4 & 6 \\
\end{ytableau}
=\quad
\begin{ytableau}
8 \\
7 & 5 \\
1 & 3 & 2 & 6 & 4 \\
\end{ytableau}
\]
%\vspace{5pt} % Uniform spacing for alignment
\end{figure}\\
Now we may define the \textit{Young row and column symmetrizers} in $ \C[S_{n}] $ by
\begin{equation}\label{eq:rowandcolsymm}
  a_{\mathcal{T}} := \sum_{\sigma\in R_{\mathcal{T}}} \sigma, \qquad b_{\mathcal{T}} := \sum_{\sigma\in C_{\mathcal{T}}} \sgn(\sigma) \sigma.
\end{equation}
Note that for $ \mathcal{T}\in YT(\lambda) $, the corresponding tabloid is precisely the orbit of $ \mathcal{T} $ under its row group, i.e. 
\[
  \{\mathcal{T}\} = R_{\mathcal{T}}\mathcal{T} = \{\sigma \mathcal{T}\in YT(\lambda) : \sigma\in R_{\mathcal{T}}\}.
\]
Now let $ M^{\lambda} $ be the free $ \C $-vector space over the set of $ \lambda $-tabloids. Extending the action $ S_{n}\acts Tab(\lambda) $ linearly to all of $ M^{\lambda} $, we obtain a $ \C[S_{n}] $-module structure on $ M^{\lambda} $. For $ \mathcal{T}\in YT(\lambda) $, the element $ e_{\mathcal{T}}\in M^{\lambda} $ given by
\[
  e_{\mathcal{T}} := b_{\mathcal{T}}\cdot \{\mathcal{T}\} = \sum_{\sigma\in C_{\mathcal{T}}} \sgn(\sigma) \{\sigma \mathcal{T}\}
\]
is called the \textit{polytabloid associated to $\mathcal{T}$}.
Let $ S^{\lambda} $ be the subspace of $ M^{\lambda} $ generated by all polytabloids, namely
\[
  S^{\lambda}:=\Span_{\C}\{e_{\mathcal{T}} : \mathcal{T}\in YT(\lambda)\}.
\]
\begin{claim}
  $ S^{\lambda} $ is a $ \C[S_{n}] $-submodule of $ M^{\lambda} $.
\end{claim}

\begin{proof}[Proof of Claim]
  Fix $ \sigma\in S_{n} $. We first show that $ C_{\sigma \mathcal{T}} = \sigma C_{\mathcal{T}} \sigma^{-1}$. Indeed, if $ T_{i} $ is the set of entries for the $ i $th column of $ \mathcal{T} $, then $ \sigma(T_{i}) $ is the entries for the $ i $th column of $ \sigma \mathcal{T} $. Now it suffices to note that $ \tau\in S_{n} $ stabilizes $ T_{i} $ if and only if $ \sigma \tau \sigma^{-1} $ stabilizes $ \sigma(T_{i}) $. Using this identity, we compute 
\begin{equation*}
  \sigma b_{\mathcal{T}} = \sum_{\gamma\in C_{\mathcal{T}}} \sgn(\gamma) \sigma \gamma \overset{\tau=\sigma \gamma \sigma^{-1}}{=} \sum_{\tau\in \sigma C_{\mathcal{T}} \sigma^{-1}} \sgn(\sigma^{-1} \tau \sigma) \tau \sigma = \sum_{\tau\in C_{\sigma\mathcal{T}}} \sgn(\tau)  \tau \sigma  = b_{\sigma \mathcal{T}} \sigma.
\end{equation*}
Now we apply $ \sigma $ to the generators of $ S^{\lambda} $ and find
\begin{equation*}
  \sigma \cdot e_{\mathcal{T}} = \sigma \cdot (b_{\mathcal{T}} \cdot\{\mathcal{T}\}) = (\sigma b_{\mathcal{T}}) \cdot \{\mathcal{T}\} = b_{\sigma \mathcal{T}} \{\sigma \mathcal{T}\} = e_{\sigma \mathcal{T}}.
\end{equation*}
As $ S_{n} $ stabilizes $ S^{\lambda} $, the claim follows. 
\end{proof}


\begin{definition}\label{def:spechtmodule}
  The $ \C[S_{n}] $-module $ S^{\lambda} $ as defined above is the \textit{Specht module corresponding to $ \lambda $}.
\end{definition}


\begin{example}[Sign Representation]
  Consider the partition $ \lambda=(1,1,\ldots, 1) $ of $ n $. Since each row of $ \lambda $ has one element, the $ \lambda $-tabloids are the same as $ \lambda $-tableaux.\\

  Let $ \mathcal{T} $ be a $ \lambda $-tableau. As $ \mathcal{T} $ has only one column, $ C_{\mathcal{T}} = S_{n} $, whence $ b_{\mathcal{T}} = \sum_{\gamma\in S_{n}} \sgn(\gamma) \gamma $ and consequently
  \[
    \sigma e_{\mathcal{T}} = \sum_{\gamma\in S_{n}} \sgn(\gamma) \sigma \gamma \{\mathcal{T}\} = \sum_{\tau\in S_{n}} \sgn(\sigma^{-1} \tau) \tau \{\mathcal{T}\} = \sgn(\sigma)e_{\mathcal{T}} \quad \text{for all } \sigma\in S_{n}.
  \]
  On the other hand, we know that $ \sigma e_{\mathcal{T}} = e_{\sigma \mathcal{T}} $, so it follows that $ S^{\lambda} = \C e_{\mathcal{T}} $ is the one-dimensional $ \sgn $ representation.
\end{example}

\begin{example}[Trivial Representation]
  Consider the partition $ \lambda=(n) $ of $ n $. Since there is one row of $ \lambda $, all $ \lambda $-tableaux are equivalent so there is only one $ \lambda $-tabloid. Fix a $ \lambda $-tableau $ \mathcal{S} $.
  
  Each $ e_{\mathcal{T}} = \{T\} = \{S\} $, so $ S^{\lambda} = \C e_{\mathcal{S}} $ is one-dimensional. The action of $ \sigma $ is given by $ \sigma e_{\mathcal{T}} = e_{\sigma \mathcal{T}} = e_{\mathcal{T}} $, so $ S^{\lambda} $ is the trivial representation of $ S_{n} $.

\end{example}

\begin{example}[Augmentation Subrepresentation]
  Consider the partition $ \lambda=(n-1,1) $ of $ n $. Observe that there are $ n $ distinct $ \lambda $-tabloids, each corresponding to the integer in singular box on the $ 2 $nd row. Denote the tabloid with $ i $ in the $ 2nd $ row by $ t_{i} $, so $ Tab(\lambda) = \{t_{1},\ldots, t_{n}\} $.

  Let $ V=\C\{v_{1}, \ldots, v_{n}\} $ be the standard representation of $ S_{n} $ (i.e. $ \sigma v_{i} = v_{\sigma(i)} $). Observe that the map $ L:V\to M^{\lambda}$ given by $L(v_{i})=t_{i}$ is an isomorphism of $ \C[S_{n}] $-modules. The \textit{augmentation subrepresentation} $ W $ of $ V $ is given by $ W:= \{ \sum_{i=1}^{n} \alpha_{i} v_{i}: \sum_{i} \alpha_{i}=0 \}$. We claim that $ S^{\lambda}\cong W $ as $ \C[S_{n}] $-modules.
  Fix $ i\in \{1,\ldots,n\} $ and let $ \mathcal{T} $ be a $ \lambda $-tableau such that $ t_{i} = \{\mathcal{T}\} $.  Let $ j $ be the integer below $ i $ on the tableau.
  \begin{figure}[h]
    \centering % This command centers the figure
    \ytableausetup{boxsize=1.5em, boxframe=normal, aligntableaux=top}
    \begin{ytableau}
      i  & \none   & \none & \none & \none \\
      j  & *(white)   & \none[\dots] & *(white) & *(white) \\
    \end{ytableau}
    \caption*{General form of $ \mathcal{T} $ when $ t_{i}=\{\mathcal{T}\} $} % Add a caption if needed
  \end{figure}
  Then the column group $ C_{\mathcal{T}} $ is then of order $ 2 $ generated by the transposition $ (i\ j) $. 
  \begin{align*}
    e_{\mathcal{T}} = \sum_{\gamma\in C_{\mathcal{T}}} \sgn(\gamma) \gamma t_{i} = t_{i} - t_{j}.
  \end{align*}
  Hence, one checks
  \[
    S^{\lambda} = \Span\{t_{i}-t_{j}: 1\leq i, j \leq n, i\neq j \} = \Span\{t_{i}-t_{i+1}: 1\leq i\leq n-1\}.
  \]
  Moreover, $ \{t_{i}-t_{i+1} : 1\leq i \leq n-1\} $ gives a basis for $ S^{\lambda} $. The restriction of $ L $ to $ W $ gives a vector space isomorphism $ L:W\to S^{\lambda} $ as $ \{v_{i}-v_{i+1}\}_{1\leq i \leq n-1} $ gives a basis for $ W $, so a basis gets mapped to a basis. Moreover, this map intertwines the $ S_{n} $-action, so it produces $ \C[S_{n}] $-module isomorphism.
\end{example}

\subsection{Alternative Construction}\label{sec:altconst}

More algebraically-minded sources on the representation theory of $ S_{n} $ will take an alternative approach to the construction of Specht modules which may elucidate some details to the intended audience and obfuscate some details from others. We present this alternative construction for personal edification as well as to appeal to the former audience.

Fix a $ \lambda $-tableau $ \mathcal{S} $ throughout this section, say the canonical one (increasing across rows and then moving up rows).
Recall the row and column symmetrizers $ a_{\lambda}:=a_{\mathcal{S}} $, $ b_{\lambda}:=b_{\mathcal{S}} $ and define the Young symmetrizer 
\[
  c_{\lambda} := a_{\lambda}\cdot b_{\lambda}\in \C[S_{n}].
\]
Set $ V_{\lambda} := \C[S_{n}]c_{\lambda} $. Define a map $ T:\C[S_{n}]a_{\lambda} \to M^{\lambda} $ by $ T(\sigma a_{\lambda})= \{\sigma \mathcal{S}\} $. 
%\begin{theorem}
%  Letting $ S^{\lambda} $ be the Specht module as defined in the previous section, we have $ S^{\lambda}\cong V_{\lambda} = \C[S_{n}]c_{\lambda} $ as $ \C[S_{n}] $-modules.
%\end{theorem}
%
%\begin{proof}
%  $M^{\lambda} =  \C\{Tab(\lambda)\} = \{t: t\in Tab(\lambda)\} $, $ Y^{\lambda}:= \C\{YT_{\lambda}\} = \C\{\mathcal{T}: \mathcal{T}\in YT(\lambda)\}$.
%
%  \[\dim_{\C} Y^{\lambda} = |YT(\lambda)| = n!\]
%  Note that elements of $ YT(\lambda) $ are in one-to-one correspondence with elements of $ S_{n} $ by defining $ \sigma_{\mathcal{T}}(i) $ to be the value in the $ i $th box of $ \mathcal{T} $.
%
%  Define $ L: Y^{\lambda}\to M^{\lambda} $ by $ L(\mathcal{T}) = \{\mathcal{T}\} $. 
%
%  Define $ R: M^{\lambda}\to Y^{\lambda}$ by $ R(t) = \sum_{\mathcal{T}\in t} \mathcal{T} = \sum_{\sigma\in R_{\mathcal{T}}} \sigma \mathcal{T} $.
%
%  As $ S^{\lambda} $ is cyclic, $ S^{\lambda}=\C[S_{n}] \cdot e_{\{\mathcal{S}\}}\sub M^{\lambda}$.
%  \begin{align*}
%    R(e_{\{\mathcal{S}\}}) &= \sum_{\tau\in C_{\mathcal{S}}}\sgn(\tau)R(\{\tau\mathcal{S}\}) = \sum_{\tau\in C_{\mathcal{S}}} \sgn(\tau) \sum_{\sigma\in R_{\tau\mathcal{S}}} \sigma \tau\mathcal{T} \\
%    &= \sum_{\tau\in C_{\mathcal{S}}} \sgn(\tau) \sum_{\sigma\in \tau R_{\mathcal{S}} \tau^{-1}} \sigma \tau\mathcal{T} \\
%    &= \sum_{\tau\in C_{\mathcal{S}}} \sgn(\tau) \sum_{\gamma\in R_{\mathcal{S}}} \tau \gamma \mathcal{T} = b_{\lambda}a_{\lambda} \mathcal{T} \\
%  \end{align*}
%  If $ \sigma\in S_{n} $, then 
%  \begin{align*}
%    R(\sigma e_{\mathcal{S}}) &= R(e_{\sigma \mathcal{S}}) = \sum_{\tau\in C_{\sigma\mathcal{S}}} \sgn(\tau) \sum_{\gamma\in R_{\sigma\mathcal{S}}} \tau \gamma \mathcal{T} \\
%    &= \sum_{\tau\in \sigma C_{\mathcal{S}} \sigma^{-1}} \sgn(\tau) \sum_{\gamma\in \sigma R_{\mathcal{S}} \sigma^{-1}} \tau \gamma \mathcal{T} \\
%    &= \sum_{\widetilde{\tau}\in C_{\mathcal{S}} } \sgn(\widetilde{\tau}) \sum_{\widetilde{\gamma}\in R_{\mathcal{S}} } \sigma \widetilde{\tau} \sigma^{-1} \sigma \widetilde{\gamma} \sigma^{-1} \mathcal{T} \\
%    &=\sigma \sum_{\widetilde{\tau}\in C_{\mathcal{S}} } \sgn(\widetilde{\tau}) \sum_{\widetilde{\gamma}\in R_{\mathcal{S}} }  \widetilde{\tau}  \widetilde{\gamma} \sigma^{-1} \mathcal{T} = \sigma b_{\lambda}a_{\lambda} \sigma^{-1} \mathcal{T}\\
%  \end{align*}
%
%\end{proof}
%

%Define $ R: M^{\lambda}\to \C[S_{n}]a_{\lambda}$ by $ R(t) =  \sum_{\sigma\in R_{\mathcal{T}}} \sigma  $ where $ t = \{\mathcal{T}\} $. Note that if $ \mathcal{T}\sim \mathcal{T}^{\prime} $, then $ R_{\mathcal{T}} = R_{\mathcal{T}^{\prime}} $ so this map is well-defined. 
%\begin{claim}
%  The map $ R $ is an isomorphism of $ \C[S_{n}] $-modules.
%\end{claim}
%
%\begin{proof}[Proof of Claim]
%  For $ \mathcal{T}\in YT(\lambda) $, there is some $ \tau\in S_{n} $ such that $ \mathcal{T} = \tau \mathcal{S} $. Then, we compute
%  \[
%    R(\{T\}) = \sum_{\sigma\in R_{\mathcal{T}}} \sigma = \sum_{\sigma\in R_{\tau \mathcal{S}}} \sigma  = \sum_{\gamma\in R_{S}} \tau \gamma \tau^{-1} = \tau a_{\lambda} \tau^{-1}
%  \]
%\end{proof}
%
%
%\begin{claim}
%  $ R(S^{\lambda}) = \C[S_{n}]c_{\lambda} = V_{\lambda} $. 
%\end{claim}
%
%\begin{proof}[Proof of Claim]
%  \begin{align*}
%    R(e_{\{\mathcal{S}\}}) &= \sum_{\tau\in C_{\mathcal{S}}}\sgn(\tau)R(\{\tau\mathcal{S}\}) = \sum_{\tau\in C_{\mathcal{S}}} \sgn(\tau) \sum_{\sigma\in R_{\tau\mathcal{S}}} \sigma \\
%    &= \sum_{\tau\in C_{\mathcal{S}}} \sgn(\tau) \sum_{\sigma\in \tau R_{\mathcal{S}} \tau^{-1}} \sigma  \\
%    &= \sum_{\tau\in C_{\mathcal{S}}} \sgn(\tau) \sum_{\gamma\in R_{\mathcal{S}}} \tau \gamma \tau^{-1}  \\
%    &= \sum_{\tau\in C_{\mathcal{S}}} \sgn(\tau) \tau\left(\sum_{\gamma\in R_{\mathcal{S}}}  \gamma \right) \tau^{-1} \\
%    &= \sum_{\tau\in C_{\mathcal{S}}} \sgn(\tau) \tau a_{\lambda} \tau^{-1} \\
%  \end{align*}
%\end{proof}


\begin{claim}
  The map $ T $ is an isomorphism of $ \C[S_{n}] $-modules.
\end{claim}

\begin{proof}[Proof of Claim]
  We first show this map is well defined. If $ \sigma a_{\lambda} = \tau a_{\lambda} $, then $ \tau^{-1}\sigma $ fixes $ a_{\lambda} $, whence $ \tau^{-1} \sigma \in R_{\mathcal{S}} $ and consequently $ \sigma \{\mathcal{S}\} = \tau\{\mathcal{S}\}$.

  Since the action of $ S_{n} $ on $ \lambda $-tableau is transitive, it follows that the map $ T $ is onto. On the other hand, suppose $ \sum_{\sigma} \alpha_{\sigma}\sigma a_{\lambda} \in \ker(T) $. Then
  \[
    0 = T(\sum_{\sigma} \alpha_{\sigma} \sigma a_{\lambda}) = \sum_{\sigma} \alpha_{\sigma} \{\sigma \mathcal{S}\}.
  \]
  Since $ M^{\lambda} $ is a free $ \C $-module, it follows that $ \sum_{\sigma} \alpha_{\sigma} \sigma = 0 $. Lastly, if $ \sigma, \gamma\in S_{n} $, then
  \[
    \sigma T(\gamma a_{\lambda}) = \sigma\{\gamma \mathcal{S}\} = \{\sigma \gamma \mathcal{S}\} = T(\sigma \gamma s_{\lambda}).
  \]

\end{proof}

\begin{claim}
  The map $ T $ restricted to the submodule $ \C[S_{n}]b_{\lambda} a_{\lambda} $ gives a $ \C[S_{n}] $-module isomorphism $\C[S_{n}]b_{\lambda} a_{\lambda}\cong S^{\lambda} $.
\end{claim}

\begin{proof}[Proof of Claim]
  For $ \sigma\in S_{n} $, we compute
  \begin{align*}
    T(\sigma b_{\lambda} a_{\lambda}) &= \sum_{\tau\in C_{\mathcal{S}}} \sgn(\tau) T(\sigma \tau a_{\lambda}) =  \sum_{\tau\in C_{\mathcal{S}}} \sgn(\tau) \{\sigma \tau \mathcal{S}\} \\
    &= \sigma\sum_{\tau\in C_{\mathcal{S}}} \sgn(\tau) \{\tau \mathcal{S}\} = \sigma e_{\mathcal{S}} = e_{\sigma \mathcal{S}}\\
  \end{align*}
  Since $ S_{n} $ acts transitively on $ \lambda $-tableaux, it follows that
  \[
    T(\C[S_{n}]b_{\lambda}a_{\lambda}) = \Span_{\C}\{e_{\sigma \mathcal{S}}: \sigma\in S_{n}\} = S^{\lambda}
  \]
  By the proof of the previous claim, $ T $ is injective and intertwines the action of $ S_{n} $, whence $ T\vert_{\C[S_{n}]b_{\lambda}a_{\lambda}} $ furnishes an isomorphism of $ \C[S_{n}] $-modules as desired.
\end{proof}


\begin{proposition}
  $ \C[S_{n}]b_{\lambda}a_{\lambda}\cong \C[S_{n}]a_{\lambda}b_{\lambda} $.
\end{proposition}
%Note that $ R_{\mathcal{S}}\cap C_{\mathcal{S}} =\{1\} $, so decomposition of elements $\gamma\in R_{\mathcal{S}}C_{\mathcal{S}} $ into products $\gamma= \sigma \tau $ with $ \sigma\in R_{\mathcal{S}} $ and $ \tau\in C_{\mathcal{S}} $ is unique. For Hence,
%\[
%c_{\lambda} = \sum_{\sigma\in R_{\mathcal{S}}} \sigma\sum_{\tau\in C_{\mathcal{S}}}  \sgn(\tau) \tau=\sum_{\sigma\in R_{\mathcal{S}}, \tau\in C_{\mathcal{S}}}  \sgn(\tau) \sigma\tau = \sum_{\gamma\in R_{\mathcal{S}} C_{\mathcal{S}}} \sgn(\tau_{\gamma}) \gamma,
%\]
%\[
%  c_{\lambda} = 
%\]
%where $ \tau_{\gamma} $ is the unique element of $ C_{\mathcal{S}} $ in the product decomposition of $ \gamma $.

\subsection{Results on Specht Modules}
Having obtained a few examples of Specht modules, we note that $ \{S^{\lambda}: \lambda\vdash n\} $ forms a complete set of non-isomorphic, irreducible representations of $ S_{n} $. This is established by the combining the following three theorems, which we leave unproven and cite standard references \cite{fultonharris}, \cite{stanley2}.
\begin{theorem}
  Given $ \lambda\vdash n $, the Specht module $ S^{\lambda} $ is irreducible as a $ \C[S_{n}] $-module (i.e. an irreducible representation of $ S_{n} $).
\end{theorem}

\begin{theorem}
  If $ \lambda,\mu\vdash n $ and $ \lambda\neq \mu $, then $ S^{\lambda}\not\cong S^{\mu} $ as $ \C[S_{n}] $-modules.
\end{theorem}

\begin{theorem}
  Every irreducible representation of $ S_{n} $ is ismorphic to $ S^{\lambda} $ for some $ \lambda\vdash n $.  
\end{theorem}



\end{document}
