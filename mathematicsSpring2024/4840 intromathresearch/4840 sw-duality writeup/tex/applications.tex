%! TEX root = ../main.tex
\documentclass[../main.tex]{subfiles}


\begin{document}

\section{Applications}

\subsection{Kronecker Multiplication}
The first step is to work over only with representations over $ \C $, since then isomorphism classes of representations are determined entirely by their characters. For a finite group $ G $, we may consider the representation ring of $ G $ over $ \C $, $ R_\C(G) $. If $  V_{1},\ldots,V_{r}  $ are the irreducible complex representations of $ G $, then the representation ring of $ G $
\[
  R_{\C}(G) = \bigoplus_{i=1}^{r}\Z V_{i} = \bigoplus_{i=1}^{r}\Z\chi_{V_{i}}
\]
i.e. it is a free $ \Z $ module of rank $ r $ generated by the (isomorphism classes of the) irreducible representations (or their characters since we are over $ \C $). 

By the Specht module construction, we know that the irreducible representations of $ S_{n} $ are the Specht modules, so $ R_{\C}(S_{n}) \cong \bigoplus_{\lambda\vdash n}\Z S_{\lambda} $. The product structure in $ \R_{\C}(S_{n}) $ is determined by its generators, so consider two partitions $ \lambda,\mu\vdash n $. Since this ring is free, there exist $ g_{\lambda,\mu}^{\nu}\in \Z $ such that
\[
  S_{\lambda}\otimes S_{\mu} = \sum_{\nu\vdash n} g_{\lambda,\mu}^{\nu} S_{\nu}.
\]
These coefficients $ g_{\lambda,\mu}^{\nu} $ are called Kronecker coefficients and they are incredibly difficult to understand, hence ordinary multiplication in $ \R_{\C}(S_{n}) $ is difficult to understand. At the level of symmetric functions, this does induce a new product called the \textit{Kronecker product}
\[
  s_{\lambda}\star s_{\mu} = \sum_{\nu} g_{\lambda,\mu}^{\nu} s_{\nu}.
\]

One difficulty with this approach is that we are only looking at one graded piece of the ring of symmetric functions and trying to stay within this piece---a philosophy counter to that of the notion of grading. 

\subsection{Frobenius Characteristic Map}
The Frobenius characteristic map is what arises when one considers the entire graded ring of symmetric functions globally under this framework. It is a shadow of the behavior of the full \textit{graded} representation ring of all the groups $ \{S_{n}\}_{n=1}^{\infty} $. Consider the graded abelian group
\[
  R = \bigoplus_{n\geq0} R_{\C}(S_{n}) =\bigoplus_{n\geq0} \bigoplus_{\lambda\in \Par(n)} \Z S_{\lambda} = \bigoplus_{\lambda\in \Par} \Z S_{\lambda}
\]
with grading $ R_{n} = R_{\C}(S_{n}) $. We define a graded ring structure on $ R $ as follows. Let $ n,m\geq 0 $ and $\lambda\vdash n $, $ \mu\vdash m $. The product of $ S_{\lambda} $  and $ S_{\mu} $ needs to be a representation of $ S_{n+m} $. Since $ S_{\lambda} $ and $ S_{\mu} $ are a priori representations of different groups, we only have access to an external tensor product $ S_{\lambda}\boxtimes S_{\mu} $. The problem now is that this is a representation of $ S_{n}\times S_{m} $, not $ S_{n+m} $.

To fix this, note that we have a canonical inclusion $ S_{n}\times S_{m}\hookrightarrow S_{n+m} $. All we have to do now is consider the induced representation under this inclusion to obtain a representation of $ S_{n+m} $. Hence, we define the product in $ R $ by 
\[
  S_{\lambda}\bigcdot S_{\mu} := \Ind_{S_{n}\times S_{m}}^{S_{n+m}} (S_{\lambda}\otimes S_{\mu}) \in R_{\C}(S_{n+m})
\]
With some work, one can show that this in fact induces the ordinary product on the ring of symmetric functions. Moreover, via the map that one utilizes to show this fact, one can also show that the characters of $ S_{\lambda}(V) $ are precisely the Schur functions evaluated at the eigenvalues of the input matrix.


\end{document}
