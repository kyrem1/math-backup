%! TEX root = ./main.tex
\documentclass[12pt]{article}

%--------Packages-------------
\usepackage{kyrem1sty}
%----------------------------

%--------Bibliography---------
\usepackage[backend=biber,style=alphabetic,doi=false,isbn=false,url=false,eprint=false]{biblatex}
\addbibresource{/home/kyrem1/Mathematics/bibs/comborefs.bib}
%\addbibresource{comborefs.bib}
%----------------------------


%--------Hyper Setup-------
\hypersetup{%
  colorlinks=true,%
  linkcolor=blue,%
  citecolor=blue,%
  filecolor=blue,%
  menucolor=blue,%
  urlcolor=blue,%
  pdfnewwindow=true,%
  pdfstartview=FitBH
}   
%----------------------------


%--------Other Setup---------
\usepackage{ytableau} % For young diagrams
\usepackage{todonotes}
\newcommand{\Jenn}[1]{\todo[size=\tiny]{#1
      \\ \hfill --- Jenn}}
\usepackage{caption}
% Adjust caption spacing and font
\captionsetup{
  font=small, % Adjust the font size
  labelfont=bf,
  format=plain, % Use plain format to avoid any unwanted effects
  justification=raggedright, % Ensure the caption is justified, which can help with spacing
  singlelinecheck=false, % Applies justification setting even when the caption is a single line
}
%----------------------------


%--------Subfiles Setup-------
%\usepackage{subfiles}
%----------------------------


%--------Page Setup-----------
\usepackage{geometry}\geometry{margin=1in}
\pagestyle{empty}%

% \setlength{\hoffset}{-1.54cm}
% \setlength{\voffset}{-1.54cm}

% \setlength{\topmargin}{0pt}
% \setlength{\headsep}{0pt}
% \setlength{\headheight}{0pt}

% \setlength{\oddsidemargin}{0pt}

% \setlength{\textwidth}{195mm}
% \setlength{\textheight}{250mm}
%----------------------------
\setlength{\marginparwidth}{2cm}
\usepackage{todonotes}


%--------Metadata------------
\title{Talk 1 and Paper 1 Abstracts and Outlines }
\author{James Harbour}
%----------------------------


%--------Content-------------
\begin{document}

\maketitle


\section{Abstracts}
\subsection*{Talk Abstract}
``Groups, as men, shall be know by their actions'' -Guillermo Moreno. In this talk, we give an example based exploration of the field of representation theory and its relations to combinatorics. We hint towards a deep connection between the combinatorics of partitions and the actions of symmetric groups on vector spaces. The only background required will be linear algebra and some knowledge of group theory.


\subsection*{Paper Abstract}
In this paper we exposit one of the fundamental results linking representation theory and algebraic combinatorics called Schur-Weyl duality. It provides a dictionary between the representation theory of finite symmetric groups and the representation theory of the general linear group of a finite dimensional complex vector space. Through this dictionary, we obtain representation theoretic constructions of many aspects of symmetric function theory, including Schur functions, Kostka numbers, and internal/external products on the symmetric function ring.

\newpage
\section{Outlines}

\subsection{Talk Outline}
\begin{itemize}
  \item Run through the standard example $ \pi:  D_{2n} \to O(3) $
  \item Introduce definition of representations of finite groups.
  \item Get relevant defs to say $ V\otimes V \cong Sym^{2} \oplus \Lambda^{2}V$.
  \item State $ V\otimes V\otimes V \cong Sym^{3}V \oplus \Lambda^{3}V \oplus \text{ something else }$.
  \item Hint how that something else is related to the partition $ (2,1) $ of $ 3 $.
\end{itemize}

\subsection{Paper Outline}
\begin{itemize}
  \item Set up relevant preliminary representation theoretic definitions (group algebra stuff etc.
  \item Talk about the generic representation theory of $ S_{n} $ 
  \item Define Young symmetrizers and Specht Modules 
  \item Talk about the commuting left and right actions of $ GL(V) $ and $ S_{n} $ respectively.
  \item State and cite the double centralizer theorem to prove Schur-Weyl duality. 
  \item Obtain relations to Schur functions,  %\Jenn{I'm not sure what you have in mind here.  The prior sections look very good. I  think you're planning to show the applications in this section, but I'm having a hard time judging how broad the scope is.  If you say a little more about what you have in mind for LR coefficients and plethsym, I can weigh in.}
  \begin{itemize}
    \item Schur functions as characters of Schur functors (hints towards categorification). Namely, if $ g\in \GL(V) $ has eigenvalues $ \alpha_{1},\ldots,\alpha_{n} $, then 
      \[
        \chi_{\mathbb{S}_{\lambda}(V)}(g) = s_{\lambda}(\alpha_{1},\alpha_{2},\ldots, \alpha_{n},0,0,\ldots)
      \]
    \item Isometric isomorphism between the representation ring $ \mathcal{R}(S_{n}) $ (viewed as the $ \Z $-module generated by irreducible complex characters with multiplication defined by inducing representations along the inclusion $ S_{m}\times S_{n}\hookrightarrow S_{m+n} $ ) and $ \Z[x_{1},\ldots,x_{n}]^{S_{n}} $ under the Frobenius characteristic map 
      \[
        \mathrm{ch}(f) = \frac{1}{n!}\sum_{\lambda\vdash n} \frac{f(\lambda)}{z_{\lambda}}p_{\lambda}
      \]
      where $ p_{\lambda} $ denotes a power sum symmetric function and $ z_{\lambda} = \frac{n!}{K_{\lambda}} $ where $ K_{\lambda} $ is the size of the conjugacy class of $ S_{n} $ determined by $ \lambda $.
  \end{itemize}
\end{itemize}




\end{document}
