%! TEX root = ./main.tex
\documentclass[12pt]{article}

%--------Packages-------------
\usepackage{kyrem1sty}
%----------------------------


%--------Bibliography---------
%\usepackage[backend=biber,style=alphabetic,doi=false,isbn=false,url=false,eprint=false]{biblatex}
%\addbibresource{INSERT .BIB PATH}
%----------------------------


%--------Hyper Setup-------
\hypersetup{%
  colorlinks=true,%
  linkcolor=blue,%
  citecolor=blue,%
  filecolor=blue,%
  menucolor=blue,%
  urlcolor=blue,%
  pdfnewwindow=true,%
  pdfstartview=FitBH
}   
%----------------------------


%--------Subfiles Setup-------
%\usepackage{subfiles}
%----------------------------


%--------Page Setup-----------
%\usepackage{geometry}\geometry{margin=1in}
\pagestyle{empty}%

\setlength{\hoffset}{-1.54cm}
\setlength{\voffset}{-1.54cm}

\setlength{\topmargin}{0pt}
\setlength{\headsep}{0pt}
\setlength{\headheight}{0pt}

\setlength{\oddsidemargin}{0pt}

\setlength{\textwidth}{195mm}
\setlength{\textheight}{250mm}
%----------------------------


%--------Metadata------------
\title{Slide Talk Feedback}
\author{James Harbour}
%----------------------------


%--------Content-------------
\begin{document}

\begin{center}
  \Large\textit{An Investigation into the Relationships between Dinv, Area, and Bounce Statistics of Dyck Paths}
\end{center}


\subsection*{David}
\begin{itemize}
  \item Your announciation and pacing is great. You strike a great balance between enthusiasm and measuredness.
  \item I love the jovial comments/interjections, like when you asked the audience to say the Dyck word corresponding to a displayed path.
  \item Would have appreciated a bit more exposition on Dyck path to Tableau association.
\end{itemize}
\newpage
\begin{center}
  \Large\textit{An Investigation into the Relationships between Dinv, Area, and Bounce Statistics of Dyck Paths}
\end{center}


\subsection*{Charlie}
\begin{itemize}
  \item I like the placing of equidistribution of bounce and dinv very close to the start of the talk. Sets up a good expectation of the content going forwards.
  \item I am glad you went pretty liesurely through the example for area statistic vector.
  \item Thank you for giving an intuition ("jaggedness") behind Dinv, as the formal definition is a pretty unintuitive (not your fault)
  \item I'm not a fan of the amount of slide backscrolling you did when you were explaining bounce examples.
  \item the last example you gave was frickin awesome cuz the diagram has all the info on it. Your diagrams are awesome.
  \item Nice humor with the equidistribution for bounce-area theorem.
  \item Coin visualizations are amazing. Y'all are TikZ gods. It also really well explains the area only going up by at most 1 but going down by anything.
  \item Coin example got very fuzzy
\end{itemize}
\newpage
\begin{center}
  \Large\textit{An Investigation into the Relationships between Dinv, Area, and Bounce Statistics of Dyck Paths}
\end{center}




\subsection*{Brendan}
\begin{itemize}
  \item I like that you used prepared boardwriting to explain $ inv + area = \binom{n}{2} $, it helped get away from the monotony of slides.
  \item Sometimes you spoke a bit too fast for me to catch. Not that heinous but definitely noticable.
  \item I couldn't catch some things from the board since you erased them really fast.
  \item Your pace when you started on the conjectural slides part is great.
  \item You seem to have really good command over the broad strokes and history of the problem.
  \item I like that you show Insertion/Deletion doesnt work with an explicit counterexample, hints towards the true complexity of this problem.
  \item You are very well paced with going through slide examples, I was able to completely follow.
\end{itemize}

\newpage

\begin{center}
  \Large\textit{From Monomials to Grothendiecks}
\end{center}


\subsection*{Anthony}
\begin{itemize}
  \item I'm glad you started right off with the goal of the talk, and going through history. Made it very clear what's novel and what is not.
  \item Overview definitely helped me follow.
  \item You talked a bit fast for my ears. I understand this is probably needed as there are a bunch of definitions to get through.
  \item The pictorial parts of definitions helped me a lot with understanding.
\end{itemize}
\newpage

\begin{center}
  \Large\textit{From Monomials to Grothendiecks}
\end{center}


\subsection*{Samir}
\begin{itemize}
  \item Good pacing with definitions, not too fast and not too slow.
  \item It's nice that you mentioned and admitted that you didn't cover the inhomogeneity in your previous talk and emphasized it here.
  \item Initial elegant filling definition was a bit fast but the example cleared up my misunderstanding.
  \item Good question answers.
\end{itemize}
\newpage


\begin{center}
  \Large\textit{From Monomials to Grothendiecks}
\end{center}


\subsection*{Russell}
\begin{itemize}
  \item I like the idea of describing the objective by first going through an example computation.
  \item Diagrams in the computation were very instructive.
  \item The example gets pretty computational and not as instructive in the end.
  \item I was actually able to follow the definition of the map $ U\mapsto T $, which is crazy because I usually never understand combinatorial arguments in talks. So good job.
\end{itemize}
\newpage

\begin{center}
  \Large\textit{From Monomials to Grothendiecks}
\end{center}


\subsection*{Christopher}
\begin{itemize}
  \item Perfect pacing.
  \item Good call on description by example, I can imagine a full formal definition of this map would be very messy.
  \item I really liked you making the latter half of the example interactive. 
  \item I would have liked if you had more slides, I think you ended up getting much less time than the others.
\end{itemize}



\end{document}
