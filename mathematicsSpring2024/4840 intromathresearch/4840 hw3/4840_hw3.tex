%! TEX root = ./main.tex
\documentclass[12pt]{article}

%--------Packages-------------
\usepackage{kyrem1sty}
%----------------------------


%--------Bibliography---------
\usepackage[backend=biber,style=alphabetic,doi=false,isbn=false,url=false,eprint=false]{biblatex}
\addbibresource{/home/kyrem1/Mathematics/bibs/comborefs.bib}
%----------------------------


%--------Hyper Setup-------
\hypersetup{%
  colorlinks=true,%
  linkcolor=blue,%
  citecolor=blue,%
  filecolor=blue,%
  menucolor=blue,%
  urlcolor=blue,%
  pdfnewwindow=true,%
  pdfstartview=FitBH
}   
%----------------------------


%--------Other Setup---------
\usepackage{ytableau} % For young diagrams
\usepackage{caption}
% Adjust caption spacing and font
\captionsetup{
  font=small, % Adjust the font size
  labelfont=bf,
  format=plain, % Use plain format to avoid any unwanted effects
  justification=raggedright, % Ensure the caption is justified, which can help with spacing
  singlelinecheck=false, % Applies justification setting even when the caption is a single line
}
%----------------------------


%--------Subfiles Setup-------
%\usepackage{subfiles}
%----------------------------


%--------Page Setup-----------
%\usepackage{geometry}\geometry{margin=1in}
\pagestyle{empty}%

\setlength{\hoffset}{-1.54cm}
\setlength{\voffset}{-1.54cm}

\setlength{\topmargin}{0pt}
\setlength{\headsep}{0pt}
\setlength{\headheight}{0pt}

\setlength{\oddsidemargin}{0pt}

\setlength{\textwidth}{195mm}
\setlength{\textheight}{250mm}
%----------------------------


%--------Metadata------------
\title{Intro Math Research Hw3}
\author{James Harbour}
%----------------------------


%--------Content-------------
\begin{document}

\maketitle
\tableofcontents


\section{Reading Comments}

\subsection*{QUOMODOCUMQUE Article}


\subsection*{Tao Article}


\subsection*{Vakil Article}


\newpage



\section{Project Topics}
\textbf{Assignment}: Look over the Project Topics document and start searching for readable
resources. List the two topics which look most interesting to you and
give two references for each of these topics (different from the one I gave).
Keep in mind the remark of § 2.2

\subsection*{Topic 1: Schur-Weyl Duality and its Analogues}
\begin{itemize}
  \item In \cite[Ch.4,6]{fultonharris}, Fulton and Harris provide a gentle introduction to Schur-Weyl duality and the various identities which arise as a consequence.
  \item In \cite{dipper:05}, Dipper, Doty, and Hu obtain an analogue of Schur-Weyl duality for symplectic groups using the Brauer algebra. 

\end{itemize}



\subsection*{Topic 2: Quasisymmetric Functions}
\begin{itemize}
  \item The book \textit{An Introduction to Quasisymmetric {S}chur Functions} \cite{qsymschur} by Luoto, Mykytiuk and van Willigenburg gives a broad overview of the machinery behind quasisymmetric functions (namely Hopf algebras) as well as a description of the basis of quasisymmetric {S}chur functions for the quasisymmetric function algebra.
  \item In \cite{stanley:95}, Stanley contstructs a quasisymmetric function from a finite graph. This construction encodes the coloring data in a way that generalizes the graph chromatic polynomial. 
\end{itemize}



\newpage

\section{Symmetric Polynomials/Functions Exposition}

\textbf{Assignment}: Revise your write-up on symmetric polynomials, focusing on the extra tip
in § 2.1 (``indispensable, interesting, illustrative.'') If you did not write about the basis of homogeneous symmetric
functions last time, this should be included in your two pages. Add a third
page dedicated to the combinatorial definition of Schur functions and their
symmetry.

\subsection*{Preliminary Considerations} 
Throughout this article, fix a (unital) commutative ring $ R $ and a field $ k $. For simplicity, we work over vector spaces instead of general modules.

\begin{notation*}
  Let $ X $ be a set such that $ X = \{x_{i}\}_{i\in I} $ for some indexing set $ I $. By $ k[X] $ and $ k\llbracket X \rrbracket $, we denote the rings of (commutative) polynomials and power series (resp.) in indeterminates $ \{x_{i}\} $. We utilize multi-index notation throughout. Hence for $ \alpha_{\bigcdot}:I\to \N\cup\{0\} $ finitely supported, we write $ x_{\alpha} = \Pi_{i\in I}x_{i}^{\alpha_{i}} $ (where $ x_{i}^0 := 1 $ formally).
\end{notation*}

\subsection{Algebraic Background}

%Many common algebraic objects possess both a vector space structure and an internal product structure. For example, the set of $ n\times n $ matrices over $ k $, denoted $ M_{n}(k) $, has a product given by matrix multiplication and is also a vector space under addition and scalar multiplication by $ k $. 
%
%\begin{definition}\label{def:kalg}
%  Let $ A $ be a $ k $-vector space equipped with a map $ \cdot: A\times A\to A $ (written $ (x,y)\mapsto x\cdot y $). The pair $ (A,\cdot) $ is a \textit{$ k $-algebra} if, for $ x,y,z\in A $ and $ a,b\in k $, the following hold:
%  \begin{itemize}
%    \item $(x+y)\cdot z = x\cdot z + y\cdot z$,
%    \item $z\cdot (x+y) = z\cdot x + z\cdot y$,
%    \item $(a x) \cdot (b y) = (a b)(x\cdot y)$.
%  \end{itemize}
%\end{definition}
%%We abuse notation and often write $ xy $ for the product $ x\cdot y $. This is standard and usually clear from context. 
%%Observe that we did not assume the binary operation $ \cdot $ is associative. We will mainly focus on associative algebras; however, some non-associative examples (for culture) include Lie algebras, the Octonions, and $ \R^{3} $ equipped with the vector cross product.
%
%\begin{keyexample*}\label{ex:poly}
%  For $ X= \{x_{i}\}_{i\in I} $, the rings $ k[X] $ and $ k\llbracket X \rrbracket $ form the prototypical example of $ k $-algebras.
%\end{keyexample*}

Often in algebra, elements of a given object may be decomposed into a sum of simpler elements which are, in a sense, ``homogenous.'' For example, any polynomial in $ n $-variable may be decomposed into a sum of simpler polynomials each of which are futher sums of monomials of the same total degree.  In this way, a polynomial is split into a sum of homogenous parts. This behavior is codified in the notion of \textit{grading}.

\begin{definition}\label{def:gralg}
A \textit{graded $ k $-algebra} is a $ k $-algebra $ A $ together with a direct sum decomposition
\[
  A = \bigoplus_{i=0}^{\infty} A_{i}
\]
with $ A_{0},A_{1},\ldots $ vector spaces such that $ A_{i}\cdot A_{j}\sub A_{i+j} $ for all $ i,j\in \N\cup\{0\} $. For fixed $ i $, elements of $ A_{i} $ are called \textit{homogenous}. The choice of such a direct sum decomposition is a \textit{grading} for $ A $.
\end{definition}

%TODO insert exposition

\begin{keyexample*}
  As before, for $ X = \{x_{i}\}_{i\in I} $, we may give the ring $ k[X] $ a canonical grading by declaring $ A_{0}:=k $ and
  \[
    A_{n} := \Span_{k}\{x_{\alpha}: \alpha \text{ multi-index such that } \sum_{i\in I} \alpha_{i} = n\}.
  \]
\end{keyexample*} 
The reader is cautioned that not every $ k $-algebra has a nontrivial grading. In fact, it can be shown that the ring of formal power series $ k\llbracket x \rrbracket $ does not have a nontrivial grading.
%\begin{claim}
%  The algebra $ k\llbracket x \rrbracket $ does not possess a nontrivial grading.
%\end{claim}
%\begin{proof}[Proof of Claim]
%  Suppose 
%\end{proof}



\subsection{Symmetric Polynomials}

\begin{definition}\label{def:sympoly}
  The permutation group $ S_{n} $ acts naturally on the polynomial ring $ k[x_{1},\ldots,x_{n}] $ by defining $ \sigma \cdot x_{i_{1}}^{\alpha_1}\cdots x_{i_l}^{\alpha_{l}} := x_{\sigma(i_{1})}^{\alpha_{1}}\cdots x_{\sigma(i_{l})}^{\alpha_{l}}$ and extending by linearity. The ring of \textit{symmetric polynomials} in $ n $ indeterminates is the fixed points of this action, namely $ k[x_{1},\ldots, x_{n}]^{S_{n}} $.
\end{definition}

\subsection{Partitions and Compositions}
\begin{definition}\label{def:comp}\ 
  \begin{itemize}
    \item A \textit{partition} of $ n\in \N $ is a finite sequence $ \alpha=(\alpha_{1},\ldots, \alpha_{l}) $ of weakly decreasing positive integers which sum to $ n $. We denote the set of partitions of $ n $ by $ \Par(n) $. We denote the statement $ [\lambda\in \Par(n)] $ by $ \lambda\vdash n $. Also, we write $ \Par:=\bigcup_{n\geq0}\Par(n) $.
    \item A \textit{weak composition} of $ n\in \N $ is a (finitely supported) sequence $\alpha=(\alpha_{i})_{i=1}^{\infty}\in (\N\cup\{0\})^{\N}$ such that $ \sum_{i} \alpha_{i} = n $. The length of a weak composition $ \alpha $ is given by
  \[
    l(\alpha) :=\max\{i\in \N: \alpha_{i}\neq 0\}.
  \]
  \end{itemize}

  
\end{definition}

\subsection{Symmetric Functions}

\begin{definition}[pg. 308 in \cite{stanley2}]\label{def:symfunc}
    The ring $ \Lambda_{k} $ of symmetric functions over a field $ k $ is the subring of all $ f\in k\llbracket x_{1}, x_{2}, \ldots\rrbracket $ such that
    \[
      f(x_{\sigma(1)}, x_{\sigma(2)},\ldots) = f(x_{1},x_{2},\ldots) \text{ for all } \sigma\in \Sym(\N).
    \]
\end{definition}
\begin{remark}
    For the algebraically-minded, there is a more natural construction of $ \Lambda_{k} $ by viewing the ring as the colimit of a certain directed system of injections of symmetric polynomial rings
    \[
      k[x_{1},\ldots,x_{n}]^{S_{n}} \xhookrightarrow{\phi_{n}} k[x_{1},\ldots,x_{n+1}]^{S_{n+1}}.
    \]
    The construction of these maps $ \phi_{n} $ is somewhat involved. This does justify the intuition that a symmetric function is simply taking a symmetric polynomial and adding more data, as any element of a direct limit of inclusions is faithfully represented by an element of one of the constituent objects.

\end{remark}
\begin{definition}
    A symmetric function $ f\in \Lambda_{k} $ is homogenous of degree $ n $ if 
    \[
      f(x) = \sum_{\alpha\text{ weak composition of }n}c_{\alpha} x^{\alpha},
    \]
    where the $ c_{\alpha} $ are elements of $ k $. The set of degree $ n $ homogenous symmetric functions is denoted $ \Lambda_{k}^{n} $. these subspaces give $ \Lambda_{k} $ the structure of a graded $ k $-algebra, namely:
\end{definition}

\begin{itemize}
  \item Each $ \Lambda_{k}^{n} $ is a $ k $-vector space,
  \item $ \Lambda_{k}^{i} \Lambda_{k}^{j}\sub \Lambda_{k}^{i+j} $,
  \item $ \Lambda_{k}=\bigoplus_{n=0}^{\infty} \Lambda_{k}^{n} $ as $ k $-vector spaces.
\end{itemize}
The first interesting basis of $ \Lambda_{k} $ is the \textit{monomial symmetric functions}. Given $ \lambda\vdash n $, define $ m_{\lambda}\in \Lambda_{k}^{n} $ by 
\[
  m_{\lambda}:= \sum_{\alpha} x^{\alpha}
\]
where the sum is over all distinct permutations of the entries of $ \lambda $. The set $ \{m_{\lambda}: \lambda\vdash n\} $ forms a basis for $ \Lambda_{k}^{n} $, whence $\bigcup_{n\geq0} \{m_{\lambda}: \lambda\vdash n\} = \{m_{\lambda}: \lambda\in\Par\} $ forms a basis for $ \Lambda_{k} $.

\subsection{Complete Homogenous Symmetric Functions}

From the monomial symmetric functions, we may form another interesting basis for $ \Lambda_{k} $ called the \textit{complete homogenous symmetric functions} $ h_{\lambda} $ by setting 
\[
  h_{\lambda}:=\prod_{i=1}^{\infty} \sum_{\nu\vdash \lambda_{i}} m_{\nu}.
\]
where $ \lambda = (\lambda_{1},\lambda_{2},\ldots) $. Again, the set $ \{h_{\lambda}: \lambda\vdash n\} $ is a basis for $ \Lambda_{k}^{n} $ and the set $ \{h_{\lambda}: \lambda\in\Par\} $ is a basis for $ \Lambda_{k} $.

\subsection{Schur Functions}

\begin{definition}\label{def:youngdiagram}
  Given $ \lambda\vdash n $, the \textit{Ferrers diagram of shape $ \lambda $} is the set $\{(i,j)\in \N^{2} : j\in \N, 1\leq i \leq \lambda_{j}\}$ depicted as points in $ \R^{2} $. The \textit{Young diagram of shape $ \lambda $} is depicted identically to the Ferrers diagram except the points are replaced with squares. The \textit{size} of the diagram is the number of entries, namely $ n $. We depict the case $ (5,2,1)\vdash 8 $ below.
\end{definition}

\begin{figure}[h]
\centering
% Align tableaux at the top of their respective minipages
\ytableausetup{boxframe=0pt, aligntableaux=top}
\begin{minipage}[t]{0.20\textwidth}
\centering
\ydiagram[\bullet]{1,2,5}
% Adjust space below to align captions uniformly
\vspace{5pt} % Adjust the space to ensure alignment if necessary
\caption*{\textit{Ferrers Diagram} for the partition (5,2,1)}
\end{minipage}
\quad % This adds some space between the diagrams
\begin{minipage}[t]{0.20\textwidth}
\centering
\ytableausetup{boxframe=normal, aligntableaux=top}
\begin{ytableau}
*(white)\\
*(white) & *(white) \\
*(white) & *(white) & *(white) & *(white) & *(white) \\
\end{ytableau}
\vspace{5pt} % Keep consistent vertical space
\caption*{\textit{Young Diagram} for the partition (5,2,1)}
\end{minipage}
\quad
\begin{minipage}[t]{0.20\textwidth}
\centering
\begin{ytableau}
9 \\
5 & 5 \\
1 & 1 & 1 & 4 & 6 \\
\end{ytableau}
\vspace{5pt} % Adjust as needed
\caption*{\textit{Semi-standard Young tableau} for the partition (5,2,1)}
\end{minipage} 
\quad
\begin{minipage}[t]{0.20\textwidth}
\centering
\begin{ytableau}
8 \\
5 & 7 \\
1 & 2 & 3 & 4 & 6 \\
\end{ytableau}
\vspace{5pt} % Uniform spacing for alignment
\caption*{\textit{Standard Young tableau} for the partition (5,2,1)}
\end{minipage} 
\end{figure}


\begin{definition}\label{def:youngtableaux}
  Given $ \lambda\vdash n $ and a Young diagram of shape $ \lambda $, a \textit{semi-standard Young tableau of shape $ \lambda $} is a filling of the boxes of the Young diagram with positive integers such that 
  \begin{itemize}
    \item the entries are weakly increasing along rows,
    \item the entries are strictly increasing up columns.
  \end{itemize}
  A semi-standard Young tableau of size $ n $ is said to be \textit{standard} if the elements of $ \{1,\ldots,n\} $ each appear exactly once in the tableau. We write $ SSYT(\lambda) $ and $ SYT(\lambda) $ for the sets of semi-standard and standard Young tableaux of shape $ \lambda $. Given a semi-standard Young tableaux $ \mathcal{T} $, the \textit{weight} of $ \mathcal{T} $ is a function $ \alpha=\alpha_{\mathcal{T}}:\N \to \N$ given by
  \[
    \alpha(i) := \text{number of times $ i $ appears in $ \mathcal{T} $}.
  \]
  Note that $ \alpha(i) = 0 $ for sufficiently large $ i $, so we may write $ x^{\alpha} = x_{1}^{\alpha(1)}x_{2}^{\alpha(2)} \cdots $ and obtain a valid monomial.
\end{definition}


\begin{definition}\label{def:schur}
 \[
  s_{\lambda} := \sum_{\mathcal{T}\in SSYT(\lambda) } x^{\alpha_{\mathcal{T}}}.
\] 
\end{definition}




\end{document}
