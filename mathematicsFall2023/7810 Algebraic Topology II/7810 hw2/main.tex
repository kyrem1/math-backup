%! TEX root = ./main.tex
\documentclass[12pt]{article}


%--------Packages-------------
\usepackage{kyrem1sty}
%----------------------------


%--------Bibliography---------
%\usepackage[backend=biber]{biblatex}
%\addbibresource{INSERT .BIB PATH}
%----------------------------


%--------Subfiles Setup-------
%\usepackage{subfiles}
%----------------------------

%--------Page Setup-----------
%\usepackage{geometry}\geometry{margin=1in}
\pagestyle{empty}%

\setlength{\hoffset}{-1.54cm}
\setlength{\voffset}{-1.54cm}

\setlength{\topmargin}{0pt}
\setlength{\headsep}{0pt}
\setlength{\headheight}{0pt}

\setlength{\oddsidemargin}{0pt}

\setlength{\textwidth}{195mm}
\setlength{\textheight}{250mm}

% No indent
\setlength\parindent{0pt}

%----------------------------


%--------Metadata------------
\title{Algebraic Topology II Homework 2}
\author{James Harbour}
%----------------------------

%--------Assignment----------
% Hatcher 3.2 problems 1,3,7,8,9
%----------------------------



%--------Content-------------
\begin{document}
\maketitle

\begin{homeworkProblem}[1]
    Assuming as known the cup product structure on the $ 2 $-torus $ \T^{2} = S^{1}\times S^{1} $, compute the cup product structure in $ H^{*}(M_{g}) $  for $ M_{g} $ the closed orientable surface of genus $ g $ using the quotient map from $ M_{g} $ to a wedge sum of $ g $ tori (see photo).
\end{homeworkProblem}

\begin{proof}
    We take as known that the cohomology ring of $ \T^{2} $ with coefficients in a ring $ R $ is given by  the exterior algebra
    \[
        H^{*}(\T^{2} ; R) = \Lambda_{R}[ \alpha_{1}, \alpha_{2} ]
    \]
    Let $ H $

    % https://q.uiver.app/#q=WzAsNCxbMiwwLCJIXjIoXFxiaWd2ZWVfZyBcXG1hdGhiYntUfV4yKSJdLFsyLDIsIkheMihNX2cpIl0sWzAsMiwiSF4xKE1fZylcXG9wbHVzIEheMShNX2cpIl0sWzAsMCwiSF4xKFxcYmlndmVlX2cgXFxtYXRoYmJ7VH1eMikgXFxvcGx1cyBIXjEoXFxiaWd2ZWVfZyBcXG1hdGhiYntUfV4yKSJdLFswLDFdLFsyLDFdLFszLDBdLFszLDJdLFsyLDFdXQ==
\[\begin{tikzcd}
	{H^1(\bigvee_g \mathbb{T}^2) \oplus H^1(\bigvee_g \mathbb{T}^2)} && {H^2(\bigvee_g \mathbb{T}^2)} \\
	\\
	{H^1(M_g)\oplus H^1(M_g)} && {H^2(M_g)}
	\arrow[from=1-3, to=3-3]
	\arrow[from=3-1, to=3-3]
	\arrow[from=1-1, to=1-3]
	\arrow[from=1-1, to=3-1]
	\arrow[from=3-1, to=3-3]
\end{tikzcd}\]
\end{proof}

\begin{homeworkProblem}[3]
    \textbf{(a)}: Using the cup product structure, show that there is no map $ \R P^{n}\to  \R P^{m} $ inducing a nontrivial map $ H^{1}(\R P^{m} ; \Z_{2})\to H^{1}(\R P^{n} ; \Z_{2}) $ if $ n > m $. What is corresponding result for maps $ \C P^{n}\to \C P^{m} $.\\

    \textbf{(b)}: Prove the Borsuk-Ulam theorem. 
\end{homeworkProblem}

\begin{homeworkProblem}[7]
    Use cup products to show that $ \R P^{3} $ is not homotopy equivalent to $ \R P^{2}\vee S^{1} $.
\end{homeworkProblem}

\begin{homeworkProblem} 
Let $ X $ be $ \C P^{2} $ with a cell $ e^{3} $ attached by a map $ S^{2}\to \C P^{1} \sub \C P^{2}$ of degree $ p $, and let $ Y:= M(\Z_{p},2)) \vee S^{4} $ . Thus $ X $ and $ Y $ have the smae $ 3 $-skeleton but differ in the way their $ 4$-cells are attached. Show that $ X $ and $ Y $ have isomorphic cohomology rings with coefficients in $ \Z $, but not with $ \Z_{p} $ coefficients.
\end{homeworkProblem}

\begin{homeworkProblem}
    Show that if $ H_{n}(X ;\Z) $ free for each $ n $, then $ H^{*}(X ; \Z_{p}) $ and $ H^{*}(X ; \Z) \otimes \Z_{p} $ are isomorphic as rings, so in particular the ring structure with $ \Z $ coefficients determines the ring structure with $ \Z_{p} $ coefficients.
\end{homeworkProblem}


\end{document}
