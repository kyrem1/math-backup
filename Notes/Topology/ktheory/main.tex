%! TEX root = ./main.tex
\documentclass[12pt]{article}

%--------Packages-------------
\usepackage{kyrem1sty}
%----------------------------


%--------Bibliography---------
%\usepackage[backend=biber]{biblatex}
%\addbibresource{INSERT .BIB PATH}
%----------------------------


%--------Subfiles Setup-------
\usepackage{subfiles}
%----------------------------


%--------Page Setup-----------
%\usepackage{geometry}\geometry{margin=1in}
\pagestyle{empty}%

\setlength{\hoffset}{-1.54cm}
\setlength{\voffset}{-1.54cm}

\setlength{\topmargin}{0pt}
\setlength{\headsep}{0pt}
\setlength{\headheight}{0pt}

\setlength{\oddsidemargin}{0pt}

\setlength{\textwidth}{195mm}
\setlength{\textheight}{250mm}
%----------------------------


%--------Metadata------------
\title{Topological $ K $-Theory Talk Outline}
\author{James Harbour}
%----------------------------


%--------Content-------------
\begin{document}
\maketitle

\section{References}
\begin{itemize}
    \item Karoubi, \textit{K-theory: an introduction}
    \item \textcolor{blue}{\href{http://math.uchicago.edu/~may/REU2021/REUPapers/Martins.pdf}{UChicago Reu Papers}}
\end{itemize}

\section{Outline}
\begin{itemize}
    \item Define category $ Vect_{\F}(X) $ for compact Hausdorff space $ X $ and $ \F\in \{\R, \C\} $. 
    \item Define Whitney sum on $ \Vect_{\F}(X)/\sim $
    \item Define Grothendieck completion $ Gr: \bold{AbMon}\to\bold{Ab} $
        \[
            \Hom_{Ab}(Gr(M), A) \cong \Hom_{AbMon}(M, UA)
        \]

    \item Define $ K^{0}(X) $ (and maybe $ KO^{0}(X) $)
    \item Define $ \widetilde{K}^{0}(X) $ for pointed $ X $.
    \begin{itemize}
        \item $ \widetilde{K}^{0}(X/A) \to \widetilde{K}^{0}(X) \to \widetilde{K}^{0}(A) $
        \item $ K^{-n}(X) := \widetilde{K}^{0}(\Sigma^{n}X) $ for $ n\in\N $
    \end{itemize}

    \item Compute $ K^{0}(\{\text{pt}\}) $
    

    \item Talk about Bott Periodicity
    \item Talk about
        \[
            K^{0}(-) \simeq [-, BU\times \Z],\quad KO^{0}(-)\simeq [-, BO\times \Z]
        \]
    \item Talk about Serre-Swan Theorem to motivate $ K^{0}(R) $
        \[
            \Vect_{\C}(X) \simeq \{\text{f.g. projective } \C(X)\text{-modules}\}
        \]
    \item Mention that we dont have suspensions to define $ K^{n}(R) $ so its much harder.
\end{itemize}


\end{document}
