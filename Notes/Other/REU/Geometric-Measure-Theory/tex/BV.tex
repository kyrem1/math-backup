%! TEX root = ../main.tex
\documentclass[../main.tex]{subfiles}


\begin{document}

\section{Functions of Bounded Variation.}
\begin{definition}\label{def:BV}
    Given a function $ u\in L^{1}(\Omega) $, define the \textit{total variation of $ u $} to be the quantity
    \[
        V(u,\Omega) = \sup\left\{ \int_{\Omega} u\div{\phi}\dd{x} : \phi\in[C_{c}^{1}(\Omega,\R)]^{n}, \norm{\phi}_{\infty} \leq 1 \right\}.
    \]
    If $ V(u,\Omega) $ is finite, then we say that $ u $ is of \textit{bounded variation} and write $ u\in BV(\Omega) $. 
\end{definition}

\begin{definition}\label{def:BVloc}
    Similarly, given $ u\in L^{1}_{loc}(\Omega) $ and $ U\Subset \Omega $, define the \textit{local variation of $ u $ in $ U $} by
    \[
        V(u, U) = \sup\left\{ \int_{U} u\div{\phi}\dd{x} : \phi\in[C_{c}^{1}(U,\R)]^{n}, \norm{\phi}_{\infty} \leq 1 \right\}.
    \]
    We define the set of functions of \textit{locally bounded variation} to be
    \[
        BV_{loc}(\Omega) = \{u\in L^{1}_{loc}(\Omega) : V(u,U)< +\infty \text{ for all } U\Subset \Omega\}.
    \]
\end{definition}

An equivalent, and admittedly more transparent, characterization of $ BV_{loc} $ functions can be given as follows. 

\begin{proposition}[Characterization of $ BV_{loc} $]\label{bvchar}
    Suppose $ u\in BV_{loc}(\Omega) $. Then there exists a Radon measure $ \mu $ on $ \Omega $ and a $ \mu $-measurable $ \sigma:\Omega\to \R^{n} $ with $ | \sigma| = 1$ $ \mu $-a.e. and
    \[
        \int_{\Omega} u\div{\phi}\dd{x} = -\int_{\Omega} \phi\cdot \sigma \dd{\mu} \text{ for all } \phi\in C_{c}^{1}(\Omega,\R^{n}).
    \]
\end{proposition}

\begin{proof}
    This is a routine application of the Riesz–Markov–Kakutani representation theorem. To this end, define a linear functional $ L:C_{c}^{1}(\Omega,\R^{n}) \to \R $ by $ L(\phi) = -\int_{\Omega} u\div{\phi}\dd{x} $.
    
    For open $ U\Subset \Omega $, the quantity $ c(U):= \sup\{L(\phi) : \phi\in C_{c}^{1}(U,\R^{n}), \norm{\phi}_{\infty} \leq 1\} $ is finite by assumption, whence 
    \[
        |L(\phi)| \leq c(U) \norm{\phi}_{\infty} \text{ for all } \phi\in C_{c}^{1}(U,\R^{n}).
    \]
    Let $ K\sub \Omega $ be a fixed compact set, and choose open $ U\Subset\Omega $ containing $ K $. Then for $ \phi\in C_{c}(\Omega,\R^{n}) $ with $ \supp(\phi)\sub K $, there exists a sequence $ (\phi_{k})_{k} $ in $ C_{c}^{1}(U,\R^{n}) $ such that $ \phi_{k}\to \phi $ uniformly on $ U $. \\

    Define an extension $ \widetilde{L}:C_{c}(\Omega,\R^{n})\to \R $ of $ L $ by $ \widetilde{L}(\phi) = \lim_{k\to\infty}L(\phi_{k}) $, which exists and is well-defined by the above inequality. Applying the Riesz Representation Theorem to $ \widetilde{L} $ gives the conclusion.
\end{proof}

\begin{definition}
    For $ u\in BV_{loc}(\Omega) $, we will write $ \norm{Du} $ for the measure $ \mu $ and 
    \[
        d[Du] := \sigma \dd{\norm{Du}}, \text{ i.e } \int\cdot \dd{[Du]} = \int \inp{\cdot}{\sigma}\dd{\norm{Du}}.
    \]
    Then the conclusion of Proposition \ref{bvchar} can be rewritten as
    \[
        \int u\div{\phi} \dd{x} = -\int \phi \cdot \sigma \dd{\norm{Du}} = -\int \phi \cdot \dd{[Du]} \text{ for all } \phi\in C_{c}^{1}(\Omega,\R^{n}).
    \]
\end{definition}

Write $ \phi = (\phi^{1}, \ldots, \phi^{n})\in C_{c}^{1}(\Omega,\R^{n}) $.

\[
    [Du] = [Du]_{ac} + [Du]_{s}
\]



\section{Caccioppoli Sets (i.e. Sets of Locally Finite Perimeter)}

\begin{definition}
    Given a set $ E\sub \R^{n}$, we say that $ E $ is of \textit{locally finite perimeter in $ \Omega $} if $ \chi_{E}\in BV_{loc}(\Omega) $.
\end{definition}

%Suppose that $ u\in BV(\Omega) $. For $ 1\leq i\leq n $, we define a linear functional $ L_{i}:C_{c}^{1}(\Omega)\to \R $ by $ L_{i}(\psi) = -\int_{\Omega} u \pdv{\psi}{x_{i}}\dd{x}$.
%
%
%\[
%     -\int_{\Omega} u \pdv{\psi}{x_{i}}\dd{x} = \inp{\partial_{i}u}{\pi}
%\]
%
%Note that integration by parts gives t
%
%Then for $ \psi\in C_{c}^{1}(\Omega,\R) $.




\end{document}
